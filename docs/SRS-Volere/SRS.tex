% THIS DOCUMENT IS FOLLOWS THE VOLERE TEMPLATE BY Suzanne Robertson and James Robertson
% ONLY THE SECTION HEADINGS ARE PROVIDED
%
% Initial draft from https://github.com/Dieblich/volere
%
% Risks are removed because they are covered by the Hazard Analysis
\documentclass[12pt]{article}

\usepackage{booktabs}
\usepackage{tabularx}
\usepackage{hyperref}
\hypersetup{
    bookmarks=true,         % show bookmarks bar?
      colorlinks=true,      % false: boxed links; true: colored links
    linkcolor=red,          % color of internal links (change box color with linkbordercolor)
    citecolor=green,        % color of links to bibliography
    filecolor=magenta,      % color of file links
    urlcolor=cyan           % color of external links
}

\newcommand{\lips}{\textit{Insert your content here.}}

\input{../Comments}
%% Common Parts

\newcommand{\progname}{ScoreGen} % PUT YOUR PROGRAM NAME HERE
\newcommand{\authname}{Team \#7, Tune Goons
\\ Emily Perica
\\ Ian Algenio
\\ Jackson Lippert
\\ Mark Kogan} % AUTHOR NAMES                  

\usepackage{hyperref}
    \hypersetup{colorlinks=true, linkcolor=blue, citecolor=blue, filecolor=blue,
                urlcolor=blue, unicode=false}
    \urlstyle{same}
                                


\begin{document}

\title{Software Requirements Specification for \progname: subtitle describing software} 
\author{\authname}
\date{\today}
	
\maketitle

~\newpage

\pagenumbering{roman}

\tableofcontents

~\newpage

\section*{Revision History}

\begin{tabularx}{\textwidth}{p{3cm}p{2cm}X}
\toprule {\textbf{Date}} & {\textbf{Version}} & {\textbf{Notes}}\\
\midrule
Date 1 & 1.0 & Notes\\
Date 2 & 1.1 & Notes\\
\bottomrule
\end{tabularx}

~\\

~\newpage
\section{Purpose of the Project}
\subsection{User Business}
\lips
\subsection{Goals of the Project}
\lips
\section{Stakeholders}
\subsection{Client}
The client is the faculty supervisor of this project, Dr. Martin Von Mohrenschildt. With direct research interests in signal processing and a personal interest in how it relates to 
music, he is investing his own time in the development of the product.
\subsection{Customer}
The archetypal customer is a musician of any skill level, with emphasis based on musicians who have little-to-no music theory knowledge. The customer will utilize the application 
to quickly produce sheet music for either original compositions or existing songs. 
\subsection{Other Stakeholders}
\begin{itemize}
  \item \textbf{Software developers:} Provide the technical knowledge and skill to design and build the system.
  \item \textbf{Composers:} As SMEs, provide the music theory knowledge needed to make the system useful for the customer.
  \item \textbf{Users of music-annotating software:} As SMEs, provide opinions on the features they like or dislike about existing music-annotating softwares. This will help inform the developers on which features of the product to prioritize.
  \item \textbf{Music teachers:} Provide similar support to students that may be replaced by usage of this product. They will have a vested interest in ensuring this product does not encourage bad composing habits in users of the product.
  \item \textbf{Music students:} Similar to the Composers, they will provide the music theory knowledge needed to make the system useful for the customer. However, their input will be from the same learning point of view as the expected user, and thus will have a higher degree of influence on the product.
  \item \textbf{IP lawyers:} Will ensure this product follows fair use and does not infringe on copyrights.
  \item \textbf{Dr. Spencer Smith:} As the course instructor, the performance and outcome of this product will be a direct reflection of his role in the capstone course.
\end{itemize}
\subsection{Hands-On Users of the Project}
\subsubsection*{User Category 1: Beginner Musicians}
\textbf{User Role:} Use this product to create sheet music, either as a learning aid or as a quick alternative to learning music notation.\\
\textbf{Subject matter experience*:} Novice - Little experience with music theory, but likely have a moderate knowledge of music in practicality (i.e., playing an instrument or listening to music). \\
\textbf{Technological experience*:} Novice - May or may not have limited experience with other music-annotating softwares. \\
\textbf{User characteristics:} 
\begin{itemize}
  \item Little to no music theory knowledge
  \item Passionate about music
  \item All ages (i.e., both children and adults)
  \item New to playing instruments
  \item Not comfortable with complex software
  \item Eager to learn
  \item Less willing to spend money
\end{itemize}

\subsubsection*{User Category 2: Composers}
\textbf{User Role:} As masters of music notation and composition, this group will use the product as a practice aid or to quickly make note of composition ideas. \\
\textbf{Subject matter experience*:} Master - have spent years learning and practicing music theory and composition. \\
\textbf{Technological experience*:} Journeyman to Master - Have experience with music-annotating software but haven't necessarily used them extensively. \\
\textbf{User characteristics:}
\begin{itemize}
  \item Advanced music theory knowledge
  \item Passionate about music
  \item 25+ years old
  \item Have a hard time learning new technologies
  \item Willing to spend money
\end{itemize}
\noindent
\textit{*User experience level may be rated as one of novice, journeyman, or master.}

\subsection{Personas}
Harold Style is a software engineering student at McMaster University with a deep interest in music, and loves playing the guitar. Despite his lack of a formal music education, 
he has perfect pitch and thus learns songs by listening to them rather than learn how to read sheet music. However, he recently joined a local rock band called The Beetles who 
expect him to contribute during their songwriting sessions. He needs a quick way to produce sheet music for his own compositions, as well as learn to read the scores written by 
his bandmates. As a future software engineer he enjoys learning new technologies that can make his life easier, and he dislikes reading of any sort.
\subsection{Priorities Assigned to Users}
\textbf{Key users:} Beginner musicians
\textbf{Secondary users:} Composers.
\subsection{User Participation}
Beginner musicians will have a high level of participation in the creation of system requirements. Their feedback will be especially essential during the early to middle development phases to ensure that the system meets their various needs for simplicity, intuitive design/usage, and music theory assistance.
Composers will have a moderate level of participation in the creation of system requiremens, slightly less than that of the Beginner Musician group. Their subject matter expertise will be useful in creating relevant requirements to ensure precision of audio transcription and its overall score editing capabilities.
\subsection{Maintenance Users and Service Technicians}
This product will be maintained by a software development team, consisting of software developers, UI/UX designers, test engineers, and a project manager. 
Users will receive product support directly from the customer support team, and IP lawyers will be engaged as necessary to address matters of intellectual property.

\section{Mandated Constraints}
\subsection{Solution Constraints}
\lips
\subsection{Implementation Environment of the Current System}
\lips
\subsection{Partner or Collaborative Applications}
\lips
\subsection{Off-the-Shelf Software}
\lips
\subsection{Anticipated Workplace Environment}
\lips
\subsection{Schedule Constraints}
\lips
\subsection{Budget Constraints}
\lips
\subsection{Enterprise Constraints}
\lips

\section{Naming Conventions and Terminology}
\subsection{Glossary of All Terms, Including Acronyms, Used by Stakeholders
involved in the Project}
\lips

\section{Relevant Facts And Assumptions}
\subsection{Relevant Facts}
\lips
\subsection{Business Rules}
\lips
\subsection{Assumptions}
\lips

\section{The Scope of the Work}
\subsection{The Current Situation}
\lips
\subsection{The Context of the Work}
\lips
\subsection{Work Partitioning}
\lips
\subsection{Specifying a Business Use Case (BUC)}
\lips

\section{Business Data Model and Data Dictionary}
\subsection{Business Data Model}
\lips
\subsection{Data Dictionary}
\lips

\section{The Scope of the Product}
\subsection{Product Boundary}
\lips
\subsection{Product Use Case Table}
\lips
\subsection{Individual Product Use Cases (PUC's)}
\lips

\section{Functional Requirements}
\subsection{Functional Requirements}
\lips

\section{Look and Feel Requirements}
\subsection{Appearance Requirements}
\lips
\subsection{Style Requirements}
\lips

\section{Usability and Humanity Requirements}
\subsection{Ease of Use Requirements}
\lips
\subsection{Personalization and Internationalization Requirements}
\lips
\subsection{Learning Requirements}
\lips
\subsection{Understandability and Politeness Requirements}
\lips
\subsection{Accessibility Requirements}
\lips

\section{Performance Requirements}
\subsection{Speed and Latency Requirements}
\lips
\subsection{Safety-Critical Requirements}
\lips
\subsection{Precision or Accuracy Requirements}
\lips
\subsection{Robustness or Fault-Tolerance Requirements}
\lips
\subsection{Capacity Requirements}
\lips
\subsection{Scalability or Extensibility Requirements}
\lips
\subsection{Longevity Requirements}
\lips

\section{Operational and Environmental Requirements}
\subsection{Expected Physical Environment}
\lips
\subsection{Wider Environment Requirements}
\lips
\subsection{Requirements for Interfacing with Adjacent Systems}
\lips
\subsection{Productization Requirements}
\lips
\subsection{Release Requirements}
\lips

\section{Maintainability and Support Requirements}
\subsection{Maintenance Requirements}
\lips
\subsection{Supportability Requirements}
\lips
\subsection{Adaptability Requirements}
\lips

\section{Security Requirements}
\subsection{Access Requirements}
\lips
\subsection{Integrity Requirements}
\lips
\subsection{Privacy Requirements}
\lips
\subsection{Audit Requirements}
\lips
\subsection{Immunity Requirements}
\lips

\section{Cultural Requirements}
\subsection{Cultural Requirements}
\lips

\section{Compliance Requirements}
\subsection{Legal Requirements}
\lips
\subsection{Standards Compliance Requirements}
\lips

\section{Open Issues}
\lips

\section{Off-the-Shelf Solutions}
\subsection{Ready-Made Products}
\lips
\subsection{Reusable Components}
\lips
\subsection{Products That Can Be Copied}
\lips

\section{New Problems}
\subsection{Effects on the Current Environment}
\lips
\subsection{Effects on the Installed Systems}
\lips
\subsection{Potential User Problems}
\lips
\subsection{Limitations in the Anticipated Implementation Environment That May
Inhibit the New Product}
\lips
\subsection{Follow-Up Problems}
\lips

\section{Tasks}
\subsection{Project Planning}
\lips
\subsection{Planning of the Development Phases}
\lips

\section{Migration to the New Product}
\subsection{Requirements for Migration to the New Product}
\lips
\subsection{Data That Has to be Modified or Translated for the New System}
\lips

\section{Costs}
\lips
\section{User Documentation and Training}
\subsection{User Documentation Requirements}
\lips
\subsection{Training Requirements}
\lips

\section{Waiting Room}
\lips

\section{Ideas for Solution}
\lips

\newpage{}
\section*{Appendix --- Reflection}

\input{../Reflection.tex}

\documentclass{article}
\usepackage{enumitem}
\begin{document}

\begin{enumerate}
  \item \textbf{What went well while writing this deliverable?} \\
  \textbf{Mark}: This deliverable was a good opportunity to coordinate our efforts across a much larger body of work compared to the first deliverable. By maintaining continuous feedback and communication, our team avoided contradictions within the documentation. Sections such as “stretch goals”, “future plans”, and “waiting room” were helpful in having the group think forward and define what was definite versus what might happen if things went well. \\
  \textbf{Ian}: I think we communicated and split the work well during this deliverable. In some ways, this deliverable acted as the official start of the capstone project because it was much larger than the previous ones. It also required us to narrow down our vision for the project, which excited us all. \\
  \textbf{Emily}: This deliverable helped us gain a better understanding of the different aspects of the project we’ll be tackling. It can be difficult to grasp a project as large as this from the start, but we worked well as a team to communicate and prioritize the features we’d like to see in the final product. \\
  \textbf{Jackson}: I thought this was great because the deliverable cemented how we will work as a team. We communicated very well on each section, referencing each other’s work and asking for clarification when needed. I also enjoyed our open discussions about what we would implement in the future, as thinking ahead is always good for projects like this. \\

  \item \textbf{What pain points did you experience during this deliverable, and how did you resolve them?} \\
  \textbf{Mark}: A pain point was trying to minimize repetition in similar sections of the requirements document. It often felt as though the same thing was being said multiple times in different sections, making it hard to maintain a cohesive flow. \\
  \textbf{Ian}: One pain point was applying the Volere template (specifically business data and adjacent business analysis) to a project that is more or less standalone. The template is geared towards complex projects involving additions to existing systems, and it was hard to draw parallels when using it. \\
  \textbf{Emily}: My biggest pain point was distinguishing between functional and non-functional requirements. I’ve struggled with this since I first started writing requirements. I resolved it by doing research and finding examples to clarify the differences. \\
  \textbf{Jackson}: I had trouble rationalizing non-functional vs functional requirements, especially with performance requirements, as they often sounded like functional ones. I resolved this by learning more about their differences and understanding how non-functional requirements deal with system performance, while functional ones focus on how the system operates. \\

  \item \textbf{How many of your requirements were inspired by speaking to your clients or their proxies (e.g. peers, stakeholders, potential users)?} \\
  Most requirements came from personal experience and understanding of necessary features for an acceptable product. However, requirements related to user proficiency in music theory were formed through conversations with potential clients with varying levels of music knowledge. This helped determine what could be assumed as common knowledge versus what would require additional documentation. \\

  \item \textbf{Which of the courses you have taken, or are currently taking, will help your team be successful with your capstone project?} \\ \\
  \textbf{Mark}: The most valuable courses for technical knowledge will be 3MX3 and 3DX4, both focused on signal processing, a key aspect of the project. Experiential courses like 3XB3 and 2AA4 provided a strong foundation for building Git-based projects and collaborating within a team. \\
  \textbf{Ian}: 3MX3 is the course most related to this project, as it provides valuable insights into signal processing, the most important part of our project. Courses like 4AA4, which deals with real-time tasks, and 4HC3, focused on stakeholder analysis and interface design, are also relevant. \\
  \textbf{Emily}: SFWRENG 3A04, Large System Design, has the largest similarities to this project. The course involved a term-long project focused on requirements elicitation, use cases, and stakeholder definitions, which aligns with our current work. \\
  \textbf{Jackson}: Signal processing will be a large part of this project, and 3MX3 and 3DX4, which directly deal with this, are crucial. Additionally, courses with assignment-based projects like 2AA4 and 3BB4 gave me confidence when starting larger projects like this. \\

  \item \textbf{What knowledge and skills will the team collectively need to acquire to successfully complete this capstone project?} \\
  The team will need to acquire both technical and non-technical skills. Project management will be vital, requiring leadership, knowledge of agile methodologies, and delegation abilities. Music theory and composition knowledge will be necessary for ensuring accurate transcriptions. UI/UX design will require graphic design knowledge and familiarity with prototyping tools like Figma. Software testing and quality analysis skills will be exercised through our Kanban board and CI pipeline. Signal processing knowledge will be crucial for handling audio input. Each team member will focus on one of these areas to ensure the project’s success. \\

  \item \textbf{For each of the knowledge areas and skills identified, what are at least two approaches to acquiring the knowledge or mastering the skill? Which approaches will each team member pursue, and why?} \\
  Two approaches for acquiring knowledge include self-directed learning with online resources and practicing these skills through personal or open-source projects. Given the team’s full-time student status, both approaches are reasonable and cater well to self-paced learning. The team will pursue both approaches collaboratively, sharing newly acquired skills with each other as needed.
\end{enumerate}

\end{document}


\end{document}