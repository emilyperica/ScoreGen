% THIS DOCUMENT IS FOLLOWS THE VOLERE TEMPLATE BY Suzanne Robertson and James Robertson
% ONLY THE SECTION HEADINGS ARE PROVIDED
%
% Initial draft from https://github.com/Dieblich/volere
%
% Risks are removed because they are covered by the Hazard Analysis
\documentclass[12pt]{article}

\usepackage{booktabs}
\usepackage{tabularx}
\usepackage{hyperref}
\hypersetup{
    bookmarks=true,         % show bookmarks bar?
      colorlinks=true,      % false: boxed links; true: colored links
    linkcolor=red,          % color of internal links (change box color with linkbordercolor)
    citecolor=green,        % color of links to bibliography
    filecolor=magenta,      % color of file links
    urlcolor=cyan           % color of external links
}

\newcommand{\lips}{\textit{Insert your content here.}}

\input{../Comments}
%% Common Parts

\newcommand{\progname}{ScoreGen} % PUT YOUR PROGRAM NAME HERE
\newcommand{\authname}{Team \#7, Tune Goons
\\ Emily Perica
\\ Ian Algenio
\\ Jackson Lippert
\\ Mark Kogan} % AUTHOR NAMES                  

\usepackage{hyperref}
    \hypersetup{colorlinks=true, linkcolor=blue, citecolor=blue, filecolor=blue,
                urlcolor=blue, unicode=false}
    \urlstyle{same}
                                


\begin{document}

\title{Software Requirements Specification for \progname: subtitle describing software} 
\author{\authname}
\date{\today}
	
\maketitle

~\newpage

\pagenumbering{roman}

\tableofcontents

~\newpage

\section*{Revision History}

\begin{tabularx}{\textwidth}{p{3cm}p{2cm}X}
\toprule {\textbf{Date}} & {\textbf{Version}} & {\textbf{Notes}}\\
\midrule
Date 1 & 1.0 & Notes\\
Date 2 & 1.1 & Notes\\
\bottomrule
\end{tabularx}

~\\

~\newpage
\section{Purpose of the Project}
\subsection{User Business}
\lips
\subsection{Goals of the Project}
\lips
\section{Stakeholders}
\subsection{Client}
\lips
\subsection{Customer}
\lips
\subsection{Other Stakeholders}
\lips
\subsection{Hands-On Users of the Project}
\lips
\subsection{Personas}
\lips
\subsection{Priorities Assigned to Users}
\lips
\subsection{User Participation}
\lips
\subsection{Maintenance Users and Service Technicians}
\lips

\section{Mandated Constraints}
\subsection{Solution Constraints}
\lips
\subsection{Implementation Environment of the Current System}
\lips
\subsection{Partner or Collaborative Applications}
\lips
\subsection{Off-the-Shelf Software}
\lips
\subsection{Anticipated Workplace Environment}
\lips
\subsection{Schedule Constraints}
\lips
\subsection{Budget Constraints}
\lips
\subsection{Enterprise Constraints}
\lips

\section{Naming Conventions and Terminology}
\subsection{Glossary of All Terms, Including Acronyms, Used by Stakeholders
involved in the Project}
\lips

\section{Relevant Facts And Assumptions}
\subsection{Relevant Facts}
\lips
\subsection{Business Rules}
\lips
\subsection{Assumptions}
\lips

\section{The Scope of the Work}
\subsection{The Current Situation}
\lips
\subsection{The Context of the Work}
\lips
\subsection{Work Partitioning}
\lips
\subsection{Specifying a Business Use Case (BUC)}
\lips

\section{Business Data Model and Data Dictionary}
\subsection{Business Data Model}
\lips
\subsection{Data Dictionary}
\lips

\section{The Scope of the Product}
\subsection{Product Boundary}
\lips
\subsection{Product Use Case Table}
\lips
\subsection{Individual Product Use Cases (PUC's)}
\lips

\section{Functional Requirements}
\subsection{Functional Requirements}
\lips

\section{Look and Feel Requirements}
\subsection{Appearance Requirements}
\lips
\subsection{Style Requirements}
\lips

\section{Usability and Humanity Requirements}
\subsection{Ease of Use Requirements}
\lips
\subsection{Personalization and Internationalization Requirements}
\lips
\subsection{Learning Requirements}
\lips
\subsection{Understandability and Politeness Requirements}
\lips
\subsection{Accessibility Requirements}
\lips

\section{Performance Requirements}
\subsection{Speed and Latency Requirements}
\subsection*{SL-PR1 User Interface Response Time}
\textbf{Requirement:} The interface between the user and the sheet music generator app shall have a maximum response time of 2 seconds when processing input or selecting menu options.\\
\textbf{Fit Criterion:} The product shall respond in less than 1 second for 90 percent of user interactions. No response shall take longer than 2.5 seconds.
\subsection*{SL-PR2 File Import and Export Speed}
\textbf{Requirement:} The app shall import and export music files within a reasonable time for files up to 100MB in size.\\
\textbf{Fit Criterion:} For files up to 100MB, the app shall import and export them within a reasonable time frame for their system for more than 95 percent of operations.

\subsection{Safety-Critical Requirements}
\subsection*{SC-PR1 Epilepsy Safety}
\textbf{Requirement:} The app shall not display rapid flashing lights, animations, or any visual effects that could trigger epileptic seizures in users.\\
\textbf{Fit Criterion:} The app’s graphical interface must comply with WCAG 2.1 (Web Content Accessibility Guidelines) for flashing content, limiting visual effects to fewer than 3 flashes per second. The interface will be tested with accessibility tools to ensure compliance with these standards.
\subsection*{SC-PR2 Instrument Input Setup}
\textbf{Requirement:} The app shall guide users through the correct setup of instrument input to prevent audio misconfiguration, damage to the user/instrument, and damage to the application’s device.\\
\textbf{Fit Criterion:} The app will include an interactive, step-by-step tutorial for configuring instrument inputs (via external devices or microphones). The setup process should be completed with a 95\% success rate based on user testing, ensuring correct audio input detection on the first attempt.

\subsection{Precision or Accuracy Requirements}
\subsection*{PA-PR1 Pitch Detection Accuracy}
\textbf{Requirement:} A critical requirement of this app is the precision at which it identifies notes. As a result, this app shall detect musical pitch from instrument input with an accuracy of ±5 cents (1/20th of a semitone) to ensure that all notes are transcribed correctly.\\
\textbf{Fit Criterion:} The app must maintain this pitch accuracy over 100 test cases involving diverse instruments and note ranges, with no more than 1\% error.
\subsection*{PA-PR2 Timing Accuracy}
\textbf{Requirement:} The app shall accurately capture note durations and rhythms within a tolerance of ±20 milliseconds to ensure correct rhythmic transcription.\\
\textbf{Fit Criterion:} The app must maintain timing accuracy within the specified tolerance for both slow and fast tempos (ranging from 40 to 240 BPM) across 100 test cases.

\subsection{Robustness or Fault-Tolerance Requirements}
\subsection*{RFT-PR1 Reliability (Time Between Failures)}
\textbf{Requirement:} The app shall operate continuously for at least 24 hours between failures or system crashes, ensuring that users experience minimal interruptions during usage.\\
\textbf{Fit Criterion:} The app must pass reliability testing by operating under typical usage conditions for 24 hours without a failure in at least 95\% of test cases.
\subsection*{RFT-PR2 Availability (Uptime)}
\textbf{Requirement:} The app shall achieve 99.5\% uptime, ensuring that it is available for use at nearly all times, excluding brief periods for necessary updates or maintenance.\\
\textbf{Fit Criterion:} The app must demonstrate an average of 99.5\% uptime over a reasonable time.
\subsection*{RFT-PR3 Crash Recovery}
\textbf{Requirement:} In an unexpected crash, the app shall automatically save the user’s current session, including any partially transcribed sheet music, and allow the user to recover their work upon restart.\\
\textbf{Fit Criterion:} During testing, the app must successfully recover user data in 98\% of cases where a crash occurs, with no data loss.
\subsection*{RFT-PR4 Performance Under Load}
\textbf{Requirement:} The app shall maintain reliable performance under conditions of high user activity (e.g., processing complex polyphonic inputs or large files) without significant slowdowns or crashes.\\
\textbf{Fit Criterion:} The app must maintain stable performance, with no more than a 30\% decrease in processing speed, when tested under maximum expected load conditions.
\subsection*{RFT-PR5 Handling Signal Interruptions}
\textbf{Requirement:} The app shall continue to operate and retain the audio signals received up to the point of interruption, even if the instrument input is temporarily lost (e.g., due to cable disconnection or microphone failure).\\
\textbf{Fit Criterion:} The app must store buffered audio data for up to 2 minutes during signal interruptions and automatically resume transcription once the input is restored, with 0\% data loss in at least 95\% of test cases.
\subsection*{RFT-PR6 Graceful Degradation}
\textbf{Requirement:} In the event of a performance slowdown (e.g., due to excessive input data or limited system resources), the app shall degrade its services gracefully by notifying the user of delays without crashing.\\
\textbf{Fit Criterion:} During stress testing under heavy load (e.g., multiple instruments or large file sizes), the app must maintain operational status and notify the user of delays or lowered performance within 5 seconds of detection, with no system crashes in 100 test cases.
\subsection*{RFT-PR7 Automatic Recovery from Software Glitches}
\textbf{Requirement:} If the app encounters a minor software error (e.g., unexpected input format or corrupted file), it shall log the error, notify the user, and continue functioning by skipping the problematic section rather than halting it.\\
\textbf{Fit Criterion:} The app must pass testing by handling minor software errors in at least 90\% of cases without crashing and will log the issue for future debugging.

\subsection{Capacity Requirements}
\subsection*{C-PR1 Audio Input Capacity}
\textbf{Requirement:} The app shall process audio input of up to 90 minutes of continuous recording in a single session without performance degradation.\\
\textbf{Fit Criterion:} The app must be tested with audio recordings of varying lengths (up to 90 minutes) and successfully process and transcribe the entire duration with no more than a 5\% slowdown in transcription speed or accuracy.
\subsection*{C-PR2 Storage of Generated Sheet Music}
\textbf{Requirement:} The app shall store up to 500 sheet music files per user, each up to 10MB in size, while maintaining access to all files without significant delays.\\
\textbf{Fit Criterion:} The app must be tested with up to 500 saved sheet music files per user, and it must be able to retrieve and open any file within a reasonable time for at least 95\% of test cases.
\subsection*{C-PR3 Simultaneous User Sessions}
\textbf{Requirement:} The app shall support at least 100 simultaneous user sessions (local or cloud-based) without degradation in performance or delays in processing.\\
\textbf{Fit Criterion:} During testing with 100 simultaneous active user sessions, the app must maintain performance levels, including stable response times and transcription accuracy, with no noticeable slowdown in 95\% of test cases.

\subsection{Scalability or Extensibility Requirements}
\subsection*{SE-PR1 User Base Growth}
\textbf{Requirement:} The app shall be capable of scaling to support an increase in the user base to 1000 users without requiring major architectural changes.\\
\textbf{Fit Criterion:} The app’s infrastructure must be designed to scale up by a factor of 10 and should undergo stress testing to ensure it can handle 1000 simultaneous user sessions with no significant performance degradation.
\subsection*{SE-PR2 Data Storage Expansion}
\textbf{Requirement:} The app shall scale to store up to 1 million sheet music files, each up to 10MB in size, by implementing efficient data storage and retrieval mechanisms.\\
\textbf{Fit Criterion:} The app must be tested for large-scale storage handling, with performance benchmarks showing the ability to manage 1 million files without delays exceeding 5 seconds for any retrieval operation.
\subsection*{SE-PR3 Feature Extensibility}
\textbf{Requirement:} The app shall be designed with a modular architecture that allows for future feature additions, such as multi-instrument detection or cloud-based collaboration tools, without requiring significant refactoring.\\
\textbf{Fit Criterion:} The app must pass code reviews and architecture validation to ensure it can integrate additional modules or features within 6 months of the product’s initial release, with minimal impact on core functionalities.
\subsection*{SE-PR4 Processing Power for Complex Music Compositions}
\textbf{Requirement:} The app shall be built to be able to scale its processing capabilities to handle future complex compositions involving multiple simultaneous instrument tracks supporting growth for polyphony.\\
\textbf{Fit Criterion:} During testing, the app must demonstrate the ability to process increasingly complex musical arrangements.

\subsection{Longevity Requirements}
\subsection*{L-PR1 Expected Lifetime}
\textbf{Requirement:} The app shall be designed to operate effectively with regular software updates and minor maintenance for a minimum of five years without requiring a major rewrite or overhaul.\\
\textbf{Fit Criterion:} The app’s development roadmap must include planned updates and feature expansions to maintain functionality and relevance for at least five years after launch.

\section{Operational and Environmental Requirements}
\subsection{Expected Physical Environment}
\lips
\subsection{Wider Environment Requirements}
\lips
\subsection{Requirements for Interfacing with Adjacent Systems}
\lips
\subsection{Productization Requirements}
\lips
\subsection{Release Requirements}
\lips

\section{Maintainability and Support Requirements}
\subsection{Maintenance Requirements}
\lips
\subsection{Supportability Requirements}
\lips
\subsection{Adaptability Requirements}
\lips

\section{Security Requirements}
\subsection{Access Requirements}
\lips
\subsection{Integrity Requirements}
\lips
\subsection{Privacy Requirements}
\lips
\subsection{Audit Requirements}
\lips
\subsection{Immunity Requirements}
\lips

\section{Cultural Requirements}
\subsection{Cultural Requirements}
\lips

\section{Compliance Requirements}
\subsection{Legal Requirements}
\lips
\subsection{Standards Compliance Requirements}
\lips

\section{Open Issues}
\lips

\section{Off-the-Shelf Solutions}
\subsection{Ready-Made Products}
\lips
\subsection{Reusable Components}
\lips
\subsection{Products That Can Be Copied}
\lips

\section{New Problems}
\subsection{Effects on the Current Environment}
\lips
\subsection{Effects on the Installed Systems}
\lips
\subsection{Potential User Problems}
\lips
\subsection{Limitations in the Anticipated Implementation Environment That May
Inhibit the New Product}
\lips
\subsection{Follow-Up Problems}
\lips

\section{Tasks}
\subsection{Project Planning}
\lips
\subsection{Planning of the Development Phases}
\lips

\section{Migration to the New Product}
\subsection{Requirements for Migration to the New Product}
\lips
\subsection{Data That Has to be Modified or Translated for the New System}
\lips

\section{Costs}
\lips
\section{User Documentation and Training}
\subsection{User Documentation Requirements}
\lips
\subsection{Training Requirements}
\lips

\section{Waiting Room}
\lips

\section{Ideas for Solution}
\lips

\newpage{}
\section*{Appendix --- Reflection}

The information in this section will be used to evaluate the team members on the
graduate attribute of Lifelong Learning.  Please answer the following questions:

\begin{enumerate}
  \item What knowledge and skills will the team collectively need to acquire to
  successfully complete this capstone project?  Examples of possible knowledge
  to acquire include domain specific knowledge from the domain of your
  application, or software engineering knowledge, mechatronics knowledge or
  computer science knowledge.  Skills may be related to technology, or writing,
  or presentation, or team management, etc.  You should look to identify at
  least one item for each team member.
  \item For each of the knowledge areas and skills identified in the previous
  question, what are at least two approaches to acquiring the knowledge or
  mastering the skill?  Of the identified approaches, which will each team
  member pursue, and why did they make this choice?
\end{enumerate}

\end{document}