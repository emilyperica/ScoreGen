\documentclass[12pt, titlepage]{article}

\usepackage{tabularx}
\usepackage{booktabs}

\input{../Comments}
%% Common Parts

\newcommand{\progname}{ScoreGen} % PUT YOUR PROGRAM NAME HERE
\newcommand{\authname}{Team \#7, Tune Goons
\\ Emily Perica
\\ Ian Algenio
\\ Jackson Lippert
\\ Mark Kogan} % AUTHOR NAMES                  

\usepackage{hyperref}
    \hypersetup{colorlinks=true, linkcolor=blue, citecolor=blue, filecolor=blue,
                urlcolor=blue, unicode=false}
    \urlstyle{same}
                                


\begin{document}

\title{GenderMag Report: \progname} 
\author{\authname}
\date{\today}
	
\maketitle

\pagenumbering{roman}

\section{Revision History}

\begin{tabularx}{\textwidth}{p{3cm}p{2cm}X}
\toprule {\bf Date} & {\bf Version} & {\bf Notes}\\
\midrule
2025-03-18 & 1.0 & Initial version.\\
\bottomrule
\end{tabularx}

~\newpage

\section{Symbols, Abbreviations and Acronyms}

\renewcommand{\arraystretch}{1.2}
\begin{table}[h!]
  \vspace{5pt}
  \begin{tabular}{l l} 
    \toprule		
    \textbf{Symbol} & \textbf{Description} \\
    \midrule 
    GM & GenderMag. \\
    \bottomrule
  \end{tabular}\\
\end{table}
\newpage

\tableofcontents

\newpage

\pagenumbering{arabic}
\newpage

\section{Introduction} %reference needed
This document details the results of the usage of the GenderMag method to evaluate gender
inclusivity as an aspect of ScoreGen's usability. It uses empirically based personas to simulate
the software system's use from diverse perspectives, revealing design issues that might be missed.\\
This document is organized into sections covering use cases, customized personas, reporting forms 
(both subgoal and action reporting), and the design changes implemented based on the GenderMag 
evaluation. Each section provides a focused look at how GenderMag helps improve software 
inclusiveness.

\section{Use Cases}
The GenderMag method was applied to the following two use cases of the software system:
\begin{enumerate}
    \item Recording audio
    \item Generating sheet music from audio
\end{enumerate}

\section{Customized Personas}



\section{Reporting Forms}
\subsection{Subgoal Reporting Forms}



\subsection{Action Reporting Forms}




\section{Changes Due to GenderMag Evaluation}


\newpage
\section*{Appendix -- Reflection}

\end{document}