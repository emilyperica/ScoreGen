\documentclass{article}

\usepackage{booktabs}
\usepackage{tabularx}
\usepackage{hyperref}

\hypersetup{
    colorlinks=true,       % false: boxed links; true: colored links
    linkcolor=red,          % color of internal links (change box color with linkbordercolor)
    citecolor=green,        % color of links to bibliography
    filecolor=magenta,      % color of file links
    urlcolor=cyan           % color of external links
}

\title{Hazard Analysis\\\progname}

\author{\authname}

\date{}

\input{../Comments}
%% Common Parts

\newcommand{\progname}{ScoreGen} % PUT YOUR PROGRAM NAME HERE
\newcommand{\authname}{Team \#7, Tune Goons
\\ Emily Perica
\\ Ian Algenio
\\ Jackson Lippert
\\ Mark Kogan} % AUTHOR NAMES                  

\usepackage{hyperref}
    \hypersetup{colorlinks=true, linkcolor=blue, citecolor=blue, filecolor=blue,
                urlcolor=blue, unicode=false}
    \urlstyle{same}
                                


\begin{document}

\maketitle
\thispagestyle{empty}

~\newpage

\pagenumbering{roman}

\begin{table}[hp]
\caption{Revision History} \label{TblRevisionHistory}
\begin{tabularx}{\textwidth}{llX}
\toprule
\textbf{Date} & \textbf{Developer(s)} & \textbf{Change}\\
\midrule
Date1 & Name(s) & Description of changes\\
Date2 & Name(s) & Description of changes\\
... & ... & ...\\
\bottomrule
\end{tabularx}
\end{table}

~\newpage

\tableofcontents

~\newpage

\pagenumbering{arabic}

\wss{You are free to modify this template.}

\section{Introduction}

\wss{You can include your definition of what a hazard is here.}

\section{Scope and Purpose of Hazard Analysis}

\wss{You should say what \textbf{loss} could be incurred because of the
hazards.}

\section{System Boundaries and Components}

\wss{Dividing the system into components will help you brainstorm the hazards.
You shouldn't do a full design of the components, just get a feel for the major
ones.  For projects that involve hardware, the components will typically include
each individual piece of hardware.  If your software will have a database, or an
important library, these are also potential components.}

\section{Critical Assumptions}

\wss{These assumptions that are made about the software or system.  You should
minimize the number of assumptions that remove potential hazards.  For instance,
you could assume a part will never fail, but it is generally better to include
this potential failure mode.}

\section{Failure Mode and Effect Analysis}

\wss{Include your FMEA table here. This is the most important part of this document.}
\wss{The safety requirements in the table do not have to have the prefix SR.
The most important thing is to show traceability to your SRS. You might trace to
requirements you have already written, or you might need to add new
requirements.}
\wss{If no safety requirement can be devised, other mitigation strategies can be
entered in the table, including strategies involving providing additional
documentation, and/or test cases.}

\section{Safety and Security Requirements}

Currently, there are no new safety or security requirements, but any new requirements will be added to this document as part of our iterative development process. Newly discovered requirements will also be evaluated and incorporated into the SRS to ensure alignment with evolving project needs and insights gained during development.


\section{Roadmap}

We will likely fulfill every one of the security and safety-critical requirements within the timeline of the capstone course, and we will be loosely following the compliance of the requirements based on the priority assigned below:

\subsection{Safety-Critical Requirements}
\begin{itemize}
    \item \textbf{PR-SC1 Epilepsy Safety}: \textit{High priority}, as it ensures the app’s visual safety for users from the start. Compliance with WCAG 2.1 should be confirmed early on to prevent any risks during development.
    \item \textbf{PR-SC2 Instrument Input Setup}: \textit{High priority} for both user safety and accurate functionality, ensuring users can set up their instruments without misconfiguration. This interactive guide can be iteratively refined through user testing early in development.
\end{itemize}

\subsection{Privacy Requirements}
\begin{itemize}
    \item \textbf{S-P2 PII}: \textit{High priority}, as ensuring the app does not collect any Personal Identifiable Information aligns with the app’s privacy-first design. This requirement should be validated as soon as any data processing is introduced.
    \item \textbf{S-P3 Input Data}: \textit{Low priority}, focusing on ensuring that temporary files are properly cleared after processing is complete. Implementing this is less critical since we will not be storing any data outside of the user’s system.
    \item \textbf{S-P1 Data Storage}: \textit{Medium priority}, ensuring that user-generated data is stored in local, user-chosen locations. This feature should be verified once output functions are introduced.
\end{itemize}

\subsection{Access Requirements}
\begin{itemize}
    \item \textbf{S-A1 User Authentication}: \textit{High priority}, by removing the need for login functionalities, we will speed up initial development. It should be verified after the core interface is complete.
\end{itemize}

\subsection{Audit and Immunity Requirements}
\begin{itemize}
    \item \textbf{Audit Requirements (N/A)}: \textit{Low priority}, as the system is offline and does not require developer responsibility over user data.
    \item \textbf{Immunity Requirements (N/A)}: \textit{Low priority}, as no data storage minimizes exposure to security risks.
\end{itemize}


\newpage{}

\section*{Appendix --- Reflection}

\wss{Not required for CAS 741}

\input{../Reflection.tex}

\begin{enumerate}
    \item What went well while writing this deliverable? 
    \item What pain points did you experience during this deliverable, and how
    did you resolve them?
    \item Which of your listed risks had your team thought of before this
    deliverable, and which did you think of while doing this deliverable? For
    the latter ones (ones you thought of while doing the Hazard Analysis), how
    did they come about?
    \item Other than the risk of physical harm (some projects may not have any
    appreciable risks of this form), list at least 2 other types of risk in
    software products. Why are they important to consider?
\end{enumerate}

\end{document}