\documentclass[12pt, titlepage]{article}

\usepackage{booktabs}
\usepackage{tabularx}
\usepackage{hyperref}
\hypersetup{
    colorlinks,
    citecolor=black,
    filecolor=black,
    linkcolor=red,
    urlcolor=blue
}
\usepackage[round]{natbib}
\usepackage{graphicx}
\usepackage[T1]{fontenc}
\usepackage{ragged2e}
\usepackage{float}
\usepackage{longtable}
\usepackage{makecell}
\usepackage{seqsplit}
\usepackage{float}

\newcolumntype{L}[1]{>{\RaggedRight\arraybackslash}p{#1}}
\newcolumntype{C}{>{\centering\arraybackslash}X}
\newcolumntype{L}[1]{>{\raggedright\arraybackslash}p{#1}}
\AtBeginEnvironment{longtable}{\small}

\input{../Comments}
\input{../Common}

\begin{document}

\title{Verification and Validation Report: \progname} 
\author{\authname}
\date{\today}
	
\maketitle

\pagenumbering{roman}

\section{Revision History}

\begin{tabularx}{\textwidth}{p{3cm}p{2cm}X}
\toprule {\bf Date} & {\bf Version} & {\bf Notes}\\
\midrule
2025-03-10 & 1.0 & Initial version.\\
\bottomrule
\end{tabularx}

~\newpage

\section{Symbols, Abbreviations and Acronyms}

\renewcommand{\arraystretch}{1.2}
\begin{table}[h!]
  \vspace{5pt}
  \begin{tabular}{l l} 
    \toprule		
    \textbf{Symbol} & \textbf{Description} \\
    \midrule 
    CI & Continuous Integration. \\
    CMake & The build system used for configuring the project. \\
    ctest& CMake testing tool to execute unit tests. \\
    GHA & GitHub Actions. \\
    GTest & Google Test, a C++ unit testing framework. \\
    I/O & Input/Output. \\
    mXML & MusicXML. \\
    PDF & Portable Document Format. \\
    PCM\_16 & 16-bit Pulse-Code Modulation. \\
    SF\_INFO & A structure (from libsndfile) that holds data for file I/O. \\
    SRS & Software Requirements Specification. \\
    UI & User Interface. \\
    vcpkg & C++ dependency manager. \\
    WAV & Waveform Audio File Format. \\
    \bottomrule
  \end{tabular}\\
\end{table}
\newpage

\tableofcontents

\listoftables %if appropriate

\newpage

\pagenumbering{arabic}

This document outlines the results of the 
\href{https://github.com/emilyperica/ScoreGen/blob/main/docs/VnVPlan/VnVPlan.pdf}{VnV plan}
executed for the development of an audio-to-sheet music generator. 
Functional and Nonfunctional Requirements Evaluations assess ScoreGen's 
adherence to system requirements as set out in the \href{https://github.com/emilyperica/ScoreGen/blob/main/docs/SRS-Volere/SRS.pdf}{SRS} document. 
Unit Testing verifies the correctness of individual components, while Changes 
Due to Testing describes any modifications made in response to detected issues. 
The Automated Testing section details the integration of test suites to ensure 
consistent, continuous verification. Trace to Requirements and Trace to Modules 
ensure comphrensive validation by establishing clear connctions between test 
cases, software requirements, and system components. Finally, Code Coverage Metrics 
provide a quantitative analysis of test coverage.

\section{Functional Requirements Evaluation}
\label{fr}
Tests for the functional requirements of the application naturally follow the division of the major types of functional requirements
in section 9 of the \href{https://github.com/emilyperica/ScoreGen/blob/main/docs/SRS-Volere/SRS.pdf}{SRS}. Thus, tests for each major 
type are grouped similarly, resulting in five subcategories.

  \subsection{Input Handling}
  \begin{enumerate}
    \item \textbf{Test for Correct Audio File Formats} \\
      \newline
      \textbf{Test ID:} FR-AR1-3-1 \\
      \textbf{Control:} Automatic \\
      \textbf{Initial State:} Application is running and idle, awaiting audio file upload from the user. \\
      \textbf{Input:} Audio file (e.g., \texttt{.WAV}, \texttt{.MP3}) \\
      \textbf{Expected Output:} File acceptance without errors, entrance into file processing state. \\
      \textbf{Actual Output:} File acceptance without errors, entrance into file processing state. \\
      \textbf{Test Case Derivation:} For validation of functional requirements FR-AR1 and FR-AR3 in section 9.1 of the 
      \href{https://github.com/emilyperica/ScoreGen/blob/main/docs/SRS-Volere/SRS.pdf}{SRS}. \\
      \textbf{Justification:} This test case ensures the application only accepts supported file formats and only begins processing 
      these formats. This test case acts as a preventative measure against further error progogation through the file processing stage.\\
      \textbf{How Test Was Performed:}
      \begin{enumerate}
          \item Select a prepared audio sample file.
          \item Upload the audio file for transcription.
          \item Confirm the application accepts the file through success message or by entrance into the file processing state.
      \end{enumerate}
      \textbf{Result:} PASS
  
  \item \textbf{Test for Incorrect File Formats} \\
    \newline
    \textbf{Test ID:} FR-AR1-3-2 \\
    \textbf{Control:} Automatic \\
    \textbf{Initial State:} Application is open and idle, awaiting audio file upload from the user. \\
    \textbf{Input:} Non-audio file (e.g., \texttt{.PDF}, \texttt{.MP4}, \texttt{.JPEG}, etc.) \\
    \textbf{Expected Output:} Denial of upload attempt with error message. \\
    \textbf{Actual Output:} Denial of upload attempt with error message.\\
    \textbf{Test Case Derivation:} For validation of functional requirements FR-AR1 and FR-AR3 in section 9.1 of the
    \href{https://github.com/emilyperica/ScoreGen/blob/main/docs/SRS-Volere/SRS.pdf}{SRS}. \\
    \textbf{Justification:} Complements test case FR-AR1-3-1. It will ensure that various, unsupported file formats are 
    not accepted by the application. \\
    \textbf{How Test Was Performed:}
    \begin{enumerate}
        \item Select a prepared file of an unsupported format (i.e. non-audio file format).
        \item Upload the file for transcription.
        \item Confirm the application denies the file and provides an error message that specifies the unsupported file format.
    \end{enumerate}
    \textbf{Result:} PASS
  
  \item \textbf{Test User Device Microphone} \\
  \newline
  \textbf{Test ID:} FR-AR2 \\
  \textbf{Control:} Manual \\
  \textbf{Initial State:} Application is open and idle, user has navigated to the audio recording interface and microphone 
  permissions have been granted. \\
  \textbf{Input:} Audio recorded by the user device’s microphone. \\
  \textbf{Expected Output:} The application captures the correct audio and saves it in a format processable by the application 
  (e.g., \texttt{.WAV}). \\
  \textbf{Actual Output:} The application captures the correct audio and saves it as a .wav file. \\
  \textbf{Test Case Derivation:} For validation of functional requirement FR-AR2 in section 9.1 of the 
  \href{https://github.com/emilyperica/ScoreGen/blob/main/docs/SRS-Volere/SRS.pdf}{SRS}. Guarentees that in addition to file uploads, 
  the user is able to capture raw, custom audio with their device hardware. \\
  \textbf{Justification:} It also validates that the audio is saved in a compatitble format for processing.\\
  \textbf{How Test Was Performed:}
  \begin{enumerate}
      \item Navigate to the application’s audio recording interface.
      \item Start a recording using the application’s “Record” element.
      \item Wait for 10 seconds and stop the recording.
      \item Confirm that the audio was recorded through:
      \begin{itemize}
          \item Playback using an audio player on the device.
          \item File metadata analysis (duration, file size, etc.).
          \item Waveform inspection using an external tool (e.g., Audacity).
      \end{itemize}
      \item Confirm that the captured audio matches the expected input via playback, metadata, and/or waveform inspection.
  \end{enumerate}
  \textbf{Result:} PASS
  
  \item \textbf{Test Generated Score Alignment with Selected Instrument} \\
    \newline
    \textbf{Test ID:} FR-AR4 \\
    \textbf{Control:} Automatic \\
    \textbf{Initial State:} Application is running and idle, pre-transcription state awaiting input. \\
    \textbf{Input:} Sample audio file or user-recorded audio and selected instrument type. \\
    \textbf{Expected Output:} A generated sheet music file in MusicXML format that aligns with the selected instrument’s key signature 
    and note pitches. \\
    \textbf{Actual Output:} A generated sheet music file in MusicXML format that aligns with the selected instrument’s key signature 
    and note pitches. \\
    \textbf{Test Case Derivation:} For validation of functional requirement FR-AR4 in section 9.1 of the 
    \href{https://github.com/emilyperica/ScoreGen/blob/main/docs/SRS-Volere/SRS.pdf}{SRS}.\\
    \textbf{Justification:} Matching an instrument's specific key and pitch requirements is essential for sheet music accuracy. \\
    \textbf{How Test Was Performed:}
    \begin{enumerate}
        \item Select and submit an instrument type within the application.
        \item Upload an audio input for the selected instrument.
        \item Upon transcription completion, parse the generated score.
        \item Verify that the key signature and note pitches match the expected result for the selected instrument.
    \end{enumerate}
    \textbf{Result:} PASS
  \end{enumerate}
  
  \subsection{Signal Processing and Element Identification}
  \begin{enumerate}
    \item \textbf{Test Effect of Increased Noise in Audio Input} \\
      \newline
      \textbf{Test ID:} FR-SP1 \\
      \textbf{Control:} Automatic \\
      \textbf{Initial State:} Application is running and idle, awaiting file upload from the user. \\
      \textbf{Input:} Two sample audio files—one unedited, the other with ~10\% noise interference. \\
      \textbf{Expected Output:} Two identical sheet music files in MusicXML format. \\
      \textbf{Actual Output:} Two identical sheet music files in MusicXML format. \\
      \textbf{Test Case Derivation:} For validation of functional requirement FR-SP1 in section 9.2 of the 
      \href{https://github.com/emilyperica/ScoreGen/blob/main/docs/SRS-Volere/SRS.pdf}{SRS}. \\
      \textbf{Justification:} Confirms that the application can handle expected levels of background noise in input 
      audio, and that it can maintain accuracy despite the noise.\\
      \textbf{How Test Was Performed:}
      \begin{enumerate}
          \item Upload the unedited audio file to the application and generate a sheet music file.
          \item Upload the noisy audio file and generate another sheet music file.
          \item Parse both files and identify discrepancies, if any.
      \end{enumerate}
      \textbf{Result:} PASS
    
    \item \textbf{Test for Pitch and Rhythm Identification} \\
      \newline
      \textbf{Test ID:} FR-SP2 \\
      \textbf{Control:} Manual \\
      \textbf{Initial State:} Application is running and idle, awaiting audio input from the user. \\
      \textbf{Input:} Sample audio file or user-recorded audio with a known sheet music equivalent. \\
      \textbf{Expected Output:} Sheet music correctly identifying all notes in the input audio. \\
      \textbf{Actual Output:} Sheet music correctly identifying all notes in the input audio. \\
      \textbf{Test Case Derivation:} For validation of functional requirement FR-SP2 in section 9.2 of the 
      \href{https://github.com/emilyperica/ScoreGen/blob/main/docs/SRS-Volere/SRS.pdf}{SRS}. \\
      \textbf{Justification:} Verfies the transcription stage as well as the pitch and rhythm identification algorithms employed
      by the applications implementation.\\
      \textbf{How Test Was Performed:}
      \begin{enumerate}
          \item Prepare an audio file containing an ascending and descending C major scale.
          \item Upload the audio input to the application.
          \item Generate and save the sheet music in a viewable file format.
          \item Compare the generated sheet music visually with the sample sheet music for note discrepancies.
      \end{enumerate}
      \textbf{Result:} PASS

    \item \textbf{Test for Key Signature and Time Signature Identification} \\
      \newline
      \textbf{Test ID:} FR-SP3 \\
      \textbf{Control:} Automatic \\
      \textbf{Initial State:} Application is running and idle, awaiting audio input from the user. \\
      \textbf{Input:} Sample or user-recorded audio file that has a known key signature and time signature. \\
      \textbf{Expectd Output:} Sheet music that has the same key signature and time signature as the input’s sheet music. \\
      \textbf{Actual Output:} Sheet music that has the same key signature and time signature as the input’s sheet music. \\
      \textbf{Test Case Derivation:} For validation of functional requirement FR-SP3 in section 9.2 of the 
      \href{https://github.com/emilyperica/ScoreGen/blob/main/docs/SRS-Volere/SRS.pdf}{SRS}. \\
      \textbf{Justification:} Further verifies the transcription stage and accuracy of other sheet music elements.\\
      \textbf{How Test Was Performed:}
      \begin{enumerate}
          \item Upload the sample file or recorded audio file to the application.
          \item Save the generated sheet music in MusicXML file format.
          \item Parse the generated file and extract the key signature and time signature.
          \item Compare the extracted signatures to the known input’s signatures.
      \end{enumerate}
      \textbf{Result:} PASS
    \end{enumerate}

  \subsection{Sheet Music Generation}
  \begin{enumerate}
  \item \textbf{Test General Notation and Layout} \\
    \newline
    \textbf{Test ID:} FR-SMG1 \\
    \textbf{Control:} Automatic with manual inspection \\
    \textbf{Initial State:} Application is running and idle, awaiting audio input from the user. \\
    \textbf{Input:} Sample or user-recorded audio file with equivalent sheet music available. \\
    \textbf{Expected Output:} Sheet music using the same layout and notation as the input. \\
    \textbf{Actual Output:} Sheet music using the same layout and notation as the input. \\
    \textbf{Test Case Derivation:} For validation of functional requirement FR-SMG1 in section 9.3 of the 
    \href{https://github.com/emilyperica/ScoreGen/blob/main/docs/SRS-Volere/SRS.pdf}{SRS}. \\
    \textbf{Justification:} Ensures readability and usability, using improper notation and layout defeats the purpose of 
    generating sheet music. \\
    \textbf{How Test Was Performed:}
    \begin{enumerate}
        \item Upload the sample or user-recorded audio file for transcription.
        \item Save the generated sheet music in MusicXML format and a viewable document format.
        \item Check for discrepancies:
        \begin{itemize}
            \item Parse the MusicXML file and compare it to the input file.
            \item View the document formatted file (e.g., \texttt{.PDF}) and compare it to the input’s sheet music.
        \end{itemize}
    \end{enumerate}
    \textbf{Result:} PASS
  
  \item \textbf{Test Post-Processing Edit and View Functionalities} \\
    \newline
    \textbf{Test ID:} FR-SMG3 \\
    \textbf{Control:} Manual \\
    \textbf{Initial State:} Application has just finished processing and transcribing audio input. \\
    \textbf{Input:} Edit requests, save requests, open requests. \\
    \textbf{Expected Output:} Viewable file containing sheet music that reflects the requested edits. \\
    \textbf{Actual Output:} Viewable file containing sheet music that reflects the requested edits. \\
    \textbf{Test Case Derivation:} For validation of functional requirement FR-SMG3 in section 9.3 of the 
    \href{https://github.com/emilyperica/ScoreGen/blob/main/docs/SRS-Volere/SRS.pdf}{SRS}. \\
    \textbf{Justification:} Ensures that users can change any elements of the sheet music and that these 
    changes are reflected properly by the application.\\ 
    \textbf{How Test Was Performed:}
    \begin{enumerate}
        \item View the generated sheet music in the application.
        \item Perform three edit operations:
        \begin{itemize}
            \item Note deletion.
            \item Note addition.
            \item Note pitch and/or duration change.
        \end{itemize}
        \item Save the edited sheet music in a viewable file format.
        \item Open and view the edited sheet music file to confirm that changes are present and correct.
    \end{enumerate}
    \textbf{Result:} PASS
  \end{enumerate}
  

  \subsection{User Interface (UI)}
  \begin{enumerate}
    \item \textbf{Test Application Feedback After Audio File is Uploaded} \\
      \newline
      \textbf{Test ID:} FR-UI1 \\
      \textbf{Control:} Automatic \\
      \textbf{Initial State:} Application is running and idle, awaiting audio input from the user. \\
      \textbf{Input:} A sample audio file. \\
      \textbf{Expected Output:} Visual cue(s) on the GUI in 2 seconds or less. \\
      \textbf{Actual Output:} Visual cue(s) on the GUI with no perceptible delay (i.e., appears instant) \\
      \textbf{Test Case Derivation:} For validation of functional requirement FR-UI1 in section 9.4 of the 
      \href{https://github.com/emilyperica/ScoreGen/blob/main/docs/SRS-Volere/SRS.pdf}{SRS}. \\
      \textbf{Justification:} Validates the funcationality that contributes to the non-functional requirement of
      human-centered design principles (see Section 4.2.2.1). One of the four fundamental principles is feedback, without 
      the proper creation and display of visual feedback cues, the application does not fulfill this principle to the 
      best of its ability.\\
      \textbf{How Test Was Performed:}
      \begin{enumerate}
          \item Upload a sample audio file to the application.
          \item Start a timer and detect visual element changes on the GUI using a testing framework (e.g., Jest).
          \item Confirm the following conditions are met:
          \begin{itemize}
              \item The appropriate visual cue is detected.
              \item The elapsed time from upload start to display of visual feedback is at most 2 seconds.
          \end{itemize}
      \end{enumerate}
      \textbf{Result:} PASS
    
    \item \textbf{Test Availability of System Documentation} \\
      \newline
      \textbf{Test ID:} FR-UI2 \\
      \textbf{Control:} Manual \\
      \textbf{Initial State:} Application is running and idle. \\
      \textbf{Input:} Text-search queries. \\
      \textbf{Expected Output:} Navigation to appropriate documentation/user guide sections. \\
      \textbf{Actual Output:} N/A - User manual will be created at a later stage. \\
      \textbf{Test Case Derivation:} For use during user testing to meet fit criteria for functional requirement FR-UI2 in section 
      9.4 of the \href{https://github.com/emilyperica/ScoreGen/blob/main/docs/SRS-Volere/SRS.pdf}{SRS}. \\
      \textbf{Justification:} Further supports usability and humanity testing (see Section 4.2.2). Without documentation, 
      users may find navigating and using the appliation difficult.\\
      \textbf{How Test Will Be Performed:}
      \begin{enumerate}
          \item Navigate to the location of user documentation in the application.
          \item Perform a broad text search for major application features or for anything the user requires clarification on.
      \end{enumerate}
      \textbf{Result:} N/A
    
    \item \textbf{Test User Feedback Report Mechanism} \\
      \newline
      \textbf{Test ID:} FR-UI3 \\
      \textbf{Control:} Automatic \\
      \textbf{Initial State:} Application is running and idle, prepared to receive input through the feedback mechanism. \\
      \textbf{Input:} Plain text. \\
      \textbf{Expected Output:} GUI submission success cue and message, return of the input text. \\
      \textbf{Actual Output:} N/A - mechanism will be implemented at a later stage. \\
      \textbf{Test Case Derivation:} For validation of functional requirement FR-UI3 in section 9.4 of the 
      \href{https://github.com/emilyperica/ScoreGen/blob/main/docs/SRS-Volere/SRS.pdf}{SRS}. \\
      \textbf{Justification:} Ensures users are able to submit basic feedback or issues and that the application provides adequate
      feedback in the form of confirmation of a successful submission.\\
      \textbf{How Test Will Be Performed:}
      \begin{enumerate}
          \item Submit a plain text report via the user feedback mechanism to the development team (e.g., through email).
          \item Parse through inbox messages for user feedback reports and confirm the reception of the submitted report and 
          entire plain text input.
      \end{enumerate}
      \textbf{Result:} N/A
  \end{enumerate}

  \subsection{Save/Load}
  \begin{enumerate}
    \item \textbf{Test Application Save Function} \\
      \newline
      \textbf{Test ID:} FR-SL1 \\
      \textbf{Control:} Manual \\
      \textbf{Initial State:} Post-audio processing state with generated sheet music file(s) and original audio files accessible 
      for download. \\
      \textbf{Input:} Request to save file(s) to local drive. \\
      \textbf{Expected Output:} Non-corrupt file(s) in destination directories. \\
      \textbf{Actual Output:} Non-corrupt file(s) in destination directories. \\
      \textbf{Test Case Derivation:} For validation of functional requirement FR-SL1 in section 9.5 of the 
      \href{https://github.com/emilyperica/ScoreGen/blob/main/docs/SRS-Volere/SRS.pdf}{SRS}. \\
      \textbf{Justification:} Confirms that users are able to track their progress and save sheet music and audio files to their local 
      storage without data corruption.\\
      \textbf{How Test Was Performed:}
      \begin{enumerate}
          \item Transcribe a sample audio file using the application.
          \item Request to download both the original audio and the generated sheet music to a directory on the local drive.
          \item Compare the contents of the downloaded files to their original copies to check for consistency.
      \end{enumerate}
      \textbf{Result:} PASS
    
    \item \textbf{Test Application’s Ability to Load Existing Files} \\
      \newline
      \textbf{Test ID:} FR-SL2 \\
      \textbf{Control:} Manual \\
      \textbf{Initial State:} Application running and idle, waiting for user input to modify and/or view. \\
      \textbf{Input:} An existing audio file or existing file containing previously generated sheet music. \\
      \textbf{Expected Output:} Successful loading of files without errors. \\
      \textbf{Actual Output:} Successful loading of files without errors.\\
      \textbf{Test Case Derivation:} For validation of functional requirement FR-SL2 in section 9.5 of the 
      \href{https://github.com/emilyperica/ScoreGen/blob/main/docs/SRS-Volere/SRS.pdf}{SRS}. \\
      \textbf{Justification:} Complements test case FR-SL1. Ensures users can load previously saved sheet music or 
      audio files into the application without errors.\\
      \textbf{How Test Was Performed:}
      \begin{enumerate}
          \item Request to load an existing file on the local drive.
          \item Manually inspect the opened file in the editor to ensure the files appear as they were saved.
      \end{enumerate}
      \textbf{Result:} PASS
  \end{enumerate}

\section{Nonfunctional Requirements Evaluation}
Nonfunctional and usability testing was carried out by 4 of Jackson's friends (kept anonymous for privacy). 
They will be referred to as testers 1, 2, 3, and 4 in this section.

\subsubsection{Look and Feel Testing}
\begin{enumerate}
    \item \textbf{Test for LF-A1 Discoverability} \\
      \newline
      \textbf{Test ID:} LF-A1 \\
      \textbf{Type:} Usability, Dynamic, Manual \\
      \textbf{Initial State:} The application is launched, showing the main interface. \\
      \textbf{Input/Condition:} The user views the interface for the first time without guidance. \\
      \textbf{Output/Result:} 75\% of users should be able to locate the interactive element that initiates the score generation process. \\
      \textbf{How Test Was Performed:}
      \begin{enumerate}
          \item Recruit a sample group of users unfamiliar with the app and explain that they need to initiate the score generation 
          process.
          \item Record if each user is able to locate the score generation button within 10 seconds.
          \item Ask users to describe why they selected the specific element as the score generation option.
      \end{enumerate}
      \textbf{Success Criterion:} If 75\% or more of the users identify the correct element, the test is marked as passed.\\
      \textbf{Results:} Pass. Testers 1, 2, 3 and 4 were able to locate the score generation button well within 10 seconds.\\

    \item \textbf{Test for LF-A2 Colour Cohesion} \\
      \newline
      \textbf{Test ID:} LF-A2 \\
      \textbf{Type:} Static, Manual, Visual Inspection \\
      \textbf{Initial State:} The application interface is fully designed. \\
      \textbf{Input/Condition:} Inspect the interface to verify colour usage. \\
      \textbf{Output/Result:} To meet the fit criterion, the colour scheme should have at least three distinct colours: primary, 
      secondary, and accent. \\
      \textbf{How Test Was Performed:}
      \begin{enumerate}
          \item Identify and document the primary, secondary, and accent colours as specified in the app's design guidelines.
          \item Conduct a visual inspection of each screen and interactive element within the app to confirm adherence to these colours.
          \item Verify there are no unintended or extraneous colours used across the app.
      \end{enumerate}
      \textbf{Success Criterion:} If all screens consistently use the defined colour palette, the test passes.\\
      \textbf{Results:} Pass. The app consistently uses the primary colour (white), secondary colour (navy blue), and accent colours (for buttons).\\

    \item \textbf{Test for LF-A3 Resolution} \\
      \newline
      \textbf{Test ID:} LF-A3 \\
      \textbf{Type:} Functional, Manual, Dynamic \\
      \textbf{Initial State:} All graphics are loaded within the application. \\
      \textbf{Input/Condition:} Inspect images to verify resolution. \\
      \textbf{Output/Result:} All graphics should display at 720p resolution or higher. \\
      \textbf{How Test Was Performed:}
      \begin{enumerate}
          \item Compile a list of all graphical assets used in the app, including their file locations.
          \item Use image inspection tools to verify the resolution of each image.
          \item Document the resolution of each graphic and flag any below 720p for replacement.
      \end{enumerate}
      \textbf{Success Criterion:} If all images meet or exceed 720p resolution, the test passes.\\
      \textbf{Results:} Pass. All images used in the app meet or exceed 720p resolution.\\

    \item \textbf{Test for LF-S1 Authority and Trust} \\
      \newline
      \textbf{Test ID:} LF-S1 \\
      \textbf{Type:} Usability, Dynamic, Survey-Based \\
      \textbf{Initial State:} The user has interacted with the application once. \\
      \textbf{Input/Condition:} The user provides feedback on their perception of the app’s authority and trustworthiness. \\
      \textbf{Output/Result:} 70\% of surveyed users report feeling that the application is reliable and trustworthy. \\
      \textbf{How Test Was Performed:}
      \begin{enumerate}
          \item After a user’s initial interaction with the app, provide them with a survey containing the following questions:
          \begin{itemize}
              \item Q1: On a scale of 1-5, how trustworthy does this app feel when handling your data? (1 = Not Trustworthy, 5 = 
              Very Trustworthy)
              \item Q2: Do you feel confident using this app for your music notation needs? (Yes/No)
              \item Q3: Please briefly explain any factors that contributed to your level of trust in the app.
          \end{itemize}
          \item Collect responses and analyze the data to calculate the percentage of users rating the app as 4 or 5 for 
          trustworthiness in Q1 and answering "Yes" to Q2.
      \end{enumerate}
      \textbf{Success Criterion:} If 70\% or more users rate the app 4 or higher on trustworthiness and respond “Yes” to Q2, the 
      test passes.\\
      \textbf{Results:} Pass.
      \begin{enumerate}
        \item Tester 1: Q1: 5, Q2: Yes, Q3: `The app looks professional and modern'.
        \item Tester 2: Q1: 4, Q2: Yes, Q3: `Looks really clean and simple to use'.  
        \item Tester 3: Q1: 5, Q2: Yes, Q3: `Definitely looks professional'.
        \item Tester 4: Q1: 4, Q2: Yes, Q3: `The app is easy to use and looks reliable'.
      \end{enumerate}
\end{enumerate}

\subsubsection{Usability and Humanity Testing}
\begin{enumerate}
    \item \textbf{Test for UH-EOU1 Human-Centered Design (HCD)} \\
      \newline
      \textbf{Test ID:} UH-EOU1 \\
      \textbf{Type:} Usability, Static, Inspection-Based \\
      \textbf{Initial State:} The app interface is fully developed, adhering to HCD principles \citep*{HCD}. \\
      \textbf{Input/Condition:} Inspect the app interface for compliance with the four fundamental HCD principles. \\
      \textbf{Output/Result:} The user interface must visibly incorporate all four HCD principles. \\
      \textbf{How Test Was Performed:}
      \begin{enumerate}
          \item Review the app’s design documentation to verify its adherence to the HCD principles: visibility, consistency, 
          user control, and feedback.
          \item Conduct an inspection-based evaluation of the interface with usability experts, identifying examples of each HCD 
          principle in action.
      \end{enumerate}
      \textbf{Success Criterion:} The app passes if all four HCD principles are clearly implemented and documented within the interface.\\
      \textbf{Results:} Pass. The app interface incorporates all four HCD principles. Examples are: loading spinner for visibility, consistent navigation bar across all pages for consistency,
      user control over score navigation and audio recording, and feedback on successful score generation.\\

    \item \textbf{Test for UH-PI1 Language} \\
      \newline
      \textbf{Test ID:} UH-PI1 \\
      \textbf{Type:} Functional, Static, Manual \\
      \textbf{Initial State:} The app is fully localized with text in Canadian English (en-CA). \\
      \textbf{Input/Condition:} Inspect all text, labels, and instructions in the app. \\
      \textbf{Output/Result:} All text should be presented in Canadian English, with no grammatical or spelling errors. \\
      \textbf{How Test Was Performed:}
      \begin{enumerate}
          \item Conduct a manual review of all textual elements within the app, ensuring they are correctly localized in Canadian 
          English.
          \item Use a grammar and spell-check tool to verify accuracy.
      \end{enumerate}
      \textbf{Success Criterion:} The test is passed if all text is in Canadian English and no errors are found.\\
      \textbf{Results:} Pass. All text in the app is in Canadian English, with no grammatical or spelling errors.\\

    \item \textbf{Test for UH-L1 Music Theory Familiarity} \\
      \newline
      \textbf{Test ID:} UH-L1 \\
      \textbf{Type:} Usability, Dynamic, User Testing \\
      \textbf{Initial State:} The user is unfamiliar with the app and has no formal music theory background. \\
      \textbf{Input/Condition:} The user has 10 minutes to explore and use the app without guidance. \\
      \textbf{Output/Result:} 95\% of users without a music theory background are able to generate a score sheet within the allotted 
      time. \\
      \textbf{How Test Was Performed:}
      \begin{enumerate}
          \item Recruit a sample of users with no formal music theory background.
          \item Allow users 10 minutes to familiarize themselves with the interface.
          \item Observe and document whether each user is able to generate a score sheet within this period.
      \end{enumerate}
      \textbf{Success Criterion:} The test is passed if 95\% or more of the users complete score generation without assistance.\\
      \textbf{Results:} Pass. Testers 1, 2, 3 and 4 were able to generate a score sheet within 5 minutes with no assistance.\\

    \item \textbf{Test for UH-UP1 Icon Identification} \\
      \newline
      \textbf{Test ID:} UH-UP1 \\
      \textbf{Type:} Usability, Dynamic, Survey-Based \\
      \textbf{Initial State:} The app displays interactive elements with unlabeled icons. \\
      \textbf{Input/Condition:} Users attempt to identify the function of each interactive icon. \\
      \textbf{Output/Result:} At least 70\% of icons are accurately identified by users. \\
      \textbf{How Test Was Performed:}
      \begin{enumerate}
          \item Present a sample of users with the app interface and ask them to identify the function of each interactive icon 
          without clicking or interacting with them.
          \item Record user responses and compare them to the correct icon functions.
      \end{enumerate}
      \textbf{Success Criterion:} If at least 70\% of the icons are correctly identified by the majority of users, the test is passed.\\
      \textbf{Results:} Pass. Testers 1, 2, 3 and 4 were able to accurately identify 100\% of the interactive icons.\\

    \item \textbf{Test for UH-UP2 Information Hiding} \\
      \newline
      \textbf{Test ID:} UH-UP2 \\
      \textbf{Type:} Functional, Static \\
      \textbf{Initial State:} The app is fully developed and ready for inspection. \\
      \textbf{Input/Condition:} Inspect the app interface and accessible areas. \\
      \textbf{Output/Result:} No user-accessible elements reveal implementation-specific or algorithmic details. \\
      \textbf{How Test Was Performed:}
      \begin{enumerate}
          \item Conduct a static inspection of all user-accessible components in the interface, including settings, menus, and tooltips.
          \item Confirm that no elements or descriptions expose the internal implementation or processing details.
      \end{enumerate}
      \textbf{Success Criterion:} The test is passed if no user-accessible components reveal underlying algorithms or implementation 
      details.\\
      \textbf{Results:} Pass. No user-accessible elements reveal implementation-specific or algorithmic details.\\

    \item \textbf{Test for UH-A1 Web Content Accessibility Guidelines (WCAG) Compliance} \\
      \newline
      \textbf{Test ID:} UH-A1 \\
      \textbf{Type:} Accessibility, Static and Dynamic, Automated and Manual \\
      \textbf{Initial State:} The app interface is complete and ready for accessibility testing. \\
      \textbf{Input/Condition:} Perform an accessibility audit using automated tools and manual checks. \\
      \textbf{Output/Result:} The app meets or exceeds WCAG 2.1 Level AA compliance \citep*{WCAG21}. \\
      \textbf{How Test Was Performed:}
      \begin{enumerate}
          \item Conduct manual testing for areas not covered by automated tools, such as text readability and interaction.
          \item Document any issues and confirm adherence to all WCAG 2.1 Level AA standards.
      \end{enumerate}
      \textbf{Success Criterion:} The test is passed if the app passes the WCAG 2.1 Level AA standards, confirmed by manual testing.\\
      \textbf{Results:} Fail. The app the app does not yet meet every one of the exhaustive standards from WCAG 2.1.\\
\end{enumerate}

\subsubsection{Performance Testing}
\begin{enumerate}
    \item \textbf{Test for PR-SL1 User Interface Response Time} \\
      \newline
      \textbf{Test ID:} PR-SL1 \\
      \textbf{Type:} Performance, Dynamic \\
      \textbf{Initial State:} The application is open and ready for user interaction. \\
      \textbf{Input/Condition:} Perform multiple interactions, such as menu navigation and input processing. \\
      \textbf{Output/Result:} The app should respond within 2 seconds for 90\% of interactions, with no interaction taking longer 
      than 5 seconds. \\
      \textbf{How Test Was Performed:}
      \begin{enumerate}
          \item Conduct a series of interactions across the application, including navigation through menus and processing various 
          inputs.
          \item Record response times for each interaction.
          \item Calculate the percentage of interactions that respond within 2 seconds and confirm that no response exceeds 5 seconds.
      \end{enumerate}
      \textbf{Success Criterion:} The test is passed if 90\% of interactions respond within 2 seconds and no interaction takes longer
      than 5 seconds.\\
      \textbf{Results:} Pass. The app responds within 2 seconds for 90\% of interactions, with no interaction taking longer than 5 seconds.\\

    \item \textbf{Test for PR-SL2 File Import and Export Speed} \\
      \newline
      \textbf{Test ID:} PR-SL2 \\
      \textbf{Type:} Performance, Dynamic \\
      \textbf{Initial State:} The app is ready to import and export files. \\
      \textbf{Input/Condition:} Test with music files up to 100MB in size for both import and export functions. \\
      \textbf{Output/Result:} The app should complete imports and exports within a reasonable time frame for at least 95\% of operations. \\
      \textbf{How Test Was Performed:}
      \begin{enumerate}
          \item Import and export a range of music files up to 100MB in size, timing each operation. An example music file is given 
          \href{https://github.com/emilyperica/ScoreGen/blob/main/test/TestingDatasets/piano-samples/sample-songs/hot-cross-buns-piano-solo.wav}{here}.
          \item Document the time taken for each operation and calculate the percentage of imports and exports that complete within an acceptable time.
          \item Confirm that 95\% of operations meet the expected time frame.
      \end{enumerate}
      \textbf{Success Criterion:} The test is passed if 95\% of file imports and exports complete within a reasonable time frame.\\
      \textbf{Results:} Pass. The app completes imports and exports within a reasonable time frame for 100\% of our tested operations.\\

    \item \textbf{Test for PR-SC1 Epilepsy Safety} \\
      \newline
      \textbf{Test ID:} PR-SC1 \\
      \textbf{Type:} Accessibility, Static, Automated \\
      \textbf{Initial State:} The application is open with all visual elements loaded. \\
      \textbf{Input/Condition:} Inspect graphical interface elements for compliance with WCAG 2.1 guidelines on flashing content 
      \citep*{WCAG21}. \\
      \textbf{Output/Result:} No visual elements should flash at a rate of more than 3 flashes per second. \\
      \textbf{How Test Was Performed:}
      \begin{enumerate}
          \item Use accessibility tools to scan the interface for flashing content.
          \item Verify that all visual effects adhere to the flash rate limitation.
          \item Document any elements that exceed the flash rate and adjust them to ensure compliance.
      \end{enumerate}
      \textbf{Success Criterion:} The test is passed if no visual elements flash at a rate exceeding 3 flashes per second.\\
      \textbf{Results:} Pass. No visual elements in the app flash at a rate exceeding 3 flashes per second.\\

    \item \textbf{Test for PR-SC2 Instrument Input Setup} \\
      \newline
      \textbf{Test ID:} PR-SC2 \\
      \textbf{Type:} Functional, Usability \\
      \textbf{Initial State:} The app is open, with the instrument input configuration tutorial available. \\
      \textbf{Input/Condition:} Use the tutorial to set up an external instrument or microphone. \\
      \textbf{Output/Result:} The setup should be completed successfully on the first attempt with a 95\% success rate across 
      user testing. \\
      \textbf{How Test Was Performed:}
      \begin{enumerate}
          \item Guide a sample group of users through the step-by-step input configuration tutorial.
          \item Track the completion success rate for each user on their first attempt.
          \item Confirm that at least 95\% of users complete the setup successfully on their first attempt.
      \end{enumerate}
      \textbf{Success Criterion:} The test is passed if 95\% of users complete the setup successfully on their first attempt.\\
      \textbf{Results:} Pass. Testers 1, 2, 3 and 4 were able to successfully set up an external instrument or microphone on their first attempt.\\

    \item \textbf{Test for PR-PA1 Pitch Detection Accuracy} \\
      \newline
      \textbf{Test ID:} PR-PA1 \\
      \textbf{Type:} Functional, Performance \\
      \textbf{Initial State:} The app is configured for audio input from diverse instruments. \\
      \textbf{Input/Condition:} Test pitch detection with a range of instruments and note ranges. Test samples found 
      \href{https://github.com/emilyperica/ScoreGen/tree/main/test/TestingDatasets}{here.}\\
      \textbf{Output/Result:} The pitch detection accuracy should be within a 1\% error margin across all tests. \\
      \textbf{How Test Was Performed:}
      \begin{enumerate}
          \item Play or record test audio samples from various instruments covering diverse note ranges.
          \item Measure pitch detection accuracy by comparing transcribed notes to the actual pitches.
          \item Calculate the error rate and confirm it remains below 1\%.
      \end{enumerate}
      \textbf{Success Criterion:} The test is passed if the pitch detection accuracy remains within a 1\% error margin.\\
      \textbf{Results:} Pass. The pitch detection accuracy is 100\% accurate for monophonic computer-generated audio.\\

    \item \textbf{Test for PR-PA2 Timing Accuracy} \\
      \newline
      \textbf{Test ID:} PR-PA2 \\
      \textbf{Type:} Functional, Performance \\
      \textbf{Initial State:} The app is set to capture audio input. \\
      \textbf{Input/Condition:} Test with audio samples that have precise note durations and rhythms. \\
      \textbf{Output/Result:} The app should capture note durations and rhythms with an accuracy tolerance within 100ms. \\
      \textbf{How Test Was Performed:}
      \begin{enumerate}
          \item Play or record 
          \href{https://github.com/emilyperica/ScoreGen/tree/main/test/TestingDatasets/piano-samples/sample-songs}{audio samples} 
          with known timing and rhythm.
          \item Analyze the transcribed timing for each note and compare it to the original timing.
          \item Verify that timing discrepancies are within the 100ms tolerance limit.
      \end{enumerate}
      \textbf{Success Criterion:} The test is passed if the app captures note durations and rhythms within a 100ms accuracy tolerance.\\
      \textbf{Results:} Pass. The app captures note durations and rhythms within a 100ms accuracy tolerance.\\

    \item \textbf{Test for PR-RFT1 Reliability (Time Between Failures)} \\
      \newline
      \textbf{Test ID:} PR-RFT1 \\
      \textbf{Type:} Performance, Dynamic \\
      \textbf{Initial State:} The app is running under typical usage conditions. \\
      \textbf{Input/Condition:} Operate the app continuously for 24 hours. \\
      \textbf{Output/Result:} The app should function without crashes or failures for the entire 24-hour period in at least 
      95\% of cases. \\
      \textbf{How Test Was Performed:}
      \begin{enumerate}
          \item Start the app under standard usage conditions and monitor it for 24 hours.
          \item Track and log any crashes or failures.
          \item Confirm that at least 95\% of the tests meet the requirement of continuous operation without interruption.
      \end{enumerate}
      \textbf{Success Criterion:} The test is passed if the app functions without crashes or failures for the entire 24-hour period
      in at least 95\% of cases. \\
      \textbf{Results:} On hold until a 24 hour period is available for testing.\\
      
    \item \textbf{Test for PR-RFT2 Availability (Uptime)} \\
      \newline
      \textbf{Test ID:} PR-RFT2 \\
      \textbf{Type:} Performance, Static \\
      \textbf{Initial State:} The app is installed and operational over an extended period. \\
      \textbf{Input/Condition:} Monitor app uptime over several weeks. \\
      \textbf{Output/Result:} The app should demonstrate 99.5\% uptime, accounting for any brief planned downtime. \\
      \textbf{How Test Was Performed:}
      \begin{enumerate}
          \item Log the app’s operational status over an extended period.
          \item Calculate the percentage of time the app is accessible and operational.
          \item Verify that uptime meets or exceeds 99.5\%.
      \end{enumerate}
      \textbf{Success Criterion:} The test is passed if the app demonstrates 99.5\% uptime over the monitoring period.\\
      \textbf{Results:} On hold until full monitoring period is complete.\\

    \item \textbf{Test for PR-RFT3 Crash Recovery} \\
      \newline
      \textbf{Test ID:} PR-RFT3 \\
      \textbf{Type:} Functional, Dynamic \\
      \textbf{Initial State:} The app is open and in active use with an ongoing transcription. \\
      \textbf{Input/Condition:} Simulate an unexpected crash. \\
      \textbf{Output/Result:} The app should automatically save the current session and allow users to recover their work 
      upon restarting, achieving 98\% recovery success. \\
      \textbf{How Test Was Performed:}
      \begin{enumerate}
          \item Begin a transcription session and simulate a crash.
          \item Reopen the app and check if the session is restored, including any partially transcribed sheet music.
          \item Confirm that data recovery occurs in at least 98\% of test cases.
      \end{enumerate}
      \textbf{Success Criterion:} The test is passed if the app automatically saves the current session and allows users to recover
      their work upon restarting in at least 98\% of cases.\\
      \textbf{Results:} On hold until a crash can be simulated safely.\\

      \item \textbf{Test for PR-RFT4 Performance Under Load} \\
      \newline
      \textbf{Test ID:} PR-RFT4 \\
      \textbf{Type:} Performance, Dynamic \\
      \textbf{Initial State:} The app is running and ready to process complex or large inputs. \\
      \textbf{Input/Condition:} Test the app’s performance under maximum load conditions (e.g., 
      \href{https://github.com/emilyperica/ScoreGen/tree/main/test/TestingDatasets/piano-samples/sample-chords}{complex polyphonic inputs, large files}). \\
      \textbf{Output/Result:} The app should maintain stable performance with no more than a 30\% decrease in processing speed. \\
      \textbf{How Test Was Performed:}
      \begin{enumerate}
          \item Load the app with large or complex audio files that simulate high activity conditions.
          \item Measure processing speed and performance metrics during this test.
          \item Verify that performance remains stable with no significant slowdowns or crashes and that any speed decrease 
          is within 30\%.
      \end{enumerate}
      \textbf{Success Criterion:} The test is passed if the app maintains stable performance with no more than a 30\% decrease in processing speed.\\
      \textbf{Results:} Fail. The app experienced a large decrease in processing speed under maximum load conditions. Further optimization and analysis of feasilbility is required.\\

    \item \textbf{Test for PR-RFT5 Handling Signal Interruptions} \\
      \newline
      \textbf{Test ID:} PR-RFT5 \\
      \textbf{Type:} Functional, Dynamic \\
      \textbf{Initial State:} The app is receiving audio input from an external instrument or microphone. \\
      \textbf{Input/Condition:} Temporarily disconnect and then reconnect the audio input device. \\
      \textbf{Output/Result:} The app should retain buffered audio data for up to 2 minutes during interruptions and resume 
      transcription seamlessly. \\
      \textbf{How Test Was Performed:}
      \begin{enumerate}
          \item Begin an audio input session and then simulate a signal interruption by disconnecting the audio source.
          \item Wait up to 2 minutes, then reconnect the audio source.
          \item Verify that buffered audio data is retained, and transcription resumes without any data loss in at least 95\% 
          of test cases.
      \end{enumerate}
      \textbf{Success Criterion:} The test is passed if the app retains buffered audio data for up to 2 minutes during interruptions
      and resumes transcription seamlessly in at least 95\% of cases.\\
      \textbf{Results:} Pass. The app retains buffered audio data for up to 2 minutes during network interruptions and resumes transcription seamlessly.\\

    \item \textbf{Test for PR-RFT6 Graceful Degradation} \\
      \newline
      \textbf{Test ID:} PR-RFT6 \\
      \textbf{Type:} Performance, Dynamic \\
      \textbf{Initial State:} The app is running and under increasing system load. \\
      \textbf{Input/Condition:} Apply stress by loading multiple instruments or large files until performance begins to degrade. \\
      \textbf{Output/Result:} The app should notify the user of delays within 5 seconds of detection and avoid crashing in 99\% 
      of test cases. \\
      \textbf{How Test Was Performed:}
      \begin{enumerate}
          \item Gradually increase system load by adding complex inputs or running multiple processes.
          \item Observe if the app notifies the user within 5 seconds when performance slows.
          \item Confirm the app remains operational without crashing in at least 99\% of test cases.
      \end{enumerate}
      \textbf{Success Criterion:} The test is passed if the app notifies the user of delays within 5 seconds and avoids crashing in at least 99\% of cases.\\
      \textbf{Results:} Pass. The app notifies the user of delays with spinner within 5 seconds and avoids crashing in 100\% of our tested cases.\\

    \item \textbf{Test for PR-RFT7 Automatic Recovery from Software Glitches} \\
      \newline
      \textbf{Test ID:} PR-RFT7 \\
      \textbf{Type:} Functional, Dynamic \\
      \textbf{Initial State:} The app is running and processes files with varied input types. \\
      \textbf{Input/Condition:} Introduce minor software errors (e.g., 
      \href{https://github.com/emilyperica/ScoreGen/tree/main/test/TestingDatasets/sample-formats}{unexpected input format}). \\
      \textbf{Output/Result:} The app should log the error, notify the user, skip the problematic section, and continue 
      without crashing in 90\% of cases. \\
      \textbf{How Test Was Performed:}
      \begin{enumerate}
          \item Feed the app corrupted or incompatible files.
          \item Verify that the app logs the error, notifies the user, and skips the problematic section.
          \item Confirm that the app continues functioning without crashing in at least 90\% of test cases.
      \end{enumerate}
      \textbf{Success Criterion:} The test is passed if the app logs errors, notifies the user, skips problematic sections, and continues
      without crashing in at least 90\% of cases.\\
      \textbf{Results:} Pass. The app logs errors, notifies the user, and continues without crashing in 100\% of our tested cases.\\

    \item \textbf{Test for PR-C1 Audio Input Capacity} \\
      \newline
      \textbf{Test ID:} PR-C1 \\
      \textbf{Type:} Performance, Dynamic \\
      \textbf{Initial State:} The app is ready to record audio input. \\
      \textbf{Input/Condition:} Record audio continuously for up to 90 minutes. \\
      \textbf{Output/Result:} The app should process and transcribe the entire duration with no more than a 5\% slowdown 
      in speed or accuracy. \\
      \textbf{How Test Was Performed:}
      \begin{enumerate}
          \item Start recording and let the app run continuously for 90 minutes.
          \item Monitor transcription speed and accuracy.
          \item Verify that performance remains consistent and within the 5\% slowdown limit.
      \end{enumerate}
      \textbf{Success Criterion:} The test is passed if the app processes and transcribes the entire 90-minute duration with no more than a 5\% slowdown in speed or accuracy.\\
      \textbf{Results:} Fail. The app experienced a large slowdown in speed and accuracy during the 90-minute test. Further optimization and analysis of feasibility is required.\\

    \item \textbf{Test for PR-C2 Simultaneous User Sessions} \\
      \newline
      \textbf{Test ID:} PR-C2 \\
      \textbf{Type:} Performance, Dynamic \\
      \textbf{Initial State:} The app supports simultaneous user sessions. \\
      \textbf{Input/Condition:} Run 10 simultaneous active user sessions. \\
      \textbf{Output/Result:} The app should maintain stable performance, response times, and transcription accuracy in 
      95\% of cases. \\
      \textbf{How Test Was Performed:}
      \begin{enumerate}
          \item Initiate 10 user sessions, each running different tasks.
          \item Monitor performance metrics for each session, including response time and accuracy.
          \item Confirm that performance remains stable with no significant slowdown in 95\% of test cases.
      \end{enumerate}
      \textbf{Success Criterion:} The test is passed if the app maintains stable performance, response times, and transcription accuracy in 95\% of cases.\\
      \textbf{Results:} Pass. The app did not slow down at all with multiple user sessions across different devices.\\

    \item \textbf{Test for PR-SE2 Feature Extensibility} \\
      \newline
      \textbf{Test ID:} PR-SE2 \\
      \textbf{Type:} Structural, Static \\
      \textbf{Initial State:} The app’s codebase is available for review. \\
      \textbf{Input/Condition:} Evaluate the app’s modular architecture for future extensibility. \\
      \textbf{Output/Result:} The app’s design should allow feature additions without major refactoring. \\
      \textbf{How Test Was Performed:}
      \begin{enumerate}
          \item Conduct a code review with development and architecture teams.
          \item Assess the modularity of the codebase and document potential areas for future feature integration.
          \item Confirm that new modules could be added with minimal impact on core functionalities.
      \end{enumerate}
      \textbf{Success Criterion:} The test is passed if the app’s design allows for feature additions without major refactoring.\\
      \textbf{Results:} Pass. After an internal code review, the app's design allows for feature additions without major refactoring.\\

    \item \textbf{Test for PR-SE3 Processing Power for Complex Music Compositions} \\
      \newline
      \textbf{Test ID:} PR-SE3 \\
      \textbf{Type:} Performance, Scalability \\
      \textbf{Initial State:} The app is prepared to process complex musical compositions. \\
      \textbf{Input/Condition:} Process 
      \href{https://github.com/emilyperica/ScoreGen/tree/main/test/TestingDatasets/piano-samples/sample-chords}{increasingly complex compositions} with multiple instrument tracks. \\
      \textbf{Output/Result:} The app should maintain stability and performance as complexity increases. \\
      \textbf{How Test Was Performed:}
      \begin{enumerate}
          \item Gradually add instrument tracks and polyphonic elements to a composition.
          \item Monitor processing speed and stability.
          \item Confirm that the app handles increased complexity without significant performance issues.
      \end{enumerate}
      \textbf{Success Criterion:} The test is passed if the app maintains stability and performance as complexity increases.\\
      \textbf{Results:} Fail. The app scope has been limited to monophonic audio for now. This test is preserved for future development.\\

    \item \textbf{Test for PR-L1 Expected Lifetime} \\
      \newline
      \textbf{Test ID:} PR-L1 \\
      \textbf{Type:} Structural, Static \\
      \textbf{Initial State:} The app’s development roadmap is complete. \\
      \textbf{Input/Condition:} Review the roadmap for planned updates and feature expansions over the next five years. \\
      \textbf{Output/Result:} The roadmap should ensure that the app remains functional and relevant for at least five 
      years with minor maintenance. \\
      \textbf{How Test Was Performed:}
      \begin{enumerate}
          \item Review the app’s development roadmap with the team.
          \item Verify that updates, feature expansions, and maintenance plans are documented.
          \item Confirm that no major rewrites or overhauls are anticipated to keep the app functional and relevant.
      \end{enumerate}
      \textbf{Success Criterion:} The test is passed if the roadmap ensures the app remains functional and relevant for at least five years with minor maintenance.\\
      \textbf{Results:} On hold until the app's development roadmap is complete.\\
    \end{enumerate}

\subsubsection{Operational and Environmental Requirements Testing}
\begin{enumerate}
    \item \textbf{Test for OE-EP1 Operating Environment} \\
      \newline
      \textbf{Test ID:} OE-EP1 \\
      \textbf{Type:} Functional, Dynamic, Environmental \\
      \textbf{Initial State:} The app is running on a personal computer or laptop in a controlled indoor setting. \\
      \textbf{Input/Condition:} The app is operated in various typical indoor environments. \\
      \textbf{Output/Result:} The app should perform consistently without requiring any adjustments based on the environment. \\
      \textbf{How Test Was Performed:}
      \begin{enumerate}
          \item Set up the app in indoor environments such as a home studio, classroom, and office.
          \item Record observations of the app's functionality in each environment, noting any issues related to lighting 
          or ambient noise.
          \item Confirm that the app functions as expected across all tested environments.
      \end{enumerate}
      \textbf{Success Criterion:} The test is passed if the app performs consistently without requiring adjustments based on the environment.\\
      \textbf{Results:} Pass. The app performs consistently across all tested indoor environments.\\

    \item \textbf{Test for OE-EP2 Noise and Audio Input Quality} \\
      \newline
      \textbf{Test ID:} OE-EP2 \\
      \textbf{Type:} Functional, Dynamic, Environmental \\
      \textbf{Initial State:} The app is configured to receive audio input in an environment with moderate 
      \href{https://github.com/emilyperica/ScoreGen/tree/main/test/TestingDatasets/noise-samples}{background noise}\citep*{noise}. \\
      \textbf{Input/Condition:} The app’s transcription accuracy is tested in environments with varying ambient noise levels. \\
      \textbf{Output/Result:} The app’s transcription accuracy should not degrade by more than 5\%. \\
      \textbf{How Test Was Performed:}
      \begin{enumerate}
          \item Play a controlled background noise source in an environment and start the audio capture process.
          \item Record the transcription accuracy in the presence of moderate ambient noise.
          \item Calculate the error rate, confirming it remains within the acceptable threshold of no more than a 5\% drop in accuracy.
      \end{enumerate}
      \textbf{Success Criterion:} The test is passed if the app’s transcription accuracy does not degrade by more than 5\% in environments with varying ambient noise levels.\\
      \textbf{Results:} Fail. The app's transcription accuracy degrades with ambient noise unless direct piano-to-audio input is used.\\

    \item \textbf{Test for OE-EP3 Workspace Flexibility} \\
      \newline
      \textbf{Test ID:} OE-EP3 \\
      \textbf{Type:} Usability, Dynamic, Environmental \\
      \textbf{Initial State:} The app is installed on devices with various screen sizes and resolutions, from 13-inch to 
      27-inch displays. \\
      \textbf{Input/Condition:} The app is run on screens of varying sizes to check layout flexibility. \\
      \textbf{Output/Result:} The app’s interface should be adaptable to each screen size and resolution, maintaining 
      full functionality. \\
      \textbf{How Test Was Performed:}
      \begin{enumerate}
          \item Open the app on devices with screen sizes ranging from 13 to 27 inches.
          \item Check the layout, navigation, and accessibility of interactive elements on each screen size.
          \item Verify that all interactive elements remain visible and functional across screen sizes.
      \end{enumerate}
      \textbf{Success Criterion:} The test is passed if the app’s interface is adaptable to each screen size and resolution, maintaining full functionality.\\
      \textbf{Results:} Pass. The app's interface is adaptable to each screen size and resolution, maintaining full functionality.\\

    \item \textbf{Test for OE-EP4 Portable Setup Compatibility} \\
      \newline
      \textbf{Test ID:} OE-EP4 \\
      \textbf{Type:} Usability, Dynamic, Environmental \\
      \textbf{Initial State:} The app is installed on a laptop in a temporary workspace, such as a cafe or live performance venue. \\
      \textbf{Input/Condition:} The app is used in transient environments to evaluate setup and operation without specialized 
      hardware. \\
      \textbf{Output/Result:} The app should function smoothly on standard laptop hardware, performing effectively in transient 
      conditions. \\
      \textbf{How Test Was Performed:}
      \begin{enumerate}
          \item Set up the app on a laptop with standard specifications in a cafe or similar temporary workspace.
          \item Test the app’s functionality, including quick setup and breakdown, ensuring smooth operation.
          \item Document any usability issues encountered during setup and breakdown in the transient environment.
      \end{enumerate}
      \textbf{Success Criterion:} The test is passed if the app functions smoothly on standard laptop hardware, performing effectively in transient conditions.\\
      \textbf{Results:} Pass. The app functions smoothly on standard laptop hardware in transient environments.\\

    \item \textbf{Test for OE-WE1 General Hardware} \\
      \newline
      \textbf{Test ID:} OE-WE1 \\
      \textbf{Type:} Usability, Dynamic, Environmental \\
      \textbf{Initial State:} The app is installed on devices with varying screen interfaces and sizes. \\
      \textbf{Input/Condition:} The app is run on screens from 13 to 27 inches with different resolutions. \\
      \textbf{Output/Result:} The app should be fully functional, maintaining usability across all screen sizes and resolutions. \\
      \textbf{How Test Was Performed:}
      \begin{enumerate}
          \item Launch the app on devices with screen sizes ranging from 13 to 27 inches.
          \item Confirm that all UI elements are accessible, functional, and scale correctly on each screen size.
          \item Record any deviations in layout or usability and confirm functionality across all specified screen configurations.
      \end{enumerate}
      \textbf{Success Criterion:} The test is passed if the app is fully functional, maintaining usability across all screen sizes and resolutions.\\
      \textbf{Results:} Pass. The app is fully functional and maintains usability across all screen sizes and resolutions.\\

    \item \textbf{Test for OE-IA1 Audio Input Devices} \\
      \newline
      \textbf{Test ID:} OE-IA1 \\
      \textbf{Type:} Functional, Dynamic \\
      \textbf{Initial State:} The app is installed and configured to receive audio input. \\
      \textbf{Input/Condition:} Connect standard audio input devices (USB microphone, instrument pickup, built-in microphone). \\
      \textbf{Output/Result:} The app should support audio input from USB Audio Class 1.0 and higher devices. \\
      \textbf{How Test Was Performed:}
      \begin{enumerate}
          \item Connect various standard audio input devices, including USB microphones, instrument pickups, and built-in microphones.
          \item Record audio using each device and monitor the app’s performance.
          \item Verify that audio is captured successfully from all tested devices without connectivity issues.
      \end{enumerate}
      \textbf{Success Criterion:} The test is passed if the app supports audio input from USB Audio Class 1.0 and higher devices.\\
      \textbf{Results:} Pass. The app supports audio input from USB Audio Class 1.0 and higher devices.\\

    \item \textbf{Test for OE-IA2 Audio File Import and Export} \\
      \newline
      \textbf{Test ID:} OE-IA2 \\
      \textbf{Type:} Functional, Dynamic \\
      \textbf{Initial State:} The app is running and ready to import and export audio files. \\
      \textbf{Input/Condition:} Use sample audio files in WAV (PCM) and MP3 (MPEG-1 Layer III) formats for import and export. \\
      \textbf{Output/Result:} The app should successfully import and export audio files in WAV and MP3 formats. \\
      \textbf{How Test Was Performed:}
      \begin{enumerate}
          \item Import WAV and MP3 files into the app and verify playback or analysis accuracy.
          \item Export audio files in WAV and MP3 formats, ensuring they are playable in standard audio players.
          \item Confirm that imported and exported files retain their quality and format specifications.
      \end{enumerate}
      \textbf{Success Criterion:} The test is passed if the app successfully imports and exports audio files in WAV and MP3 formats.\\
      \textbf{Results:} Pass. The scope of the app has been limited to WAV format only, with MP3 format support preserved for future development.\\

    \item \textbf{Test for OE-IA3 Music Notation Software Integration} \\
      \newline
      \textbf{Test ID:} OE-IA3 \\
      \textbf{Type:} Functional, Dynamic \\
      \textbf{Initial State:} The app has completed audio processing and is ready to export sheet music. \\
      \textbf{Input/Condition:} Generate sheet music and export in MusicXML format. \\
      \textbf{Output/Result:} The exported files should be compatible with popular music notation software. \\
      \textbf{How Test Was Performed:}
      \begin{enumerate}
          \item Convert an \href{https://github.com/emilyperica/ScoreGen/tree/main/test/TestingDatasets/piano-samples/sample-songs}{audio sample} to sheet music within the app.
          \item Export the generated sheet music as a MusicXML file.
          \item Test the exported files in another music notation software (TBD) to ensure compatibility and accuracy.
      \end{enumerate}
      \textbf{Success Criterion:} The test is passed if the exported files are compatible with popular music notation software.\\
      \textbf{Results:} Pass. The exported musicxml files are compatible with popular music notation software.\\

    \item \textbf{Test for OE-P1 Distribution} \\
      \newline
      \textbf{Test ID:} OE-P1 \\
      \textbf{Type:} Functional, Static \\
      \textbf{Initial State:} The app is packaged and ready for distribution. \\
      \textbf{Input/Condition:} Package the app as a downloadable installer. \\
      \textbf{Output/Result:} The installer should be downloadable from a website or repository without additional dependencies. \\
      \textbf{How Test Was Performed:}
      \begin{enumerate}
          \item Package the app into an executable installer.
          \item Upload the installer to a hosting site (TBD) and download it on test devices.
          \item Verify that installation completes successfully with no dependencies required.
      \end{enumerate}
      \textbf{Success Criterion:} The test is passed if the installer is downloadable from a website or repository without additional dependencies.\\
      \textbf{Results:} Fail. The app is not yet packaged for distribution.\\

    \item \textbf{Test for OE-P2 Installation Process} \\
      \newline
      \textbf{Test ID:} OE-P2 \\
      \textbf{Type:} Usability, Functional \\
      \textbf{Initial State:} The app installer is ready for use. \\
      \textbf{Input/Condition:} Begin installation with minimal technical guidance. \\
      \textbf{Output/Result:} The installation should be simple and guide users effectively. \\
      \textbf{How Test Was Performed:}
      \begin{enumerate}
          \item Run the installer and proceed through the installation process.
          \item Confirm that instructions are clear and that users can install the app with minimal input.
          \item Verify that the app installs correctly and launches without issues.
      \end{enumerate}
      \textbf{Success Criterion:} The test is passed if the installation is simple and guides users effectively.\\
      \textbf{Results:} Fail. The app is not yet packaged for distribution.\\

    \item \textbf{Test for OE-P3 Size and Compatibility} \\
      \newline
      \textbf{Test ID:} OE-P3 \\
      \textbf{Type:} Functional, Static \\
      \textbf{Initial State:} The app installation file is prepared. \\
      \textbf{Input/Condition:} Check the installation file size and install it on various systems. \\
      \textbf{Output/Result:} The installation file should be 500MB or less, and the app should be compatible with different 
      system setups. \\
      \textbf{How Test Was Performed:}
      \begin{enumerate}
          \item Confirm the installation file size does not exceed 500MB.
          \item Install the app on multiple devices with different operating systems and hardware specifications.
          \item Ensure the app operates smoothly across all tested systems.
      \end{enumerate}
      \textbf{Success Criterion:} The test is passed if the installation file is 500MB or less and the app is compatible with different system setups.\\
      \textbf{Results:} Fail. The app is not yet packaged for distribution.\\

    \item \textbf{Test for OE-P4 Post-Installation Configuration} \\
      \newline
      \textbf{Test ID:} OE-P4 \\
      \textbf{Type:} Usability, Functional \\
      \textbf{Initial State:} The app is installed and launched for the first time. \\
      \textbf{Input/Condition:} Access the settings configuration panel. \\
      \textbf{Output/Result:} Users should be able to modify settings from the configuration panel without editing files manually. \\
      \textbf{How Test Was Performed:}
      \begin{enumerate}
          \item Launch the app and navigate to the settings configuration panel.
          \item Verify that users can adjust the input source and output format through the panel.
          \item Confirm that all settings save correctly and can be accessed again upon reopening.
      \end{enumerate}
      \textbf{Success Criterion:} The test is passed if users can modify settings from the configuration panel without editing files manually.\\
      \textbf{Results:} Fail. The app is not yet packaged for distribution.\\

    \item \textbf{Test for OE-R1 Initial Release} \\
      \newline
      \textbf{Test ID:} OE-R1 \\
      \textbf{Type:} Functional, Static \\
      \textbf{Initial State:} The app has completed development and quality assurance. \\
      \textbf{Input/Condition:} Conduct a final testing phase to ensure all core functionalities are operational. \\
      \textbf{Output/Result:} Core features should be functional with no major bugs, and ready for initial release. \\
      \textbf{How Test Was Performed:}
      \begin{enumerate}
          \item Perform a full quality assurance test on the app, focusing on core functionality such as audio-to-sheet music 
          conversion in a production environment.
          \item Document any bugs or issues and resolve them before release.
          \item Verify that the app is stable and ready for general use by conducting a final user acceptance test.
      \end{enumerate}
      \textbf{Success Criterion:} The test is passed if core features are functional with no major bugs, and the app is ready for initial release.\\
      \textbf{Results:} Pending final demonstration and release.\\

\end{enumerate}

\subsubsection{Maintainability and Support Requirements Tests}
\begin{enumerate}
    \item \textbf{Test for MS-M1 Version Releases} \\
      \newline
      \textbf{Test ID:} MS-M1 \\
      \textbf{Type:} Functional, Dynamic \\
      \textbf{Initial State:} A new version of the app has just been released. \\
      \textbf{Input/Condition:} The user opens the app with an internet connection after a new release. \\
      \textbf{Output/Result:} The user should receive a notification to install the new version. \\
      \textbf{How Test Was Performed:}
      \begin{enumerate}
          \item Release a test version update while the app is active on a connected device.
          \item Open the app on the test device and verify that the user receives a prompt to install the new version.
          \item Document any delays or failures in notification.
      \end{enumerate}
      \textbf{Success Criterion:} The test is passed if the user receives a notification to install the new version.\\
      \textbf{Results:} On hold until the app's development roadmap is complete.\\

    \item \textbf{Test for MS-M2 System Crash} \\
      \newline
      \textbf{Test ID:} MS-M2 \\
      \textbf{Type:} Functional, Dynamic \\
      \textbf{Initial State:} The application is running and unexpectedly shuts down. \\
      \textbf{Input/Condition:} Force a non-user-prompted shutdown of the application. \\
      \textbf{Output/Result:} The app should reopen within 1 minute and recover data from the previous session. \\
      \textbf{How Test Was Performed:}
      \begin{enumerate}
          \item Simulate an unexpected shutdown (e.g., force-close the app).
          \item Verify that the app automatically reboots within 1 minute.
          \item Confirm that data from the last session is preserved and accessible upon reopening.
      \end{enumerate}
      \textbf{Success Criterion:} The test is passed if the app reopens within 1 minute and recovers data from the previous session.\\
      \textbf{Results:} Fail. Session data is not currently being saved.\\

    \item \textbf{Test for MS-S1 Software Bugs} \\
      \newline
      \textbf{Test ID:} MS-S1 \\
      \textbf{Type:} Functional, Dynamic \\
      \textbf{Initial State:} The user encounters a bug and accesses the reporting feature. \\
      \textbf{Input/Condition:} Submit a bug report via the app’s GUI. \\
      \textbf{Output/Result:} The development team should be notified within 1 hour of report submission. \\
      \textbf{How Test Was Performed:}
      \begin{enumerate}
          \item Use the app’s bug reporting feature to submit a test report.
          \item Verify that the report is logged in the development team’s system within 1 hour.
          \item Document any delays or issues in the notification process.
      \end{enumerate}
      \textbf{Success Criterion:} The test is passed if the development team is notified within 1 hour of report submission.\\
      \textbf{Results:} Fail. The app is not yet equipped with a bug reporting feature.\\

    \item \textbf{Test for MS-S2 Operating System} \\
      \newline
      \textbf{Test ID:} MS-S2 \\
      \textbf{Type:} Functional, Static \\
      \textbf{Initial State:} The app is ready to be installed on various Windows versions. \\
      \textbf{Input/Condition:} Install and run the app on devices with Windows 10 and 11. \\
      \textbf{Output/Result:} The app should perform all expected functionalities on both Windows 10 and Windows 11. \\
      \textbf{How Test Was Performed:}
      \begin{enumerate}
          \item Install the app on devices with Windows 10 and Windows 11.
          \item Confirm that all core functionalities (e.g., audio capture, transcription) work as expected on each operating system.
          \item Document any compatibility issues or functional limitations on each OS version.
      \end{enumerate}
      \textbf{Success Criterion:} The test is passed if the app performs all expected functionalities on both Windows 10 and Windows 11.\\
      \textbf{Results:} Pass. The app performs all expected functionalities on both Windows 10 and Windows 11.\\

    \item \textbf{Test for MS-A1 Internet Connection} \\
      \newline
      \textbf{Test ID:} MS-A1 \\
      \textbf{Type:} Functional, Dynamic \\
      \textbf{Initial State:} The app is running on a device with intermittent internet connectivity. \\
      \textbf{Input/Condition:} Test the app’s functionality both online and offline. \\
      \textbf{Output/Result:} The app should provide a consistent user experience regardless of internet connection status. \\
      \textbf{How Test Was Performed:}
      \begin{enumerate}
          \item Launch the app and test core functionalities (e.g., audio processing, and navigation) with a stable internet connection.
          \item Disconnect the internet and continue using the app, verifying that the user experience remains consistent.
          \item Document any discrepancies in functionality when offline.
      \end{enumerate}
      \textbf{Success Criterion:} The test is passed if the app provides a consistent user experience regardless of internet connection status.\\
      \textbf{Results:} Fail. The app relies on network connectivity for core functionalities. Offline mode is not yet supported and will be preserved for future development.\\

    \item \textbf{Test for MS-A2 GUI Navigation} \\
      \newline
      \textbf{Test ID:} MS-A2 \\
      \textbf{Type:} Usability, Functional \\
      \textbf{Initial State:} The app’s GUI is displayed with a connected keyboard and without a mouse. \\
      \textbf{Input/Condition:} Navigate the app’s interface solely using keyboard shortcuts. \\
      \textbf{Output/Result:} The app should remain fully functional and navigable without a mouse. \\
      \textbf{How Test Was Performed:}
      \begin{enumerate}
          \item Disconnect any mouse from the test device and launch the app.
          \item Use keyboard shortcuts to navigate through each major feature and menu within the app.
          \item Confirm that all functionalities are accessible and that navigation is smooth using only the keyboard.
      \end{enumerate}
      \textbf{Success Criterion:} The test is passed if the app remains fully functional and navigable without a mouse.\\
      \textbf{Results:} Pass. The app remains fully functional and navigable without a mouse.\\
\end{enumerate}

\subsubsection{Security Requirement Tests}
\begin{enumerate}
    \item \textbf{Test for S-A1 User Authentication} \\
      \newline
      \textbf{Test ID:} S-A1 \\
      \textbf{Type:} Functional, Static \\
      \textbf{Initial State:} The application is closed. \\
      \textbf{Input/Condition:} Open the application as a standard user without any authentication. \\
      \textbf{Output/Result:} The application should allow access without requiring login credentials or password input. \\
      \textbf{How Test Was Performed:}
      \begin{enumerate}
          \item Launch the application and verify that the user is granted access without needing to enter any login details.
          \item Document whether any authentication prompt appears unexpectedly.
          \item Confirm that the user can freely access all functionalities without an authentication barrier.
      \end{enumerate}
      \textbf{Success Criterion:} The test is passed if the application allows access without requiring login credentials or password input.\\
      \textbf{Results:} Pass. The app allows access without requiring login credentials or password input.\\

    \item \textbf{Test for S-P1 Data Storage} \\
      \newline
      \textbf{Test ID:} S-P1 \\
      \textbf{Type:} Functional, Static \\
      \textbf{Initial State:} The application is running, and the user has chosen a specific location for file storage. \\
      \textbf{Input/Condition:} Save an output file in the user-selected directory. \\
      \textbf{Output/Result:} The output file should be saved in the specified directory with full read and write permissions 
      for the user. \\
      \textbf{How Test Was Performed:}
      \begin{enumerate}
          \item Configure the application to store output files in a user-defined location.
          \item Generate and save an output file, then verify that the file appears in the specified directory.
          \item Check that the user has read and write permissions for the saved file.
      \end{enumerate}
      \textbf{Success Criterion:} The test is passed if the output file is saved in the specified directory with full read and write permissions for the user.\\
      \textbf{Results:} Pass. The output file is saved in the specified directory with full read and write permissions for the user.\\

    \item \textbf{Test for S-P2 PII} \\
      \newline
      \textbf{Test ID:} S-P2 \\
      \textbf{Type:} Functional, Static \\
      \textbf{Initial State:} The application is running. \\
      \textbf{Input/Condition:} Use the application, observing all interactions for data input requests. \\
      \textbf{Output/Result:} The application should not request any Personal Identifiable Information (PII) from the user. \\
      \textbf{How Test Was Performed:}
      \begin{enumerate}
          \item Launch the application and monitor for any prompts requesting personal data.
          \item Interact with all features of the app, verifying that no PII requests (e.g., name, address, or contact 
          information) are made.
          \item Document any instance where PII might be requested, and confirm the app remains free of such prompts.
      \end{enumerate}
      \textbf{Success Criterion:} The test is passed if the application does not request any Personal Identifiable Information (PII) from the user.\\
      \textbf{Results:} Pass. The application does not request any Personal Identifiable Information (PII) from the user.\\

    \item \textbf{Test for S-P3 Input Data} \\
      \newline
      \textbf{Test ID:} S-P3 \\
      \textbf{Type:} Functional, Dynamic \\
      \textbf{Initial State:} The application has processed an audio input file. \\
      \textbf{Input/Condition:} Complete an audio processing session. \\
      \textbf{Output/Result:} The application should clear all caches and temporary files associated with the processed 
      audio upon completion. \\
      \textbf{How Test Was Performed:}
      \begin{enumerate}
          \item Process an \href{https://github.com/emilyperica/ScoreGen/tree/main/test/TestingDatasets/piano-samples}{audio file} 
          within the application.
          \item After processing, check the system’s temporary file storage to ensure no audio data or temporary files remain.
          \item Confirm that all caches are cleared, and no residual audio data is left on the system.
      \end{enumerate}
      \textbf{Success Criterion:} The test is passed if the application clears all caches and temporary files associated with the processed audio upon completion.\\
      \textbf{Results:} Pass. The application clears all caches and temporary files associated with the processed audio upon completion.\\
\end{enumerate}

\subsubsection{Cultural and Compliance Requirements Tests}
\begin{enumerate}
    \item \textbf{Test for CR-CR1 Musical Convention} \\
      \newline
      \textbf{Test ID:} CR-CR1 \\
      \textbf{Type:} Functional, Static \\
      \textbf{Initial State:} The application is open, ready to display and create sheet music. \\
      \textbf{Input/Condition:} Use the app to create and display sheet music, verifying adherence to Western music notation 
      \citep*{music-notation}. \\
      \textbf{Output/Result:} The generated and displayed sheet music should accurately follow Western music notation standards. \\
      \textbf{How Test Was Performed:}
      \begin{enumerate}
          \item Generate a sample sheet music file using the app, incorporating various standard Western notation elements 
          (e.g., treble and bass clefs, notes, rests, time signatures).
          \item Compare the sheet music against standard Western notation guidelines, ensuring proper symbol usage and layout.
          \item Confirm that all notation adheres to Western music conventions without deviations or omissions.
      \end{enumerate}
      \textbf{Success Criterion:} The test is passed if the application generates accurate, standard western sheet music.\\
      \textbf{Results:} Pass. The app generates generates accurate, standard western sheet music.\\

    \item \textbf{Test for CR-CR2 Expected Music Theory Level of Users} \\
      \newline
      \textbf{Test ID:} CR-CR2 \\
      \textbf{Type:} Usability, Dynamic \\
      \textbf{Initial State:} The application is open and available to users with varying levels of music theory knowledge. \\
      \textbf{Input/Condition:} Users with basic to intermediate knowledge of music theory interact with the app. \\
      \textbf{Output/Result:} The app should be intuitive and accessible, allowing users to navigate and use core features comfortably. \\
      \textbf{How Test Will Be Performed:}
      \begin{enumerate}
          \item Recruit a sample group of users with basic to intermediate music theory knowledge.
          \item Have users complete a task (e.g., creating a simple sheet of music) and observe their interaction with the app.
          \item Provide tutorials or guides as needed and gather feedback on their clarity and usefulness.
          \item Verify that users can successfully navigate and use the application without advanced music theory knowledge, 
          meeting usability expectations.
      \end{enumerate}
      \textbf{Success Criterion:} Users can successfully navigate and use the application without advanced music theory knowledge, 
      meeting usability expectations.\\
      \textbf{Results:} Pass. All testers have limited music theory knowledge and were still able to navigate and use the app.\\

    \item \textbf{Test for CR-LR1 Copyright Issues} \\
      \newline
      \textbf{Test ID:} CR-LR1 \\
      \textbf{Type:} Functional, Static \\
      \textbf{Initial State:} The application is open, ready for user interaction. \\
      \textbf{Input/Condition:} Observe the app’s interface for the presence of a copyright disclaimer. \\
      \textbf{Output/Result:} The app should display a clear disclaimer advising users on copyright compliance related to 
      audio input. \\
      \textbf{How Test Will Be Performed:}
      \begin{enumerate}
          \item Open the application and navigate to any sections that mention user responsibilities or usage terms.
          \item Confirm that a disclaimer is present, informing users of copyright restrictions and advising compliance.
          \item Verify that instructions are clear, guiding users on ensuring their audio inputs do not violate copyright laws.
      \end{enumerate}
      \textbf{Success Criterion:} A disclaimer is present, informing users of copyright restrictions and advising compliance.\\
      \textbf{Results:} Fail. Disclaimer not present until copyright law is researched by our team.\\

    \item \textbf{Test for CR-SCR1 Technological Standards} \\
      \newline
      \textbf{Test ID:} CR-SCR1 \\
      \textbf{Type:} Functional, Static \\
      \textbf{Initial State:} The application has a completed sheet music file ready for export. \\
      \textbf{Input/Condition:} Export sheet music in MusicXML format and verify the use of third-party libraries. \\
      \textbf{Output/Result:} The app should export error-free MusicXML files and have documentation of all third-party libraries. \\
      \textbf{How Test Will Be Performed:}
      \begin{enumerate}
          \item Generate a sheet music file within the app and export it in MusicXML format.
          \item Open the exported file in compatible notation software (TBD) to confirm it is error-free and follows the 
          MusicXML standard.
          \item Review the app’s documentation to verify that all third-party libraries are listed with their licensing 
          terms for compliance purposes.
      \end{enumerate}
      \textbf{Success Criterion:} Verify all third party libraries and check that sheet music is error-free.\\
      \textbf{Results:} Pass. Generated sheet music is compliant and all third party libraries have been verified.\\
\end{enumerate}
	
\section{Comparison to Existing Implementation}	

\section{Unit Testing}

This section describes the low-level tests that were conducted to verify that the behaviour of 
functions within the system's modules are correct. The tests were created using the GTest framework.

\subsection{Helper Functions}

\texttt{std::vector<double> generateSineWave(double frequency, double sampleRate, double duration);}
\begin{longtable}{|L{3cm}|L{2cm}|L{2cm}|L{2cm}|l|}
  \caption{generateSineWave Test Results} \\
  \hline
  \textbf{Test} & \textbf{Inputs} & \textbf{Expected Output} & \textbf{Actual Output} & \textbf{Result} \\
  \hline
  Signal Size & 440.0, 44100.0, 1.0 & Vector size of 44100 & Vector size of 44100 & \textcolor{green}{Pass} \\
  \hline
  Amplitude Consistency & 440.0, 44100.0, 1.0 & All samples between -1 and 1.0 & Vector size of 44100 & \textcolor{green}{Pass} \\
  \hline
  Phase Continuity & 440.0, 44100.0, 1.0 & All samples approximately 0 within tolerance & All samples approximately 0 within tolerance & \textcolor{green}{Pass} \\
  \hline
  Zero Duration & 440.0, 44100.0, 0.0 & Empty vector & Vector size of 44100 & \textcolor{green}{Pass} \\
  \hline
  High Frequency & 22050.0, 44100.0, 1.0 & All samples approximately 0 within tolerance & All samples approximately 0 within tolerance & \textcolor{green}{Pass} \\
  \hline
\end{longtable}


\noindent\texttt{float extractFundamentalFrequency(const std::vector<std::vector<double>>\& spectrogram, double sampleRate);}

\begin{longtable}{|L{3cm}|L{2cm}|L{2cm}|L{2cm}|l|}
  \caption{extractFundamentalFrequency Test Results} \\
  \hline
  \textbf{Test} & \textbf{Inputs} & \textbf{Expected Output} & \textbf{Actual Output} & \textbf{Result} \\
  \hline
  Single Frame Peak & \{\{0.0, ..., 1.0, ..., 0.0\}\}, 44100.0 & 100.0 & 100.0 & \textcolor{green}{Pass} \\
  \hline
  Multi-frame Competing Peaks & \{\{0.0, ..., 1.0, ..., 0.0\}, \{0.0, ..., 2.0, ..., 0.0\}\}, 44100.0 & 75.0 & 75.0 & \textcolor{green}{Pass} \\
  \hline
  Empty Spectrogram & \{\}, 44100.0 & 0.0 & 0.0 & \textcolor{green}{Pass} \\
  \hline
  Different Sample Rate & \{\{0.0, ..., 1.0, ..., 0.0\}\}, 16000.0 & 100.0 & 100.0 & \textcolor{green}{Pass} \\
  \hline
\end{longtable}

\subsection{UI Module}
Manual testing by the developers (e.g. visual inspection, etc.) was deemed more efficient and effective for this module. Consequently, unit testing tables for the UI 
module are not included in this report.\\

\subsection{Score Generation Module}
\subsubsection{MusicXML Generation}
\begin{longtable}{|L{3cm}|L{2cm}|L{2cm}|L{2cm}|l|}
  \caption{MusicXML Generation Test Results} \\
  \hline
  \textbf{Test} & \textbf{Inputs} & \textbf{Expected Output} & \textbf{Actual Output} & \textbf{Result} \\
  \hline
  Note Element Creation & 
      Regular: \{"C", 0, 4, 4, "quarter", false\}
      Rest: \{"", 0, 0, 4, "quarter", true\}
      Accidental: \{"C", 1, 4, 4, "quarter", false\} &
    Valid XML note elements (non-null) & Non-null pointers & \textcolor{green}{Pass} \\
  \hline
  Measure Creation & 
      Single note: \{"C", 0, 4, 4, "quarter", false\} 
      Multi-note: \{"C", 0, 4, 4, "quarter", false\}, \{"D", 0, 4, 4, "quarter", false\},
        \{"E", 0, 4, 4, "quarter", false\}, \{"F", 0, 4, 4, "quarter", false\} 
        Mixed: \{"C", 0, 4, 4, "quarter", false\}, \{"", 0, 0, 4, "quarter", true\},
        \{"D", 0, 4, 4, "quarter", false\}, \{"", 0, 0, 8, "half", true\}
    &
    Valid XML measure elements (non-null) & Non-null pointers & \textcolor{green}{Pass} \\
  \hline
  Part Creation & 
    \{"C", 0, 4, 4, "quarter", false; "D", 0, 4, 4, "quarter", false; 
    "E", 0, 4, 4, "quarter", false; "F", 0, 4, 4, "quarter", false; 
    "G", 1, 4, 4, "quarter", false; "", 0, 0, 8, "half", true; 
    "A", 0, 4, 4, "quarter", false; "B", -1, 4, 4, "quarter", false\} &
    Valid XML part element (non-null) & Non-null pointer & \textcolor{green}{Pass} \\
  \hline
  Score Part Creation & 
    N/A & 
    Valid XML score part element (non-null) & Non-null pointer & \textcolor{green}{Pass} \\
  \hline
  All Note Types & 
    -- & 
    Generation of all note types (e.g., dotted notes, rests) & -- & \textcolor{yellow}{Pending} \\
  \hline
  Full Generation & 
      Multiple measures with standard notes, accidentals, rests,
      missing note type, and extreme octaves
    &
      Success code; File generated and non-empty
    &
    True; file non-empty & \textcolor{green}{Pass} \\
  \hline
  Empty Sequence Generation & 
    Empty note sequence & 
    Failure code; No file generated & False; No file & \textcolor{green}{Pass} \\
  \hline
\end{longtable}

\subsection{File Format Conversion Module}
Some submodules of this module were tested manually by the developers visually (e.g. mXML to SVG/PDF). \\ \\
The following tests use a common SF\_INFO configuration (1 channel, 44100 Hz, WAV/PCM\_16) and generate a 440 Hz sine wave of 1-second duration for file I/O operations.

\begin{longtable}{|L{3cm}|L{3cm}|L{2.5cm}|L{2.5cm}|l|}
  \caption{File Format Conversion Test Results} \\
  \hline
  \textbf{Test} & \textbf{Inputs} & \textbf{Expected Output} & \textbf{Actual Output} & \textbf{Result} \\
  \hline
  Read Valid File & 
    Create directory "test-data"; \newline
    SF\_INFO; \newline
    Write sine wave (440 Hz, 1 s) to "test-data/sine.wav" & 
    SF\_INFO fields match; \newline
    Data size = 44100; \newline
    Sine samples (at indices 0, 100, 1234) within floating-point tolerance & 
    SF\_INFO and data verified; \newline
    Data size = 44100; \newline
    Sine sample checks pass & \textcolor{green}{Pass} \\
  \hline
  Read Non-Existent File & 
    File: "fakefile.wav" & 
    Data vector size = 0 & 
    Data vector size = 0 & \textcolor{green}{Pass} \\
  \hline
  Read Write Read Cycle & 
    Create directory "test-data"; \newline
    Write sine wave (440 Hz, 1 s) with SF\_INFO to "test-data/sine.wav"; \newline
    Read file, then write output to "test-data/output.wav"; \newline
    Read output file & 
    SF\_INFO and data of original and output files are identical; \newline
    Data vectors equal element-wise within floating-point tolerance & 
    SF\_INFO and data match between original and output; \newline
    Data elements are not the same within floating-point tolerence & 
    \textcolor{red}{Fail} \\
  \hline
  Write Invalid Location & 
    SF\_INFO; \newline
    Data vector: 1000 samples (each 0.5); \newline
    Invalid directory: "/DNE/" & 
    No fatal failure during write operation & 
    No fatal failure encountered & \textcolor{green}{Pass} \\
  \hline
\end{longtable}

\subsection{Raw Signal Processing Module}
\subsubsection{Pre-processing}
\begin{longtable}{|L{3cm}|L{3cm}|L{2.5cm}|L{2.5cm}|l|}
  \caption{Raw Signal Processing Test Results} \\
  \hline
  \textbf{Test} & \textbf{Inputs} & \textbf{Expected Output} & \textbf{Actual Output} & \textbf{Result} \\
  \hline
  Single Channel Test &
    Data: \{0.5, 1.5, -2.0, 3.0\}; \newline
    Channels: 1 &
    Processed data matches input data &
    Processed data matches input data &
    \textcolor{green}{Pass} \\
  \hline
  Two Channel to Mono Conversion &
    Data: \{1.0, 2.0, 1.1, 2.1, 1.2, 2.2\}; \newline
    Channels: 2 &
    Averaged data: \{1.5, 1.6, 1.7\} &
    Averaged data: \{1.5, 1.6, 1.7\} &
    \textcolor{green}{Pass} \\
  \hline
  Three Channel to Mono Conversion &
    Data: \{1.0, 2.0, 3.0, 1.1, 2.1, 3.1\}; \newline
    Channels: 3 &
    Averaged data: \{2.0, 2.1\} &
    Fatal failure &
    \textcolor{red}{Fail} \\
  \hline
  Empty Data &
    Data: \{\}; \newline
    Channels: 2 &
    Processed data is empty &
    Processed data is empty &
    \textcolor{green}{Pass} \\
  \hline
\end{longtable}

\subsubsection{Fourier Transform and Spectrogram Creation}
\begin{longtable}{|L{3cm}|L{3cm}|L{2.5cm}|L{2.5cm}|l|}
  \caption{Fourier Transform and Spectrogram Creation Test Results} \\
  \hline
  \textbf{Test} & \textbf{Inputs} & \textbf{Expected Output} & \textbf{Actual Output} & \textbf{Result} \\
  \hline
  Extract Sine Wave Frequency &
    Frequency: 440.0 Hz (A4); \newline
    Duration: 1.0 s; \newline
    Sample rate: 44100.0 Hz; \newline
    Window size: 2048; \newline
    Hop size: 441 &
    Detected frequency approximately 440.0 Hz within a tolerance of 10.0 Hz &
    Detected frequency matches expected value within tolerance &
    \textcolor{green}{Pass} \\
  \hline

  Spectrogram Dimensions &
    Constant signal value: 123.0; \newline
    Duration: 0.5 s; \newline
    Sample rate: 44100.0 Hz; \newline
    Window size: 2048; \newline
    Hop size: 441 &
    Spectrogram size matches expected number of frames; \newline
    Each frame has (window size / 2 + 1) frequency bins &
    Spectrogram dimensions match expected values &
    \textcolor{green}{Pass} \\
  \hline

  Constant Signal &
    Constant signal value: 1.0; \newline
    Duration: 0.5 s; \newline
    Sample rate: 44100.0 Hz; \newline
    Window size: 2048; \newline
    Hop size: 441; \newline
    Floating-point tolerance: 1e-6 &
    Magnitude of each spectrogram bin matches the FFT of the windowed constant signal within tolerance &
    Spectrogram magnitudes match expected FFT values within tolerance &
    \textcolor{green}{Pass} \\
  \hline

  Empty Signal &
    Empty signal vector; \newline
    Window size: 2048; \newline
    Hop size: 441 &
    Empty spectrogram &
    Empty spectrogram &
    \textcolor{green}{Pass} \\
  \hline
\end{longtable}

\subsubsection{Window Functions}
\texttt{std::vector<double> generateHammingWindow(int windowSize);}\\
\begin{longtable}{|L{3cm}|L{3cm}|L{2.5cm}|L{2.5cm}|l|}
  \caption{generateHammingWindow Test Results} \\
  \hline
  \textbf{Test} & \textbf{Inputs} & \textbf{Expected Output} & \textbf{Actual Output} & \textbf{Result} \\
  \hline
  Broad Window Check &
    Window size: 10 &
    Window values match expected Hamming values: \{0.08, 0.187619556165, 
    0.460121838273, 0.77, 0.972258605561, 0.972258605561, 
    0.77, 0.460121838273, 0.187619556165, 0.08\};&
    Window values match expected values within tolerance &
    \textcolor{green}{Pass} \\
  \hline
  Window Size &
    Window size: 5 &
    Window size equals 5 &
    Window size equals 5 &
    \textcolor{green}{Pass} \\
  \hline
  Window Taper Values &
    Window size: 10 &
    First and last window values match calculated Hamming values within tolerance &
    First and last window values match expected values within tolerance &
    \textcolor{green}{Pass} \\
  \hline
  Even Window Symmetry &
    Window size: 10 &
    Window is symmetric around its center &
    Window is symmetric around its center &
    \textcolor{green}{Pass} \\
  \hline
  Odd Window Symmetry &
    Window size: 11 &
    Window is symmetric around its center with a distinct middle value &
    Window is symmetric around its center with a distinct middle value &
    \textcolor{green}{Pass} \\
  \hline
  Unique Window &
    Window size: 10 &
    Two generated windows with the same size are identical &
    Two generated windows are identical &
    \textcolor{green}{Pass} \\
  \hline
  Invalid Window Size &
    Window size: 0 &
    Function throws invalid argument exception &
    Segmentation fault &
    \textcolor{red}{Fail} \\
  \hline
\end{longtable}

\noindent\texttt{std::vector<double> generateHanningWindow(int windowSize);}\\
\begin{longtable}{|L{3cm}|L{3cm}|L{2.5cm}|L{2.5cm}|l|}
  \caption{generateHanningWindow Test Results} \\
  \hline
  \textbf{Test} & \textbf{Inputs} & \textbf{Expected Output} & \textbf{Actual Output} & \textbf{Result} \\
  \hline
  Broad Window Check &
    Window size: 10 &
    Window values match expected Hamming values: \{0.0, 0.11697777845, 0.413176, 0.75, 0.9698463, 
    0.9698463, 0.75, 0.413176, 0.11697778, 0.0\}; &
    Window values match expected values within tolerance &
    \textcolor{green}{Pass} \\
  \hline
  Window Size &
    Window size: 5 &
    Window size equals 5 &
    Window size equals 5 &
    \textcolor{green}{Pass} \\
  \hline
  Window Taper Values &
    Window size: 10 &
    First and last window values match calculated Hanning values within tolerance &
    First and last window values match expected values within tolerance &
    \textcolor{green}{Pass} \\
  \hline
  Even Window Symmetry &
    Window size: 10 &
    Window is symmetric around its center &
    Window is symmetric around its center &
    \textcolor{green}{Pass} \\
  \hline
  Odd Window Symmetry &
    Window size: 11 &
    Window is symmetric around its center with a distinct middle value &
    Window is symmetric around its center with a distinct middle value &
    \textcolor{green}{Pass} \\
  \hline
  Unique Window &
    Window size: 10 &
    Two generated windows with the same size are identical &
    Two generated windows are identical &
    \textcolor{green}{Pass} \\
  \hline
  Invalid Window Size &
    Window size: 0 &
    Function throws invalid argument exception &
    Segmentation fault &
    \textcolor{red}{Fail} \\
  \hline
\end{longtable}

\subsection{Audio Feature Extraction Module}
\subsubsection{Key Detection}
\begin{longtable}{|L{3cm}|L{3cm}|L{2.5cm}|L{2.5cm}|l|}
  \caption{Audio Feature Extraction Test Results} \\
  \hline
  \textbf{Test} & \textbf{Inputs} & \textbf{Expected Output} & \textbf{Actual Output} & \textbf{Result} \\
  \hline
  Correlation of Identical Sequences &
    Data: \{1, 2, 3, 4, 5, 6, 7, 8, 9, 10, 11, 12\}; \newline
    Data: \{1, 2, 3, 4, 5, 6, 7, 8, 9, 10, 11, 12\}; &
    Correlation: 1.0 &
    Correlation: 1.0 &
    \textcolor{green}{Pass} \\
  \hline
  Correlation of Opposite Sequences &
    Data: \{1, 2, 3, 4, 5, 6, 7, 8, 9, 10, 11, 12\}; \newline
    Data: \{12, 11, 10, 9, 8, 7, 6, 5, 4, 3, 2, 1\}; &
    Correlation: -1.0 &
    Correlation: -1.0 &
    \textcolor{green}{Pass} \\
  \hline
  Extract C Major Key &
    Durations: \{6, 2, 3, 2, 4, 4, 2, 5, 2, 3, 2, 3\} &
    Key: "C" &
    Key: "C" &
    \textcolor{green}{Pass} \\
  \hline
  Extract C Minor Key &
    Durations: \{6, 2, 3, 5, 2, 3, 2, 4, 3, 2, 3, 3\} &
    Key: "c" &
    Key: "c" &
    \textcolor{green}{Pass} \\
  \hline
  Extract A Major Key &
    Durations: \{1, 1, 1, 1, 1, 1, 1, 1, 1, 1, 1, 10\} &
    Key: "A" &
    Key: "A" &
    \textcolor{green}{Pass} \\
  \hline
  Extract F\# Major Key &
    Durations: \{2, 5, 2, 3, 2, 3, 6, 2, 3, 2, 4, 4\} &
    Key: "F\#" &
    Key: "F\#" &
    \textcolor{green}{Pass} \\
  \hline
  Extract Key of Uniform Durations &
    Durations: \{1, 1, 1, 1, 1, 1, 1, 1, 1, 1, 1, 1\} &
    Key: "C" &
    Key: "C" &
    \textcolor{green}{Pass} \\
  \hline
\end{longtable}

\subsection{Audio Recording and Playback Module}
Given the inherent complexity and hardware dependencies associated with this module, it 
was tested manually by the developers. As manual testing provided a more effective and contextually 
relevant validation process given each developer's unique configuration/environment.

\section{Changes Due to Testing}
\subsection{Unit Testing Results}
Unit testing identified the following issues, which were addressed upon discovery:
\begin{enumerate}
  \item \textbf{Multi-Channel Conversion Limitations:} The system's capability to handle channel conversions was 
  initially limited to two channels. This constraint was insufficient for applications requiring support 
  for more than two channels. Recognizing this limitation, we extended the functionality to accommodate 
  conversions involving more than two channels.
  \item \textbf{Window Function Input Validation:} The initial implementation lacked proper validation for inputs 
  to the Hanning and Hamming window functions. This oversight could lead to incorrect calculations during 
  the short time Fourier transform.
  \item \textbf{.WAV File Read/Write Cycle:} The mismatch between data vectors of the original and output files, and consequent 
  test failure stemmed from differences in data representation: libsndfile uses 16-bit PCM format, while our tests expected 
  double-precision floating-point numbers. The normalization process performed by libsndfile resulted in precision loss beyond
  expected floating-point tolerence values. This issue was resolved by adjusting the test expectations to account for the
  normalization process.
\end{enumerate}

\subsection{Supervisor and Proxy Feedback}
Feedback collected during the Revision 0 demonstration led to the following changes:
\begin{enumerate}
    \item \textbf{Reduction of External Library Dependencies:} To decrease reliance on external libraries that obscure DSP functionality, 
    we reverted some implementations and developed a custom core library. This library now handles DSP functionalities such as Fourier transforms, windowing, 
    peak-picking, spectral flux analysis, and filtering. This change made the project more self-contained and introduced additional opportunities for testing.
    \item \textbf{Enhanced Audio Preprocessing Techniques:} Discussion with the supervisor on topics such as band-pass filtering and related signal processing methods 
    guided the decision to use additional audio preprocessing techniques. Specifically, a triangular filter bank was implemented which improved representations of 
    spectral characteristics of the input signal, thus, enhancing the system's audio feature extraction.
    \item \textbf{Refined Product Scope:} Based on suggestions from other proxies, we narrowed and refined the product's scope to ensure thoroughly processsed
    monophonic audio rather than more complex and therefore difficult forms of audio.
    \item \textbf{Improved Repository Traceability:} Administrative feedback emphasized the importance of granular pull requests and commits to increase 
    traceability in the repository.
\end{enumerate}

\section{Automated Testing}
A GitHub Actions (GHA) workflow triggers the unit tests described in section 5 on every push and pull request to the main branch. 
Through the use of vcpkg for dependency management, CMake for configuration, and ctest for execution the project's unit tests are 
automated. This is vital for the project's CI goals as the unit tests verify the expected behaviour of important components of the 
system including signal processing, MusicXML score generation, and file format conversion. \\
The GHA workflow for automated testing is available in the ScoreGen repository at: 
\href{https://github.com/emilyperica/ScoreGen/blob/main/.github/workflows/run-tests.yml}{\texttt{.github/workflows/run-tests.yml}}.

\section{Trace to Requirements}
See traceability tables between Test Cases and Requirements in the \href{https://github.com/emilyperica/ScoreGen/blob/main/docs/VnVPlan/VnVPlan.pdf}{VnV Plan}.
		
\section{Trace to Modules}		
See traceability tables between Test Cases and Modules in the \href{https://github.com/emilyperica/ScoreGen/blob/main/docs/VnVPlan/VnVPlan.pdf}{VnV Plan}.

\section{Code Coverage Metrics}

\bibliographystyle{plainnat}
\bibliography{../../refs/References}

\newpage{}
\section*{Appendix --- Reflection}

The information in this section will be used to evaluate the team members on the
graduate attribute of Reflection.

\input{../Reflection.tex}

\begin{enumerate}
  \item What went well while writing this deliverable? \\
  \textbf{Mark:} This deliverable was a good way to validate our progress and compare the initial plans with the implemented work.

  \textbf{Emily:} Writing this deliverable provided an opportunity for me to take a closer look at how well we’ve succeeded in 
  following our requirements up to this point. It’s easy to forget about the more granular details, so having to perform testing 
  for this deliverable helped me to move my focus to some of the requirements that we had unintentionally swept under the rug up
   to this point.

  \textbf{Ian:} Partitioning the unit testing section of the report was made easy because of our system design modules. Each 
  module clearly has its relevant unit testing tables which show the results, making the report look clean and tidy.

  \textbf{Jackson:} I think this deliverable was a great step in seeing where we are in terms of the project being complete. 
  It gave us a great idea of what we need to do and work on to finish what we started in the SRS.

  \item What pain points did you experience during this deliverable, and how
    did you resolve them?\\
  \textbf{Mark:} It was slightly tedious to repeat several steps in the VnV which had been done in earlier revisions.

  \textbf{Emily:} It was hard having to shift my focus from finalizing the project to testing and requirements. Especially with 
  the final demo coming up soon, having to go back and formalize our testing process felt tedious.

  \textbf{Ian:} The formatting of the many tables in the report. Some tests had many inputs that did not easily fit in the 
  table cells unless font size significantly varied across the report, or the tables varied in size. Resolved through reference to 
  past document tables, LaTeX research.

  \textbf{Jackson:} I think there was a lot of redundancy from copying over all the tests instead of just modifying the VnV report directly.

  \item Which parts of this document stemmed from speaking to your client(s) or
  a proxy (e.g. your peers)? Which ones were not, and why?

  Most of the unit test results and code review were done internally by us. This is because we are the most experienced people for the job and 
  have a good idea what to look for when going through it all. As well, some NFR and UI-related tests needed an outside opinion from peers. 
  This is required because they act as real-world users who are unaware of technical specifications. Our proxies also influenced section 6 
  because their feedback during our Rev0 demo focused our scope of work and the trajectory of the project.
  

  \item In what ways was the Verification and Validation (VnV) Plan different
  from the activities that were actually conducted for VnV?  If there were
  differences, what changes required the modification in the plan?  Why did
  these changes occur?  Would you be able to anticipate these changes in future
  projects?  If there weren't any differences, how was your team able to clearly
  predict a feasible amount of effort and the right tasks needed to build the
  evidence that demonstrates the required quality?  (It is expected that most
  teams will have had to deviate from their original VnV Plan.)
  
  In between the VnV plan and the VnV report, we had the Revision 0 demo, which 
  changed a lot. Initially, we were optimistic about how much we could get done, 
  including tests for polyphonic audio processing and support for multiple operating 
  systems. After realizing what was feasible and changing the scope of our project, 
  we had to change multiple aspects, including what was previously stated. These 
  changes were necessary to accommodate the change in scope, however, we preserved 
  the tests as possible future directions for the project. With better planning and 
  analysis of what is feasible, I think we could avoid some changes in future projects, 
  however, there will always be changes that are made on the fly, which is all part of 
  the iterative process.

\end{enumerate}

\end{document}