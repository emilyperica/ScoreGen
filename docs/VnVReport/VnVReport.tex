\documentclass[12pt, titlepage]{article}

\usepackage{booktabs}
\usepackage{tabularx}
\usepackage{hyperref}
\hypersetup{
    colorlinks,
    citecolor=black,
    filecolor=black,
    linkcolor=red,
    urlcolor=blue
}
\usepackage[round]{natbib}
\usepackage{graphicx}
\usepackage[T1]{fontenc}
\usepackage{ragged2e}
\usepackage{float}
\usepackage{longtable}
\usepackage{makecell}
\usepackage{seqsplit}

\newcolumntype{L}[1]{>{\RaggedRight\arraybackslash}p{#1}}
\newcolumntype{C}{>{\centering\arraybackslash}X}
\newcolumntype{L}[1]{>{\raggedright\arraybackslash}p{#1}}
\AtBeginEnvironment{longtable}{\small}

\input{../Comments}
\input{../Common}

\begin{document}

\title{Verification and Validation Report: \progname} 
\author{\authname}
\date{\today}
	
\maketitle

\pagenumbering{roman}

\section{Revision History}

\begin{tabularx}{\textwidth}{p{3cm}p{2cm}X}
\toprule {\bf Date} & {\bf Version} & {\bf Notes}\\
\midrule
2025-03-10 & 1.0 & Initial version.\\
\bottomrule
\end{tabularx}

~\newpage

\section{Symbols, Abbreviations and Acronyms}

\renewcommand{\arraystretch}{1.2}
\begin{tabular}{l l} 
  \toprule		
  \textbf{symbol} & \textbf{description}\\
  \midrule 
  T & Test\\
  \bottomrule
\end{tabular}\\

\wss{symbols, abbreviations or acronyms -- you can reference the SRS tables if needed}

\newpage

\tableofcontents

\listoftables %if appropriate

\listoffigures %if appropriate

\newpage

\pagenumbering{arabic}

This document ...

\section{Functional Requirements Evaluation}

\section{Nonfunctional Requirements Evaluation}

\subsection{Usability}
		
\subsection{Performance}

\subsection{etc.}

\section{Unit Testing}

This section describes the low-level tests that were conducted to verify that the behaviour of 
functions within the system's modules are correct. The tests were created using the GTest framework.

\subsection{Helper Functions}

\texttt{std::vector<double> generateSineWave(double frequency, double sampleRate, double duration);}
\begin{longtable}{|L{3cm}|L{2cm}|L{2cm}|L{2cm}|l|}
  \hline
  \textbf{Test} & \textbf{Inputs} & \textbf{Expected Output} & \textbf{Actual Output} & \textbf{Result} \\
  \hline
  Signal Size & 440.0, 44100.0, 1.0 & Vector size of 44100 & Vector size of 44100 & \textcolor{green}{Pass} \\
  \hline
  Amplitude Consistency & 440.0, 44100.0, 1.0 & All samples between -1 and 1.0 & Vector size of 44100 & \textcolor{green}{Pass} \\
  \hline
  Phase Continuity & 440.0, 44100.0, 1.0 & All samples approximately 0 within tolerance & All samples approximately 0 within tolerance & \textcolor{green}{Pass} \\
  \hline
  Zero Duration & 440.0, 44100.0, 0.0 & Empty vector & Vector size of 44100 & \textcolor{green}{Pass} \\
  \hline
  High Frequency & 22050.0, 44100.0, 1.0 & All samples approximately 0 within tolerance & All samples approximately 0 within tolerance & \textcolor{green}{Pass} \\
  \hline
\end{longtable}


\noindent\texttt{float extractFundamentalFrequency(const std::vector<std::vector<double>>\& spectrogram, double sampleRate);}

\begin{longtable}{|L{3cm}|L{2cm}|L{2cm}|L{2cm}|l|}
  \hline
  \textbf{Test} & \textbf{Inputs} & \textbf{Expected Output} & \textbf{Actual Output} & \textbf{Result} \\
  \hline
  Single Frame Peak & \{\{0.0, ..., 1.0, ..., 0.0\}\}, 44100.0 & 100.0 & 100.0 & \textcolor{green}{Pass} \\
  \hline
  Multi-frame Competing Peaks & \{\{0.0, ..., 1.0, ..., 0.0\}, \{0.0, ..., 2.0, ..., 0.0\}\}, 44100.0 & 75.0 & 75.0 & \textcolor{green}{Pass} \\
  \hline
  Empty Spectrogram & \{\}, 44100.0 & 0.0 & 0.0 & \textcolor{green}{Pass} \\
  \hline
  Different Sample Rate & \{\{0.0, ..., 1.0, ..., 0.0\}\}, 16000.0 & 100.0 & 100.0 & \textcolor{green}{Pass} \\
  \hline
\end{longtable}

\subsection{UI Module}
Manual testing by the developers (e.g. visual inspection, etc.) was deemed more efficient and effective for this module. Consequently, unit testing tables for the UI 
module are not included in this report.\\

\subsection{Score Generation Module}
\subsubsection{MusicXML Generation}
\begin{longtable}{|L{3cm}|L{2cm}|L{2cm}|L{2cm}|l|}
  \hline
  \textbf{Test} & \textbf{Inputs} & \textbf{Expected Output} & \textbf{Actual Output} & \textbf{Result} \\
  \hline
  Note Element Creation & 
      Regular: \{"C", 0, 4, 4, "quarter", false\}
      Rest: \{"", 0, 0, 4, "quarter", true\}
      Accidental: \{"C", 1, 4, 4, "quarter", false\} &
    Valid XML note elements (non-null) & Non-null pointers & \textcolor{green}{Pass} \\
  \hline
  Measure Creation & 
      Single note: \{"C", 0, 4, 4, "quarter", false\} 
      Multi-note: \{"C", 0, 4, 4, "quarter", false\}, \{"D", 0, 4, 4, "quarter", false\},
        \{"E", 0, 4, 4, "quarter", false\}, \{"F", 0, 4, 4, "quarter", false\} 
        Mixed: \{"C", 0, 4, 4, "quarter", false\}, \{"", 0, 0, 4, "quarter", true\},
        \{"D", 0, 4, 4, "quarter", false\}, \{"", 0, 0, 8, "half", true\}
    &
    Valid XML measure elements (non-null) & Non-null pointers & \textcolor{green}{Pass} \\
  \hline
  Part Creation & 
    \{"C", 0, 4, 4, "quarter", false; "D", 0, 4, 4, "quarter", false; 
    "E", 0, 4, 4, "quarter", false; "F", 0, 4, 4, "quarter", false; 
    "G", 1, 4, 4, "quarter", false; "", 0, 0, 8, "half", true; 
    "A", 0, 4, 4, "quarter", false; "B", -1, 4, 4, "quarter", false\} &
    Valid XML part element (non-null) & Non-null pointer & \textcolor{green}{Pass} \\
  \hline
  Score Part Creation & 
    N/A & 
    Valid XML score part element (non-null) & Non-null pointer & \textcolor{green}{Pass} \\
  \hline
  All Note Types & 
    -- & 
    Generation of all note types (e.g., dotted notes, rests) & -- & \textcolor{yellow}{Pending} \\
  \hline
  Full Generation & 
      Multiple measures with standard notes, accidentals, rests,
      missing note type, and extreme octaves
    &
      Success code; File generated and non-empty
    &
    True; file non-empty & \textcolor{green}{Pass} \\
  \hline
  Empty Sequence Generation & 
    Empty note sequence & 
    Failure code; No file generated & False; No file & \textcolor{green}{Pass} \\
  \hline
\end{longtable}

\subsection{File Format Conversion Module}
Some submodules of this module were tested manually by the developers visually (e.g. mXML to SVG/PDF). \\ \\
The following tests use a common SF\_INFO configuration (1 channel, 44100 Hz, WAV/PCM\_16) and generate a 440 Hz sine wave of 1-second duration for file I/O operations.

\begin{longtable}{|L{3cm}|L{3cm}|L{2.5cm}|L{2.5cm}|l|}
  \hline
  \textbf{Test} & \textbf{Inputs} & \textbf{Expected Output} & \textbf{Actual Output} & \textbf{Result} \\
  \hline
  Read Valid File & 
    Create directory "test-data"; \newline
    SF\_INFO; \newline
    Write sine wave (440 Hz, 1 s) to "test-data/sine.wav" & 
    SF\_INFO fields match; \newline
    Data size = 44100; \newline
    Sine samples (at indices 0, 100, 1234) within floating-point tolerance & 
    SF\_INFO and data verified; \newline
    Data size = 44100; \newline
    Sine sample checks pass & \textcolor{green}{Pass} \\
  \hline
  Read Non-Existent File & 
    File: "fakefile.wav" & 
    Data vector size = 0 & 
    Data vector size = 0 & \textcolor{green}{Pass} \\
  \hline
  Read Write Read Cycle & 
    Create directory "test-data"; \newline
    Write sine wave (440 Hz, 1 s) with SF\_INFO to "test-data/sine.wav"; \newline
    Read file, then write output to "test-data/output.wav"; \newline
    Read output file & 
    SF\_INFO and data of original and output files are identical; \newline
    Data vectors equal element-wise within floating-point tolerance & 
    SF\_INFO and data match between original and output; \newline
    Verification within floating-point tolerance succeeds & \textcolor{green}{Pass} \\
  \hline
  Write Invalid Location & 
    SF\_INFO; \newline
    Data vector: 1000 samples (each 0.5); \newline
    Invalid directory: "/DNE/" & 
    No fatal failure during write operation & 
    No fatal failure encountered & \textcolor{green}{Pass} \\
  \hline
\end{longtable}

\subsection{Raw Signal Processing Module}
\subsubsection{Pre-processing}
\begin{longtable}{|L{3cm}|L{3cm}|L{2.5cm}|L{2.5cm}|l|}
  \hline
  \textbf{Test} & \textbf{Inputs} & \textbf{Expected Output} & \textbf{Actual Output} & \textbf{Result} \\
  \hline
  Single Channel Test &
    Data: \{0.5, 1.5, -2.0, 3.0\}; \newline
    Channels: 1 &
    Processed data matches input data &
    Processed data matches input data &
    \textcolor{green}{Pass} \\
  \hline
  Two Channel to Mono Conversion &
    Data: \{1.0, 2.0, 1.1, 2.1, 1.2, 2.2\}; \newline
    Channels: 2 &
    Averaged data: \{1.5, 1.6, 1.7\} &
    Averaged data: \{1.5, 1.6, 1.7\} &
    \textcolor{green}{Pass} \\
  \hline
  Three Channel to Mono Conversion &
    Data: \{1.0, 2.0, 3.0, 1.1, 2.1, 3.1\}; \newline
    Channels: 3 &
    Averaged data: \{2.0, 2.1\} &
    Fatal failure &
    \textcolor{red}{Fail} \\
  \hline
  Empty Data &
    Data: \{\}; \newline
    Channels: 2 &
    Processed data is empty &
    Processed data is empty &
    \textcolor{green}{Pass} \\
  \hline
\end{longtable}

\subsubsection{Fourier Transform and Spectrogram Creation}
\begin{longtable}{|L{3cm}|L{3cm}|L{2.5cm}|L{2.5cm}|l|}
  \hline
  \textbf{Test} & \textbf{Inputs} & \textbf{Expected Output} & \textbf{Actual Output} & \textbf{Result} \\
  \hline
  Extract Sine Wave Frequency &
    Frequency: 440.0 Hz (A4); \newline
    Duration: 1.0 s; \newline
    Sample rate: 44100.0 Hz; \newline
    Window size: 2048; \newline
    Hop size: 441 &
    Detected frequency approximately 440.0 Hz within a tolerance of 10.0 Hz &
    Detected frequency matches expected value within tolerance &
    \textcolor{green}{Pass} \\
  \hline

  Spectrogram Dimensions &
    Constant signal value: 123.0; \newline
    Duration: 0.5 s; \newline
    Sample rate: 44100.0 Hz; \newline
    Window size: 2048; \newline
    Hop size: 441 &
    Spectrogram size matches expected number of frames; \newline
    Each frame has (window size / 2 + 1) frequency bins &
    Spectrogram dimensions match expected values &
    \textcolor{green}{Pass} \\
  \hline

  Constant Signal &
    Constant signal value: 1.0; \newline
    Duration: 0.5 s; \newline
    Sample rate: 44100.0 Hz; \newline
    Window size: 2048; \newline
    Hop size: 441; \newline
    Floating-point tolerance: 1e-6 &
    Magnitude of each spectrogram bin matches the FFT of the windowed constant signal within tolerance &
    Spectrogram magnitudes match expected FFT values within tolerance &
    \textcolor{green}{Pass} \\
  \hline

  Empty Signal &
    Empty signal vector; \newline
    Window size: 2048; \newline
    Hop size: 441 &
    Empty spectrogram &
    Empty spectrogram &
    \textcolor{green}{Pass} \\
  \hline
\end{longtable}

\subsubsection{Window Functions}
\texttt{std::vector<double> generateHammingWindow(int windowSize);}\\
\begin{longtable}{|L{3cm}|L{3cm}|L{2.5cm}|L{2.5cm}|l|}
  \hline
  \textbf{Test} & \textbf{Inputs} & \textbf{Expected Output} & \textbf{Actual Output} & \textbf{Result} \\
  \hline
  Broad Window Check &
    Window size: 10 &
    Window values match expected Hamming values: \{0.08, 0.187619556165, 
    0.460121838273, 0.77, 0.972258605561, 0.972258605561, 
    0.77, 0.460121838273, 0.187619556165, 0.08\};&
    Window values match expected values within tolerance &
    \textcolor{green}{Pass} \\
  \hline
  Window Size &
    Window size: 5 &
    Window size equals 5 &
    Window size equals 5 &
    \textcolor{green}{Pass} \\
  \hline
  Window Taper Values &
    Window size: 10 &
    First and last window values match calculated Hamming values within tolerance &
    First and last window values match expected values within tolerance &
    \textcolor{green}{Pass} \\
  \hline
  Even Window Symmetry &
    Window size: 10 &
    Window is symmetric around its center &
    Window is symmetric around its center &
    \textcolor{green}{Pass} \\
  \hline
  Odd Window Symmetry &
    Window size: 11 &
    Window is symmetric around its center with a distinct middle value &
    Window is symmetric around its center with a distinct middle value &
    \textcolor{green}{Pass} \\
  \hline
  Unique Window &
    Window size: 10 &
    Two generated windows with the same size are identical &
    Two generated windows are identical &
    \textcolor{green}{Pass} \\
  \hline
  Invalid Window Size &
    Window size: 0 &
    Function throws invalid argument exception &
    Segmentation fault &
    \textcolor{red}{Fail} \\
  \hline
\end{longtable}

\noindent\texttt{std::vector<double> generateHanningWindow(int windowSize);}\\
\begin{longtable}{|L{3cm}|L{3cm}|L{2.5cm}|L{2.5cm}|l|}
  \hline
  \textbf{Test} & \textbf{Inputs} & \textbf{Expected Output} & \textbf{Actual Output} & \textbf{Result} \\
  \hline
  Broad Window Check &
    Window size: 10 &
    Window values match expected Hamming values: \{0.0, 0.11697777845, 0.413176, 0.75, 0.9698463, 
    0.9698463, 0.75, 0.413176, 0.11697778, 0.0\}; &
    Window values match expected values within tolerance &
    \textcolor{green}{Pass} \\
  \hline
  Window Size &
    Window size: 5 &
    Window size equals 5 &
    Window size equals 5 &
    \textcolor{green}{Pass} \\
  \hline
  Window Taper Values &
    Window size: 10 &
    First and last window values match calculated Hanning values within tolerance &
    First and last window values match expected values within tolerance &
    \textcolor{green}{Pass} \\
  \hline
  Even Window Symmetry &
    Window size: 10 &
    Window is symmetric around its center &
    Window is symmetric around its center &
    \textcolor{green}{Pass} \\
  \hline
  Odd Window Symmetry &
    Window size: 11 &
    Window is symmetric around its center with a distinct middle value &
    Window is symmetric around its center with a distinct middle value &
    \textcolor{green}{Pass} \\
  \hline
  Unique Window &
    Window size: 10 &
    Two generated windows with the same size are identical &
    Two generated windows are identical &
    \textcolor{green}{Pass} \\
  \hline
  Invalid Window Size &
    Window size: 0 &
    Function throws invalid argument exception &
    Segmentation fault &
    \textcolor{red}{Fail} \\
  \hline
\end{longtable}

\subsubsection{Onset Detection}


\subsection{Audio Feature Extraction Module}
\subsubsection{Key Detection}
\begin{longtable}{|L{3cm}|L{3cm}|L{2.5cm}|L{2.5cm}|l|}
  \hline
  \textbf{Test} & \textbf{Inputs} & \textbf{Expected Output} & \textbf{Actual Output} & \textbf{Result} \\
  \hline
  Correlation of Identical Sequences &
    Data: \{1, 2, 3, 4, 5, 6, 7, 8, 9, 10, 11, 12\}; \newline
    Data: \{1, 2, 3, 4, 5, 6, 7, 8, 9, 10, 11, 12\}; &
    Correlation: 1.0 &
    Correlation: 1.0 &
    \textcolor{green}{Pass} \\
  \hline
  Correlation of Opposite Sequences &
    Data: \{1, 2, 3, 4, 5, 6, 7, 8, 9, 10, 11, 12\}; \newline
    Data: \{12, 11, 10, 9, 8, 7, 6, 5, 4, 3, 2, 1\}; &
    Correlation: -1.0 &
    Correlation: -1.0 &
    \textcolor{green}{Pass} \\
  \hline
  Extract C Major Key &
    Durations: \{6, 2, 3, 2, 4, 4, 2, 5, 2, 3, 2, 3\} &
    Key: "C" &
    Key: "C" &
    \textcolor{green}{Pass} \\
  \hline
  Extract C Minor Key &
    Durations: \{6, 2, 3, 5, 2, 3, 2, 4, 3, 2, 3, 3\} &
    Key: "c" &
    Key: "c" &
    \textcolor{green}{Pass} \\
  \hline
  Extract A Major Key &
    Durations: \{1, 1, 1, 1, 1, 1, 1, 1, 1, 1, 1, 10\} &
    Key: "A" &
    Key: "A" &
    \textcolor{green}{Pass} \\
  \hline
  Extract F\# Major Key &
    Durations: \{2, 5, 2, 3, 2, 3, 6, 2, 3, 2, 4, 4\} &
    Key: "F\#" &
    Key: "F\#" &
    \textcolor{green}{Pass} \\
  \hline
  Extract Key of Uniform Durations &
    Durations: \{1, 1, 1, 1, 1, 1, 1, 1, 1, 1, 1, 1\} &
    Key: "C" &
    Key: "C" &
    \textcolor{green}{Pass} \\
  \hline
\end{longtable}

\subsubsection{Pitch Detection}
\textcolor{yellow}{Active WIP...} \\

\subsection{Audio Recording and Playback Module}
Given the inherent complexity and hardware dependencies associated with this module, it 
was tested manually by the developers. As manual testing provided a more effective and contextually 
relevant validation process given each developer's unique configuration/environment.

\section{Changes Due to Testing}
FAILURES:
- mention hanning and hamming window functions input Validation
- mention channel conversion for >2 channels
\wss{This section should highlight how feedback from the users and from 
the supervisor (when one exists) shaped the final product.  In particular 
the feedback from the Rev 0 demo to the supervisor (or to potential users) 
should be highlighted.}

\section{Automated Testing}
- two sections:\\
1. Unit Testing GHA workflow in CI \\
2. CI workflow(s) for system testing
		
\section{Trace to Requirements}
		
\section{Trace to Modules}		

\section{Code Coverage Metrics}

\bibliographystyle{plainnat}
\bibliography{../../refs/References}

\newpage{}
\section*{Appendix --- Reflection}

The information in this section will be used to evaluate the team members on the
graduate attribute of Reflection.

\input{../Reflection.tex}

\begin{enumerate}
  \item What went well while writing this deliverable? 
  \item What pain points did you experience during this deliverable, and how
    did you resolve them?
  \item Which parts of this document stemmed from speaking to your client(s) or
  a proxy (e.g. your peers)? Which ones were not, and why?
  \item In what ways was the Verification and Validation (VnV) Plan different
  from the activities that were actually conducted for VnV?  If there were
  differences, what changes required the modification in the plan?  Why did
  these changes occur?  Would you be able to anticipate these changes in future
  projects?  If there weren't any differences, how was your team able to clearly
  predict a feasible amount of effort and the right tasks needed to build the
  evidence that demonstrates the required quality?  (It is expected that most
  teams will have had to deviate from their original VnV Plan.)
\end{enumerate}

\end{document}