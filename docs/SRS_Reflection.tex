\documentclass{article}
\usepackage{enumitem}
\begin{document}

\begin{enumerate}
  \item \textbf{What went well while writing this deliverable?} \\
  \textbf{Mark}: This deliverable was a good opportunity to coordinate our efforts across a much larger body of work compared to the first deliverable. By maintaining continuous feedback and communication, our team avoided contradictions within the documentation. Sections such as “stretch goals”, “future plans”, and “waiting room” were helpful in having the group think forward and define what was definite versus what might happen if things went well. \\
  \textbf{Ian}: I think we communicated and split the work well during this deliverable. In some ways, this deliverable acted as the official start of the capstone project because it was much larger than the previous ones. It also required us to narrow down our vision for the project, which excited us all. \\
  \textbf{Emily}: This deliverable helped us gain a better understanding of the different aspects of the project we’ll be tackling. It can be difficult to grasp a project as large as this from the start, but we worked well as a team to communicate and prioritize the features we’d like to see in the final product. \\
  \textbf{Jackson}: I thought this was great because the deliverable cemented how we will work as a team. We communicated very well on each section, referencing each other’s work and asking for clarification when needed. I also enjoyed our open discussions about what we would implement in the future, as thinking ahead is always good for projects like this. \\

  \item \textbf{What pain points did you experience during this deliverable, and how did you resolve them?} \\
  \textbf{Mark}: A pain point was trying to minimize repetition in similar sections of the requirements document. It often felt as though the same thing was being said multiple times in different sections, making it hard to maintain a cohesive flow. \\
  \textbf{Ian}: One pain point was applying the Volere template (specifically business data and adjacent business analysis) to a project that is more or less standalone. The template is geared towards complex projects involving additions to existing systems, and it was hard to draw parallels when using it. \\
  \textbf{Emily}: My biggest pain point was distinguishing between functional and non-functional requirements. I’ve struggled with this since I first started writing requirements. I resolved it by doing research and finding examples to clarify the differences. \\
  \textbf{Jackson}: I had trouble rationalizing non-functional vs functional requirements, especially with performance requirements, as they often sounded like functional ones. I resolved this by learning more about their differences and understanding how non-functional requirements deal with system performance, while functional ones focus on how the system operates. \\

  \item \textbf{How many of your requirements were inspired by speaking to your clients or their proxies (e.g. peers, stakeholders, potential users)?} \\
  Most requirements came from personal experience and understanding of necessary features for an acceptable product. However, requirements related to user proficiency in music theory were formed through conversations with potential clients with varying levels of music knowledge. This helped determine what could be assumed as common knowledge versus what would require additional documentation. \\

  \item \textbf{Which of the courses you have taken, or are currently taking, will help your team be successful with your capstone project?} \\ \\
  \textbf{Mark}: The most valuable courses for technical knowledge will be 3MX3 and 3DX4, both focused on signal processing, a key aspect of the project. Experiential courses like 3XB3 and 2AA4 provided a strong foundation for building Git-based projects and collaborating within a team. \\
  \textbf{Ian}: 3MX3 is the course most related to this project, as it provides valuable insights into signal processing, the most important part of our project. Courses like 4AA4, which deals with real-time tasks, and 4HC3, focused on stakeholder analysis and interface design, are also relevant. \\
  \textbf{Emily}: SFWRENG 3A04, Large System Design, has the largest similarities to this project. The course involved a term-long project focused on requirements elicitation, use cases, and stakeholder definitions, which aligns with our current work. \\
  \textbf{Jackson}: Signal processing will be a large part of this project, and 3MX3 and 3DX4, which directly deal with this, are crucial. Additionally, courses with assignment-based projects like 2AA4 and 3BB4 gave me confidence when starting larger projects like this. \\

  \item \textbf{What knowledge and skills will the team collectively need to acquire to successfully complete this capstone project?} \\
  The team will need to acquire both technical and non-technical skills. Project management will be vital, requiring leadership, knowledge of agile methodologies, and delegation abilities. Music theory and composition knowledge will be necessary for ensuring accurate transcriptions. UI/UX design will require graphic design knowledge and familiarity with prototyping tools like Figma. Software testing and quality analysis skills will be exercised through our Kanban board and CI pipeline. Signal processing knowledge will be crucial for handling audio input. Each team member will focus on one of these areas to ensure the project’s success. \\

  \item \textbf{For each of the knowledge areas and skills identified, what are at least two approaches to acquiring the knowledge or mastering the skill? Which approaches will each team member pursue, and why?} \\
  Two approaches for acquiring knowledge include self-directed learning with online resources and practicing these skills through personal or open-source projects. Given the team’s full-time student status, both approaches are reasonable and cater well to self-paced learning. The team will pursue both approaches collaboratively, sharing newly acquired skills with each other as needed.
\end{enumerate}

\end{document}
