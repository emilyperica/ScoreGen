\documentclass[12pt, titlepage]{article}

\usepackage{booktabs}
\usepackage{tabularx}
\usepackage{hyperref}
\hypersetup{
    colorlinks,
    citecolor=blue,
    filecolor=black,
    linkcolor=red,
    urlcolor=blue
}
\usepackage[round]{natbib}

\input{../Comments}
\input{../Common}

\begin{document}

\title{System Verification and Validation Plan for \progname{}} 
\author{\authname}
\date{\today}
	
\maketitle

\pagenumbering{roman}

\section*{Revision History}

\begin{tabularx}{\textwidth}{p{3cm}p{2cm}X}
\toprule {\bf Date} & {\bf Version} & {\bf Notes}\\
\midrule
01/11/2024 & 0 & Initial draft\\
\bottomrule
\end{tabularx}

~\\

\newpage

\tableofcontents

\listoftables

\listoffigures
\wss{Remove this section if it isn't needed}

\newpage

\section{Symbols, Abbreviations, and Acronyms}
See the \href{https://github.com/emilyperica/ScoreGen/blob/main/docs/SRS-Volere/SRS.pdf}{SRS} document for additional definitions. \\

\renewcommand{\arraystretch}{1.2}
\begin{tabular}{l l} 
  \toprule		
  \textbf{Symbol} & \textbf{Description}\\
  \midrule 
  MG & Module Guide Specification\\
  MIS & Module Interface Specification\\ 
  SRS & Software Requirement Specification\\
  T & Test\\
  VnV & Verification and Validation\\
  \bottomrule
\end{tabular}\\

\newpage

\pagenumbering{arabic}

This document outlines the Verification and Validation (VnV) plan for the development 
of an audio-to-sheet music generator. This includes plans to verify the SRS, system 
design, the VnV itself, implementation, and software, as well as a plan for an automated 
testing suite. This document will also describe the specific tests that will be used to 
ensure all Functional and Non-functional Requirements are met.

\section{General Information}

\subsection{Summary}

This Verification and Validation Plan describes the testing process for ScoreGen, an 
audio-to-sheet music generator, such that the following core functionalities perform as desired:
\begin{itemize}
  \item Signal processing of audio input
  \item Identification of pitch, rhythm, and timing
  \item Generation of readable and accurate sheet music
\end{itemize}

\subsection{Objectives}

\subsubsection{Desired System Qualities}

The primary objective of this document is to ensure that the audio-to-sheet music system produces 
highly accurate, reliable, and usable output, successfully meeting the needs of all stakeholders. 
Key qualities this VnV will aim to accomplish include the following:
\begin{itemize}
  \item Demonstrate accuracy in music notation
  \item Ease of use amongst users with varying characteristics
  \item Readable, easy to understand outputs
  \item An enjoyable user experience
  \item Efficiency and Responsiveness
  \item Portability of the software
\end{itemize}

\subsubsection{Out of Scope Objectives}
In order to meet time and cost requirements, the following objectives will be out of the scope of 
this VnV plan:
\begin{itemize}
  \item Advanced polyphonic and multi-instrumental audio:
  
  Monophonic audio is easily processed using signal analysis, but introducing a multitude of 
  simultaneous signals will require advanced algorithms to separate each note.
  
  \item Cross-platform support:
  
  Majority of the team members will be writing the system’s software on their personal Windows PC. 
  Testing should thus occur in the same environment it is being developed in in order to reduce 
  performance variability of the final system.
  
  \item Advanced music notation, such as dynamics and accents:
  
  Dynamics and accents can be tricky to differentiate, and attempting to identify them is likely to 
  introduce a level of error to the output that is undesired. As well, a beginner musician may feel 
  overwhelmed being faced with a large variety of notation that they don’t recognize.
  
  \item Support for non-Western music notation:
  
  This software will exclusively produce sheet music according to Western music conventions. Expanding 
  to alternative notation conventions would require significant adapting of the output format, and is 
  infeasible given the current scope of the project. 
  
\end{itemize}

\subsection{Challenge Level and Extras}

\subsubsection{Challenge Level}

The ScoreGen project is categorized in the general challenge level, meaning it is not particularly novel 
and requires a senior highschool level or junior undergraduate level of domain knowledge/implementation. 
Projects in this category are required to include ‘extras’ in order to meet the difficulty level expected 
of a final-year capstone course.

\subsubsection{Project Extras}
\begin{itemize}
  \item User manual
  \item Usability testing
  \item GenderMag personas
\end{itemize}

\subsection{Relevant Documentation}

This document is just one of many written to provide a comprehensive overview of the project and its desired 
outcomes. The other project documents include:
\begin{itemize}
  \item \href{https://github.com/emilyperica/ScoreGen/blob/main/docs/DevelopmentPlan/DevelopmentPlan.pdf}{Development Plan} \citep*{ScoreGenDP}: 
  This document acts as a blueprint for the software’s creation as well as the overall structuring of the project itself. Leveraging the information 
  outlined in this document will ensure all verification and validation activities are relevant and within the scope of the project.
  \item \href{https://github.com/emilyperica/ScoreGen/blob/main/docs/ProblemStatementAndGoals/ProblemStatement.pdf}{Problem Statement and Goals} \citep*{ScoreGenPS}:
  This document contains an explicit definition of the problem the ScoreGen system aims to solve. It provides the VnV team with a clear understanding of how 
  tests may shape the project to align with the system’s intended impact.
  \item \href{https://github.com/emilyperica/ScoreGen/blob/main/docs/SRS-Volere/SRS.pdf}{Software Requirement Specification (SRS)} \citep*{ScoreGenSRS}:
  States the functional and non-functional requirements that this VnV plan is developing tests for.
  \item \href{https://github.com/emilyperica/ScoreGen/blob/main/docs/HazardAnalysis/HazardAnalysis.pdf}{Hazard Analysis} \citep*{ScoreGenHA}:
  Similar to the SRS, this document will guide VnV test plans to ensure all possible hazards to the system are mitigated.
  \item \href{https://github.com/emilyperica/ScoreGen/blob/main/docs/VnVReport/VnVReport.pdf}{Verification and Validation Report} \citep*{ScoreGenVnVReport}:
  This document will be guided entirely by the frameworks described in this VnV plan. Its outcome will depend on the relevance and quality of the described plans.
  \item \href{https://github.com/emilyperica/ScoreGen/blob/main/docs/UserGuide/UserGuide.pdf}{User Guide} \citep*{ScoreGenUG}:
  This guide is intended to be used directly by the end-user to inform them on usage of the software. The VnV aims to 
  verify all use cases and user interactions, and can thus be used to identify the necessary features or steps whose inclusion 
  in the User Guide is critical.
  \item \href{https://github.com/emilyperica/ScoreGen/blob/main/docs/Design/SoftDetailedDes/MIS.pdf}{Module Interface Specification (MIS)} \citep*{ScoreGenMIS}:
  This document defines how different software modules within the system interact with one another, making note of all inputs and outputs. These inputs 
  and outputs will require verification outlined in the VnV Plan to ensure feasibility and accurate constraints are put into place.
  \item \href{https://github.com/emilyperica/ScoreGen/blob/main/docs/Design/SoftArchitecture/MG.pdf}{Module Guide (MG)} \citep*{ScoreGenMG}:
  Where the MIS defines module interactions, this document defines the modules themselves. The VnV tests should target all modules in 
  the system - since the MG will be written after the VnV Plan, it may be necessary to revisit this plan and add/remove tests where necessary 
  as the modules are formalized.
\end{itemize}


\section{Plan}

This section will provide formal validation and verification plans for various aspects of the project. 
Starting with a definition of all individuals involved in the VnV process, it goes on to inform the 
reader of VnV plans for the SRS, design, VnV plan itself, implementation, automated tests, and the 
system software.

\subsection{Verification and Validation Team}

The first 4 entries in the below table are the software developers creating the system; they will all 
contribute to VnV activities to ensure the work is evenly distributed, but each has a specific responsibility 
they will oversee. The last 3 entries in the table are not involved in ideation/implementation of VnV, but 
their feedback will shape the final plan. \newpage


\begin{table}[h!]
  \centering
  \caption{VnV Team}
  \begin{tabular}{|p{0.3\textwidth}|p{0.6\textwidth}|}
  \hline
    \textbf{Name} & \textbf{Role} \\
  \hline
    Emily Perica & \textbf{Validation analyst}. Generates test reports and ensures the test results can be 
    linked back to the \href{https://github.com/emilyperica/ScoreGen/blob/main/docs/SRS-Volere/SRS.pdf}{SRS}.\\
  \hline
    Ian Algenio & \textbf{FR tester}. Leads the testing process for functional requirements. \\
  \hline
    Jackson Lippert & \textbf{NFR tester}. Leads the testing process for non-functional requirements. \\
  \hline
    Mark Kogan & \textbf{Verification specialist}. Defines the process for establishing ground truths and 
    ensures all tests being implemented follow the VnV plans. \\
  \hline
    Dr. Martin von \newline Mohrenschildt & \textbf{Project supervisor}. Provides insight and suggests 
    strategies for creating relevant tests. \\
  \hline
    Dr. Spencer Smith & \textbf{Course supervisor}. Similar to the project supervisor; provides insight and 
    guides the team in creating a robust VnV plan.\\
  \hline
    Hunter Ceranic & \textbf{Teaching assistant}. Will provide feedback to the team on the feasibility, scope, 
    and overall quality of the VnV plan.\\
  \hline
  \end{tabular}
\end{table}

\subsection{SRS Verification Plan}
\label{sec:srs_verification}

To verify the SRS for our sheet music generation project, we plan to use a combination of informal feedback and structured reviews to ensure accuracy and alignment with the project's goals. The approaches are as follows:

\subsection*{Ad Hoc Feedback from Peers}
We will request informal reviews from our primary reviewers (team 8). Their feedback will help identify potential ambiguities or areas needing clarification. We will ask them to focus on the clarity and feasibility of requirements, ensuring alignment with the intended functionality of the app.

\subsection*{Supervisor Review Meeting}

\textbf{Structured Review Meeting}: We plan to conduct a concise, structured meeting with our project supervisor to leverage their expertise effectively while respecting their time constraints. This meeting will include the following steps:

\begin{enumerate}
    \item \textbf{Overview Presentation}: We will present a high-level overview of our initial requirements, focusing on the key features and functionalities we believe are essential for the project. The purpose of this presentation is to quickly bring the supervisor up to speed on the project's direction.
    
    \item \textbf{Discussion on Feasibility and Priority}: We will ask the supervisor to evaluate the feasibility of the proposed requirements, help identify any that are overly ambitious or unrealistic, and highlight those that are core to the project’s success. This will help us prioritize our work based on expert advice.
    
    \item \textbf{Focused Questions}: To guide the discussion and make the best use of the supervisor’s time, we will prepare targeted questions that focus on critical project areas. Examples include:
    \begin{itemize}
        \item Which requirements are technically challenging or might present significant roadblocks?
        \item Are there any high-priority requirements that we might have overlooked?
        \item Which elements align closely with the project’s objectives, and which might be deferred?
    \end{itemize}
    
    \item \textbf{Utilizing Issue Tracker}: To maintain transparency and track action items, we will record the supervisor’s feedback and any new insights or tasks identified during the meeting in our issue tracker. This will enable us to follow up efficiently and ensure that the supervisor's input is incorporated into subsequent iterations of our requirements document.
\end{enumerate}


\subsection*{Issue Tracking System}
We will use the git issue tracker to log any concerns, suggestions, or changes requested by reviewers. This system will help maintain a clear record of all feedback and revisions made to the SRS, ensuring transparency and accountability in addressing reviewer comments.

This combination of peer and supervisor reviews, structured meetings, and issue tracking should provide comprehensive coverage for SRS verification.

\subsection{Design Verification Plan}

To ensure the design of the sheet music generation app meets all functional and non-functional requirements, the following verification plan will be implemented:

\subsubsection*{Peer Review and Feedback}
The design will undergo reviews by our classmates, focusing on key design elements such as the user interface and algorithm selection for audio-to-sheet music conversion. Feedback from peers will help identify usability issues, potential improvements, and ensure that the design aligns with the project goals.

\subsubsection*{Checklist-Based Review}
A checklist will be created to guide reviewers in assessing each design aspect. The checklist will include items such as:
\begin{itemize}
    \item Are the core algorithms well-suited for the required audio-to-sheet music conversion?
    \item Are system constraints (e.g., processing time, memory usage) addressed in the design?
    \item Is the design modular and maintainable?
\end{itemize}
This checklist will help standardize feedback, ensuring a thorough and consistent review process.

\subsubsection*{Supervisor Review}
In addition to peer review, we will present the design to our project supervisor for further validation. The supervisor will be provided with the design documents and checklist before the review session to facilitate a structured examination of each design component. Key discussion points will include scalability, performance optimization, and potential risks. 

\subsubsection*{Issue Tracking and Follow-Up}
Git Issues will be used to log any design issues or recommendations identified during the review process. This will provide a centralized record of all feedback and allow for systematic tracking of design revisions, ensuring that each suggestion is addressed or appropriately documented.

This design verification plan, incorporating both peer and supervisor feedback along with checklist-guided reviews and structured follow-up, will ensure the design is robust, user-friendly, and aligned with project requirements.


\subsection{Verification and Validation Plan Verification Plan}

\textbf{Verification and Validation Plan Checklist}:

\begin{itemize}
    \item \textbf{Plan Review by Peers}: Ensure the verification and validation plan is reviewed by classmates to gain diverse insights and identify potential improvements.
    \item \textbf{Requirement Coverage}: Verify that all requirements are adequately covered by the verification and validation plan, ensuring no critical areas are omitted.
    \item \textbf{Traceability}: Confirm traceability of each verification and validation activity to specific project requirements to maintain alignment with project goals.
    \item \textbf{Clarity and Completeness}: Review the plan for clarity, making sure all instructions and testing procedures are clear and complete for future reference.
\end{itemize}


\subsection{Implementation Verification Plan}


The Implementation Verification Plan outlines the strategies and methodologies we will employ to verify the correctness and quality of our implementation.

\begin{itemize}
    \item \textbf{Unit Testing Plan}: A comprehensive unit testing plan will be developed, which includes a detailed list of tests designed to verify the functionality of individual components of the system. Each unit test will target specific functions or classes to ensure they perform as expected under various conditions. The results of these tests will be crucial for identifying and resolving issues early in the development process.

    \item \textbf{Static Verification Methods}: In addition to dynamic testing, we will implement several static verification techniques to enhance the quality of our code. These methods allow us to analyze the code without executing it, thereby identifying potential issues at an early stage. The techniques we plan to use include:
    \begin{itemize}
        \item \textbf{Code Walkthroughs}: We will conduct peer reviews in the form of code walkthroughs, where team members will systematically review code segments. This collaborative approach encourages knowledge sharing and helps catch defects or design flaws that may not be evident to the original developer.
        \item \textbf{Static Analyzers}: We will employ static analysis tools to automatically analyze the code for common programming errors, potential security vulnerabilities, and adherence to coding best practices. These tools provide insights that can help us improve code quality and reduce technical debt.
    \end{itemize}

    By integrating both unit testing and static verification methods into our implementation verification plan, we aim to ensure that our code is robust, maintainable, and free from critical defects. This dual approach not only enhances the reliability of our implementation but also streamlines the development process by identifying issues early on, reducing the risk of more significant problems later in the project lifecycle.
\end{itemize}

\subsection{Automated Testing and Verification Tools}

For our project, we plan to utilize GitHub Actions to create a continuous integration pipeline, which will automate the build and testing processes. We will employ the Google Test framework for unit testing, as it provides a robust and easy-to-use interface for writing and executing C++ test cases.

In addition to unit testing, we will implement linters to verify coding standards and maintain consistency across the codebase. The coding standards we will follow include:

\begin{itemize}
    \item \href{https://google.github.io/styleguide/cppguide.html}{Google's C++ Style Guide} \citep*{GoogleCppStyleGuide} for C++.
    \item \href{https://peps.python.org/pep-0008/}{PEP8 Style Guide} \citep*{PEP8PythonStyleGuide}: for Python.
    \item \href{https://getbem.com/}{BEM Methodology} \citep*{BEMMethodology}: for HTML and CSS.
\end{itemize}

We will summarize code coverage metrics by integrating a coverage tool like gcov with our CI pipeline, allowing us to visualize the effectiveness of our tests and identify untested code paths.

\subsection{Software Validation Plan}

For our project, we plan to validate the functionality of our application using imported music files. These music files will serve as test cases to ensure that our software correctly processes and generates sheet music from audio inputs.

To further validate the product we will conduct user testing to gather feedback on the functionality and usability of our application, allowing us to make improvements based on user experiences.

There will also be a quality assurance testing effort to ensure a robust adherence to the defined SRS requirements, as further defined in the \hyperref[sec:srs_verification]{SRS Verification Plan}

\section{System Tests}

\wss{There should be text between all headings, even if it is just a roadmap of
the contents of the subsections.}

\subsection{Tests for Functional Requirements}

\wss{Subsets of the tests may be in related, so this section is divided into
  different areas.  If there are no identifiable subsets for the tests, this
  level of document structure can be removed.}

\wss{Include a blurb here to explain why the subsections below
  cover the requirements.  References to the SRS would be good here.}

\subsubsection{Area of Testing1}

\wss{It would be nice to have a blurb here to explain why the subsections below
  cover the requirements.  References to the SRS would be good here.  If a section
  covers tests for input constraints, you should reference the data constraints
  table in the SRS.}
		
\paragraph{Title for Test}

\begin{enumerate}

\item{test-id1\\}

Control: Manual versus Automatic
					
Initial State: 
					
Input: 
					
Output: \wss{The expected result for the given inputs.  Output is not how you
are going to return the results of the test.  The output is the expected
result.}

Test Case Derivation: \wss{Justify the expected value given in the Output field}
					
How test will be performed: 
					
\item{test-id2\\}

Control: Manual versus Automatic
					
Initial State: 
					
Input: 
					
Output: \wss{The expected result for the given inputs}

Test Case Derivation: \wss{Justify the expected value given in the Output field}

How test will be performed: 

\end{enumerate}

\subsubsection{Area of Testing2}

...

\subsection{Tests for Nonfunctional Requirements}

\wss{The nonfunctional requirements for accuracy will likely just reference the
  appropriate functional tests from above.  The test cases should mention
  reporting the relative error for these tests.  Not all projects will
  necessarily have nonfunctional requirements related to accuracy.}

\wss{For some nonfunctional tests, you won't be setting a target threshold for
passing the test, but rather describing the experiment you will do to measure
the quality for different inputs.  For instance, you could measure speed versus
the problem size.  The output of the test isn't pass/fail, but rather a summary
table or graph.}

\wss{Tests related to usability could include conducting a usability test and
  survey.  The survey will be in the Appendix.}

\wss{Static tests, review, inspections, and walkthroughs, will not follow the
format for the tests given below.}

\wss{If you introduce static tests in your plan, you need to provide details.
How will they be done?  In cases like code (or document) walkthroughs, who will
be involved? Be specific.}

\subsubsection{Area of Testing1}
		
\paragraph{Title for Test}

\begin{enumerate}

\item{test-id1\\}

Type: Functional, Dynamic, Manual, Static etc.
					
Initial State: 
					
Input/Condition: 
					
Output/Result: 
					
How test will be performed: 
					
\item{test-id2\\}

Type: Functional, Dynamic, Manual, Static etc.
					
Initial State: 
					
Input: 
					
Output: 
					
How test will be performed: 

\end{enumerate}

\subsubsection{Area of Testing2}

...

\subsection{Traceability Between Test Cases and Requirements}

\wss{Provide a table that shows which test cases are supporting which
  requirements.}

\section{Unit Test Description}

\wss{This section should not be filled in until after the MIS (detailed design
  document) has been completed.}

\wss{Reference your MIS (detailed design document) and explain your overall
philosophy for test case selection.}  

\wss{To save space and time, it may be an option to provide less detail in this section.  
For the unit tests you can potentially layout your testing strategy here.  That is, you 
can explain how tests will be selected for each module.  For instance, your test building 
approach could be test cases for each access program, including one test for normal behaviour 
and as many tests as needed for edge cases.  Rather than create the details of the input 
and output here, you could point to the unit testing code.  For this to work, you code 
needs to be well-documented, with meaningful names for all of the tests.}

\subsection{Unit Testing Scope}

\wss{What modules are outside of the scope.  If there are modules that are
  developed by someone else, then you would say here if you aren't planning on
  verifying them.  There may also be modules that are part of your software, but
  have a lower priority for verification than others.  If this is the case,
  explain your rationale for the ranking of module importance.}

\subsection{Tests for Functional Requirements}

\wss{Most of the verification will be through automated unit testing.  If
  appropriate specific modules can be verified by a non-testing based
  technique.  That can also be documented in this section.}

\subsubsection{Module 1}

\wss{Include a blurb here to explain why the subsections below cover the module.
  References to the MIS would be good.  You will want tests from a black box
  perspective and from a white box perspective.  Explain to the reader how the
  tests were selected.}

\begin{enumerate}

\item{test-id1\\}

Type: \wss{Functional, Dynamic, Manual, Automatic, Static etc. Most will
  be automatic}
					
Initial State: 
					
Input: 
					
Output: \wss{The expected result for the given inputs}

Test Case Derivation: \wss{Justify the expected value given in the Output field}

How test will be performed: 
					
\item{test-id2\\}

Type: \wss{Functional, Dynamic, Manual, Automatic, Static etc. Most will
  be automatic}
					
Initial State: 
					
Input: 
					
Output: \wss{The expected result for the given inputs}

Test Case Derivation: \wss{Justify the expected value given in the Output field}

How test will be performed: 

\item{...\\}
    
\end{enumerate}

\subsubsection{Module 2}

...

\subsection{Tests for Nonfunctional Requirements}

\wss{If there is a module that needs to be independently assessed for
  performance, those test cases can go here.  In some projects, planning for
  nonfunctional tests of units will not be that relevant.}

\wss{These tests may involve collecting performance data from previously
  mentioned functional tests.}

\subsubsection{Module ?}
		
\begin{enumerate}

\item{test-id1\\}

Type: \wss{Functional, Dynamic, Manual, Automatic, Static etc. Most will
  be automatic}
					
Initial State: 
					
Input/Condition: 
					
Output/Result: 
					
How test will be performed: 
					
\item{test-id2\\}

Type: Functional, Dynamic, Manual, Static etc.
					
Initial State: 
					
Input: 
					
Output: 
					
How test will be performed: 

\end{enumerate}

\subsubsection{Module ?}

...

\subsection{Traceability Between Test Cases and Modules}

\wss{Provide evidence that all of the modules have been considered.}
				
\bibliographystyle{plainnat}

\bibliography{../../refs/References}

\newpage

\section{Appendix}

This is where you can place additional information.

\subsection{Symbolic Parameters}

The definition of the test cases will call for SYMBOLIC\_CONSTANTS.
Their values are defined in this section for easy maintenance.

\subsection{Usability Survey Questions?}

\wss{This is a section that would be appropriate for some projects.}

\newpage{}
\section*{Appendix --- Reflection}

The information in this section will be used to evaluate the team members on the
graduate attribute of Lifelong Learning.

\input{../Reflection.tex}

\begin{enumerate}
  \item What went well while writing this deliverable? 
  \textbf{Mark} This deliverable went very well. I feel that our group was able to work seperately on different sections of the document, yet when the changes were merged together, the results fit well together. This suggests we operate together smoothly without need of excessive feedback \\ \\
  \textbf{Emily} \\ \\
  \textbf{Jackson} \\ \\
  \textbf{Ian} \\ \\
  \item What pain points did you experience during this deliverable, and how
    did you resolve them?
    \textbf{Mark} There were no particular pain points in this deliverable, though there was significant volume of writing out each section, and there had to be significant recall of previous documents to reference them where necessary.\\ \\
    \textbf{Emily} \\ \\
    \textbf{Jackson} \\ \\
    \textbf{Ian} \\ \\
  \item What knowledge and skills will the team collectively need to acquire to
  successfully complete the verification and validation of your project?
  Examples of possible knowledge and skills include dynamic testing knowledge,
  static testing knowledge, specific tool usage, Valgrind etc.  You should look to
  identify at least one item for each team member.

  To properly complete the verification and validation of our project, our team will need to acquire both technical and procedural knowledge to ensure thorough and effective testing. Each team member will contribute by developing specific testing skills tailored to different aspects of the project:

  \begin{itemize}
      \item \textbf{Proficiency in Testing Frameworks:} All team members will need to become familiar with testing frameworks such as Google Test to conduct unit tests, integration tests, and automated testing. This knowledge will allow us to design and execute a comprehensive test suite covering various functionalities of the project.
  
      \item \textbf{YAML Syntax and Configuration Management:} Since some of our test configurations and project settings are defined in YAML, each team member must be comfortable with YAML syntax. This knowledge will enable team members to adjust and manage configurations accurately, helping to avoid syntax-related issues and ensuring consistency across testing environments.
    
      \item \textbf{Static Testing and Code Review Skills:} Each team member will develop static testing capabilities, focusing on code walkthroughs, inspections, and manual code reviews. Regular code review sessions will ensure adherence to code standards, and feedback sessions will help all members understand key areas of the codebase and identify potential issues before they manifest at runtime.
  
      \item \textbf{Documentation and Reporting:} One team member will specialize in documentation and reporting, ensuring that all test cases, configurations, and results are well-documented and stored. This role will help track progress and identify recurring issues, facilitating clear communication within the team and making it easier to maintain consistency as the project evolves.
  \end{itemize}
  
  With these knowledge areas, the team will be well-prepared to handle the testing requirements for verification and validation, ensuring that the project is robust, optimized, and well-documented at every stage.
  \item For each of the knowledge areas and skills identified in the previous
  question, what are at least two approaches to acquiring the knowledge or
  mastering the skill?  Of the identified approaches, which will each team
  member pursue, and why did they make this choice?
\end{enumerate}

\end{document}