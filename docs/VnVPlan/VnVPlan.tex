\documentclass[12pt, titlepage]{article}

\usepackage{booktabs}
\usepackage{tabularx}
\usepackage{longtable}
\usepackage{multirow}
\usepackage{array}
\usepackage{hyperref}
\hypersetup{
    colorlinks,
    citecolor=blue,
    filecolor=black,
    linkcolor=red,
    urlcolor=blue
}
\usepackage[round]{natbib}

\input{../Comments}
\input{../Common}

\begin{document}

\title{System Verification and Validation Plan for \progname{}} 
\author{\authname}
\date{\today}
	
\maketitle

\pagenumbering{roman}

\section*{Revision History}

\begin{tabularx}{\textwidth}{p{3cm}p{2cm}X}
\toprule {\bf Date} & {\bf Version} & {\bf Notes}\\
\midrule
01/11/2024 & 0 & Initial draft\\
\bottomrule
\end{tabularx}

~\\

\newpage

\tableofcontents

\listoftables

\listoffigures
\wss{Remove this section if it isn't needed}

\newpage

\section{Symbols, Abbreviations, and Acronyms}
See the \href{https://github.com/emilyperica/ScoreGen/blob/main/docs/SRS-Volere/SRS.pdf}{SRS} document for additional definitions. \\

\renewcommand{\arraystretch}{1.2}
\begin{tabular}{l l} 
  \toprule		
  \textbf{Symbol} & \textbf{Description}\\
  \midrule 
  MG & Module Guide Specification\\
  MIS & Module Interface Specification\\ 
  SRS & Software Requirement Specification\\
  T & Test\\
  VnV & Verification and Validation\\
  \bottomrule
\end{tabular}\\

\newpage

\pagenumbering{arabic}

This document outlines the Verification and Validation (VnV) plan for the development 
of an audio-to-sheet music generator. This includes plans to verify the SRS, system 
design, the VnV itself, implementation, and software, as well as a plan for an automated 
testing suite. This document will also describe the specific tests that will be used to 
ensure all Functional and Non-functional Requirements are met.

\section{General Information}

\subsection{Summary}

This Verification and Validation Plan describes the testing process for ScoreGen, an 
audio-to-sheet music generator, such that the following core functionalities perform as desired:
\begin{itemize}
  \item Signal processing of audio input
  \item Identification of pitch, rhythm, and timing
  \item Generation of readable and accurate sheet music
\end{itemize}

\subsection{Objectives}

\subsubsection{Desired System Qualities}

The primary objective of this document is to ensure that the audio-to-sheet music system produces 
highly accurate, reliable, and usable output, successfully meeting the needs of all stakeholders. 
Key qualities this VnV will aim to accomplish include the following:
\begin{itemize}
  \item Demonstrate accuracy in music notation
  \item Ease of use amongst users with varying characteristics
  \item Readable, easy to understand outputs
  \item An enjoyable user experience
  \item Efficiency and Responsiveness
  \item Portability of the software
\end{itemize}

\subsubsection{Out of Scope Objectives}
In order to meet time and cost requirements, the following objectives will be out of the scope of 
this VnV plan:
\begin{itemize}
  \item Advanced polyphonic and multi-instrumental audio:
  
  Monophonic audio is easily processed using signal analysis, but introducing a multitude of 
  simultaneous signals will require advanced algorithms to separate each note.
  
  \item Cross-platform support:
  
  Majority of the team members will be writing the system’s software on their personal Windows PC. 
  Testing should thus occur in the same environment it is being developed in in order to reduce 
  performance variability of the final system.
  
  \item Advanced music notation, such as dynamics and accents:
  
  Dynamics and accents can be tricky to differentiate, and attempting to identify them is likely to 
  introduce a level of error to the output that is undesired. As well, a beginner musician may feel 
  overwhelmed being faced with a large variety of notation that they don’t recognize.
  
  \item Support for non-Western music notation:
  
  This software will exclusively produce sheet music according to Western music conventions. Expanding 
  to alternative notation conventions would require significant adapting of the output format, and is 
  infeasible given the current scope of the project. 
  
\end{itemize}

\subsection{Challenge Level and Extras}

\subsubsection{Challenge Level}

The ScoreGen project is categorized in the general challenge level, meaning it is not particularly novel 
and requires a senior highschool level or junior undergraduate level of domain knowledge/implementation. 
Projects in this category are required to include ‘extras’ in order to meet the difficulty level expected 
of a final-year capstone course.

\subsubsection{Project Extras}
\begin{itemize}
  \item User manual
  \item Usability testing
  \item GenderMag personas
\end{itemize}

\subsection{Relevant Documentation}

This document is just one of many written to provide a comprehensive overview of the project and its desired 
outcomes. The other project documents include:
\begin{itemize}
  \item \href{https://github.com/emilyperica/ScoreGen/blob/main/docs/DevelopmentPlan/DevelopmentPlan.pdf}{Development Plan} \citep*{ScoreGenDP}: 
  This document acts as a blueprint for the software’s creation as well as the overall structuring of the project itself. Leveraging the information 
  outlined in this document will ensure all verification and validation activities are relevant and within the scope of the project.
  \item \href{https://github.com/emilyperica/ScoreGen/blob/main/docs/ProblemStatementAndGoals/ProblemStatement.pdf}{Problem Statement and Goals} \citep*{ScoreGenPS}:
  This document contains an explicit definition of the problem the ScoreGen system aims to solve. It provides the VnV team with a clear understanding of how 
  tests may shape the project to align with the system’s intended impact.
  \item \href{https://github.com/emilyperica/ScoreGen/blob/main/docs/SRS-Volere/SRS.pdf}{Software Requirement Specification (SRS)} \citep*{ScoreGenSRS}:
  States the functional and non-functional requirements that this VnV plan is developing tests for.
  \item \href{https://github.com/emilyperica/ScoreGen/blob/main/docs/HazardAnalysis/HazardAnalysis.pdf}{Hazard Analysis} \citep*{ScoreGenHA}:
  Similar to the SRS, this document will guide VnV test plans to ensure all possible hazards to the system are mitigated.
  \item \href{https://github.com/emilyperica/ScoreGen/blob/main/docs/VnVReport/VnVReport.pdf}{Verification and Validation Report} \citep*{ScoreGenVnVReport}:
  This document will be guided entirely by the frameworks described in this VnV plan. Its outcome will depend on the relevance and quality of the described plans.
  \item \href{https://github.com/emilyperica/ScoreGen/blob/main/docs/UserGuide/UserGuide.pdf}{User Guide} \citep*{ScoreGenUG}:
  This guide is intended to be used directly by the end-user to inform them on usage of the software. The VnV aims to 
  verify all use cases and user interactions, and can thus be used to identify the necessary features or steps whose inclusion 
  in the User Guide is critical.
  \item \href{https://github.com/emilyperica/ScoreGen/blob/main/docs/Design/SoftDetailedDes/MIS.pdf}{Module Interface Specification (MIS)} \citep*{ScoreGenMIS}:
  This document defines how different software modules within the system interact with one another, making note of all inputs and outputs. These inputs 
  and outputs will require verification outlined in the VnV Plan to ensure feasibility and accurate constraints are put into place.
  \item \href{https://github.com/emilyperica/ScoreGen/blob/main/docs/Design/SoftArchitecture/MG.pdf}{Module Guide (MG)} \citep*{ScoreGenMG}:
  Where the MIS defines module interactions, this document defines the modules themselves. The VnV tests should target all modules in 
  the system - since the MG will be written after the VnV Plan, it may be necessary to revisit this plan and add/remove tests where necessary 
  as the modules are formalized.
\end{itemize}


\section{Plan}

This section will provide formal validation and verification plans for various aspects of the project. 
Starting with a definition of all individuals involved in the VnV process, it goes on to inform the 
reader of VnV plans for the SRS, design, VnV plan itself, implementation, automated tests, and the 
system software.

\subsection{Verification and Validation Team}

The first 4 entries in the below table are the software developers creating the system; they will all 
contribute to VnV activities to ensure the work is evenly distributed, but each has a specific responsibility 
they will oversee. The last 3 entries in the table are not involved in ideation/implementation of VnV, but 
their feedback will shape the final plan.


\begin{table}[h!]
  \centering
  \caption{VnV Team}
  \begin{tabular}{|p{0.3\textwidth}|p{0.6\textwidth}|}
  \hline
    \textbf{Name} & \textbf{Role} \\
  \hline
    Emily Perica & \textbf{Validation analyst}. Generates test reports and ensures the test results can be 
    linked back to the \href{https://github.com/emilyperica/ScoreGen/blob/main/docs/SRS-Volere/SRS.pdf}{SRS}.\\
  \hline
    Ian Algenio & \textbf{FR tester}. Leads the testing process for functional requirements. \\
  \hline
    Jackson Lippert & \textbf{NFR tester}. Leads the testing process for non-functional requirements. \\
  \hline
    Mark Kogan & \textbf{Verification specialist}. Defines the process for establishing ground truths and 
    ensures all tests being implemented follow the VnV plans. \\
  \hline
    Dr. Martin von \newline Mohrenschildt & \textbf{Project supervisor}. Provides insight and suggests 
    strategies for creating relevant tests. \\
  \hline
    Dr. Spencer Smith & \textbf{Course supervisor}. Similar to the project supervisor; provides insight and 
    guides the team in creating a robut VnV plan.\\
  \hline
    Hunter Ceranic & \textbf{Teaching assistant}. Will provide feedback to the team on the feasibility, scope, 
    and overall quality of the VnV plan.\\
  \hline
  \end{tabular}
\end{table}

\subsection{SRS Verification Plan}

\wss{List any approaches you intend to use for SRS verification.  This may
  include ad hoc feedback from reviewers, like your classmates (like your
  primary reviewer), or you may plan for something more rigorous/systematic.}

\wss{If you have a supervisor for the project, you shouldn't just say they will
read over the SRS.  You should explain your structured approach to the review.
Will you have a meeting?  What will you present?  What questions will you ask?
Will you give them instructions for a task-based inspection?  Will you use your
issue tracker?}

\wss{Maybe create an SRS checklist?}

\subsection{Design Verification Plan}

\wss{Plans for design verification}

\wss{The review will include reviews by your classmates}

\wss{Create a checklists?}

\subsection{Verification and Validation Plan Verification Plan}

\wss{The verification and validation plan is an artifact that should also be
verified.  Techniques for this include review and mutation testing.}

\wss{The review will include reviews by your classmates}

\wss{Create a checklists?}

\subsection{Implementation Verification Plan}

\wss{You should at least point to the tests listed in this document and the unit
  testing plan.}

\wss{In this section you would also give any details of any plans for static
  verification of the implementation.  Potential techniques include code
  walkthroughs, code inspection, static analyzers, etc.}

\wss{The final class presentation in CAS 741 could be used as a code
walkthrough.  There is also a possibility of using the final presentation (in
CAS741) for a partial usability survey.}

\subsection{Automated Testing and Verification Tools}

\wss{What tools are you using for automated testing.  Likely a unit testing
  framework and maybe a profiling tool, like ValGrind.  Other possible tools
  include a static analyzer, make, continuous integration tools, test coverage
  tools, etc.  Explain your plans for summarizing code coverage metrics.
  Linters are another important class of tools.  For the programming language
  you select, you should look at the available linters.  There may also be tools
  that verify that coding standards have been respected, like flake9 for
  Python.}

\wss{If you have already done this in the development plan, you can point to
that document.}

\wss{The details of this section will likely evolve as you get closer to the
  implementation.}

\subsection{Software Validation Plan}

\wss{If there is any external data that can be used for validation, you should
  point to it here.  If there are no plans for validation, you should state that
  here.}

\wss{You might want to use review sessions with the stakeholder to check that
the requirements document captures the right requirements.  Maybe task based
inspection?}

\wss{For those capstone teams with an external supervisor, the Rev 0 demo should 
be used as an opportunity to validate the requirements.  You should plan on 
demonstrating your project to your supervisor shortly after the scheduled Rev 0 demo.  
The feedback from your supervisor will be very useful for improving your project.}

\wss{For teams without an external supervisor, user testing can serve the same purpose 
as a Rev 0 demo for the supervisor.}

\wss{This section might reference back to the SRS verification section.}

\section{System Tests}

System tests are categorized into two subsections. Section 4.1 lists test cases for functional requirements defined in 
the SRS. Section 4.2 lists test cases for nonfunctional requirements defined in the SRS. 

\subsection{Tests for Functional Requirements}
Tests for the functional requirements of the application naturally follow the division of the major types of functional requirements
in the SRS (see SRS section 9). Thus, tests for each major type are grouped similarly, resulting in five subcategories.

  \subsubsection{Input Handling}
  \begin{enumerate}
    \item \textbf{Test for Correct Audio File Formats} \\
      \newline
      \textbf{Test ID:} FR-AR1-3-1 \\
      \textbf{Control:} Automatic \\
      \textbf{Initial State:} Application is running and idle, awaiting audio file upload from the user. \\
      \textbf{Input:} Audio file (e.g., \texttt{.WAV}, \texttt{.MP3}) \\
      \textbf{Output:} File acceptance without errors, entrance into file processing state. \\
      \textbf{Test Case Derivation:} For validation of functional requirements FR-AR1 and FR-AR3 in section 9.1 of the system requirement specification (SRS).\\
      \textbf{How Test Will Be Performed:}
      \begin{enumerate}
          \item Use prepared audio sample file.
          \item Upload the audio file for transcription.
          \item Confirm the system accepts the file through success message or by entrance into the file processing state.
      \end{enumerate}
  
  \item \textbf{Test for Incorrect File Formats} \\
    \newline
    \textbf{Test ID:} FR-AR1-3-2 \\
    \textbf{Control:} Automatic \\
    \textbf{Initial State:} Application is open and idle, awaiting audio file upload from the user. \\
    \textbf{Input:} Non-audio file (e.g., \texttt{.PDF}, \texttt{.MP4}, \texttt{.JPEG}, etc.) \\
    \textbf{Output:} Denial of upload attempt with error message. \\
    \textbf{Test Case Derivation:} For validation of functional requirements FR-AR1 and FR-AR3 in section 9.1 of the system requirement specification (SRS). \\
    \textbf{How Test Will Be Performed:}
    \begin{enumerate}
        \item Use prepared files of unsupported formats.
        \item Upload the file for transcription.
        \item Confirm the system denies the file and provides an error message that specifies the unsupported file format.
    \end{enumerate}
  
  \item \textbf{Test User Device Microphone} \\
  \newline
  \textbf{Test ID:} FR-AR2 \\
  \textbf{Control:} Manual \\
  \textbf{Initial State:} Application is open and idle, user has navigated to the audio recording interface and microphone permissions have been granted. \\
  \textbf{Input:} Audio recorded by the user device’s microphone. \\
  \textbf{Output:} The application captures the correct audio and saves it in a format processable by the application (e.g., \texttt{.WAV}). \\
  \textbf{Test Case Derivation:} For validation of functional requirement FR-AR2 in section 9.1 of the system requirement specification (SRS). \\
  \textbf{How Test Will Be Performed:}
  \begin{enumerate}
      \item Navigate to the application’s audio recording interface.
      \item Start a recording using the application’s “Record” element.
      \item Wait for 10 seconds and stop the recording.
      \item Confirm that the audio was recorded through:
      \begin{itemize}
          \item Playback using an audio player on the device.
          \item File metadata analysis (duration, file size, etc.).
          \item Waveform inspection using an external tool (e.g., Audacity).
      \end{itemize}
      \item Confirm that the captured audio matches the expected input via playback, metadata, and/or waveform inspection.
  \end{enumerate}
  
  \item \textbf{Test Generated Score Alignment with Selected Instrument} \\
    \newline
    \textbf{Test ID:} FR-AR4 \\
    \textbf{Control:} Automatic \\
    \textbf{Initial State:} Application is running and idle, pre-transcription state awaiting input. \\
    \textbf{Input:} Sample audio file or user-recorded audio and selected instrument type. \\
    \textbf{Output:} A generated sheet music file in MusicXML format that aligns with the selected instrument’s key signature and note pitches. \\
    \textbf{Test Case Derivation:} For validation of functional requirement FR-AR4 in section 9.1 of the system requirement specification (SRS). \\
    \textbf{How Test Will Be Performed:}
    \begin{enumerate}
        \item Select and submit an instrument type within the application.
        \item Upload an audio input for the selected instrument.
        \item Upon transcription completion, parse the generated score.
        \item Verify that the key signature and note pitches match the expected result for the selected instrument.
    \end{enumerate}
  \end{enumerate}

  \subsubsection{Signal Processing and Element Identification}
  \begin{enumerate}
    \item \textbf{Test Effect of Increased Noise in Audio Input} \\
      \newline
      \textbf{Test ID:} FR-SP1 \\
      \textbf{Control:} Automatic \\
      \textbf{Initial State:} Application is running and idle, awaiting file upload from the user. \\
      \textbf{Input:} Two sample audio files—one unedited, the other with ~10\% noise interference. \\
      \textbf{Output:} Two identical sheet music files in MusicXML format. \\
      \textbf{Test Case Derivation:} For validation of functional requirement FR-SP1 in section 9.2 of the system requirement specification (SRS). \\
      \textbf{How Test Will Be Performed:}
      \begin{enumerate}
          \item Upload the unedited audio file to the application and generate a sheet music file.
          \item Upload the noisy audio file and generate another sheet music file.
          \item Parse both files and identify discrepancies, if any.
      \end{enumerate}
    
    \item \textbf{Test for Pitch and Rhythm Identification} \\
      \newline
      \textbf{Test ID:} FR-SP2 \\
      \textbf{Control:} Manual \\
      \textbf{Initial State:} Application is running and idle, awaiting audio input from the user. \\
      \textbf{Input:} Sample audio file or user-recorded audio with a known sheet music equivalent. \\
      \textbf{Output:} Sheet music correctly identifying all notes in the input audio. \\
      \textbf{Test Case Derivation:} For validation of functional requirement FR-SP2 in section 9.2 of the system requirement specification (SRS). \\
      \textbf{How Test Will Be Performed:}
      \begin{enumerate}
          \item Prepare an audio file containing an ascending and descending C major scale.
          \item Upload the audio input to the application.
          \item Generate and save the sheet music in a viewable file format.
          \item Compare the generated sheet music visually with the sample sheet music for note discrepancies.
      \end{enumerate}

    \item \textbf{Test for Key Signature and Time Signature Identification} \\
      \newline
      \textbf{Test ID:} FR-SP3 \\
      \textbf{Control:} Automatic \\
      \textbf{Initial State:} Application is running and idle, awaiting audio input from the user. \\
      \textbf{Input:} Sample or user-recorded audio file that has a known key signature and time signature. \\
      \textbf{Output:} Sheet music that has the same key signature and time signature as the input’s sheet music. \\
      \textbf{Test Case Derivation:} For validation of functional requirement FR-SP3 in section 9.2 of the system requirement specification (SRS). \\
      \textbf{How Test Will Be Performed:}
      \begin{enumerate}
          \item Upload the sample file or recorded audio file to the application.
          \item Save the generated sheet music in MusicXML file format.
          \item Parse the generated file and extract the key signature and time signature.
          \item Compare the extracted signatures to the known input’s signatures.
      \end{enumerate}

    \item \textbf{Test for Polyphonic Audio Identification} \\
      \newline
      \textbf{Test ID:} FR-SP5 \\
      \textbf{Control:} Automatic \\
      \textbf{Initial State:} Application is running and idle, awaiting audio input from the user. \\
      \textbf{Input:} Sample file or user-recorded audio that contains known chords. \\
      \textbf{Output:} Sheet music file in MusicXML format that matches the input file’s chords. \\
      \textbf{Test Case Derivation:} For validation of functional requirement FR-SP5 in section 9.2 of the system requirement specification (SRS). \\
      \textbf{How Test Will Be Performed:}
      \begin{enumerate}
          \item Upload the sample audio file to the application.
          \item Ensure the sample audio file includes at least two distinct chords; single notes may or may not be interleaved.
          \item Save the generated sheet music in MusicXML file format.
          \item Parse the generated file.
          \item Compare the processed and identified chords to the known input audio chords and notes.
      \end{enumerate}
    \item \textbf{Test for Monophonic Audio Identification} \\
    Subsumed by test case FR-SP2 (see section 4.1.2.2)
  \end{enumerate}

  \subsubsection{Sheet Music Generation}
  \begin{enumerate}
  \item \textbf{Test General Notation and Layout} \\
    \newline
    \textbf{Test ID:} FR-SMG1 \\
    \textbf{Control:} Automatic with manual inspection \\
    \textbf{Initial State:} Application is running and idle, awaiting audio input from the user. \\
    \textbf{Input:} Sample or user-recorded audio file with equivalent sheet music available. \\
    \textbf{Output:} Sheet music using the same layout and notation as the input. \\
    \textbf{Test Case Derivation:} For validation of functional requirement FR-SMG1 in section 9.3 of the system requirement specification (SRS). \\
    \textbf{How Test Will Be Performed:}
    \begin{enumerate}
        \item Upload the sample or user-recorded audio file for transcription.
        \item Save the generated sheet music in MusicXML format and a viewable document format.
        \item Check for discrepancies:
        \begin{itemize}
            \item Parse the MusicXML file and compare it to the input file.
            \item View the document formatted file (e.g., PDF) and compare it to the input’s sheet music.
        \end{itemize}
    \end{enumerate}
  
  \item \textbf{Test Instrument Specific Concert Pitch} \\
    \newline
    \textbf{Test ID:} FR-SMG2 \\
    \textbf{Control:} Manual \\
    \textbf{Initial State:} Application is running and idle, awaiting audio input from the user. \\
    \textbf{Input:} Two audio files—one from a non-transposing instrument and another from a transposing instrument. \\
    \textbf{Output:} Two sets of sheet music that structurally and visually match their corresponding instrument’s concert pitch. \\
    \textbf{Test Case Derivation:} For validation of functional requirement FR-SMG2 in section 9.3 of the system requirement specification (SRS). \\
    \textbf{How Test Will Be Performed:}
    \begin{enumerate}
        \item Upload the non-transposing instrument’s audio file for transcription and save the generated sheet music in a viewable file format.
        \item Upload the transposing instrument’s audio file for transcription and save the generated sheet music in a viewable file format.
        \item Compare the two sets of sheet music to ensure appropriate notes are in concert pitch relative to the input sheet music.
    \end{enumerate}
  
  \item \textbf{Test Post-Processing Edit and View Functionalities} \\
    \newline
    \textbf{Test ID:} FR-SMG3 \\
    \textbf{Control:} Manual \\
    \textbf{Initial State:} Application has just finished processing and transcribing audio input. \\
    \textbf{Input:} Edit requests, save requests, open requests. \\
    \textbf{Output:} Viewable file containing sheet music that reflects the requested edits. \\
    \textbf{Test Case Derivation:} For validation of functional requirement FR-SMG3 in section 9.3 of the system requirement specification (SRS). \\
    \textbf{How Test Will Be Performed:}
    \begin{enumerate}
        \item View the generated sheet music in the application.
        \item Perform three edit operations:
        \begin{itemize}
            \item Note deletion.
            \item Note addition.
            \item Note pitch and/or duration change.
        \end{itemize}
        \item Save the edited sheet music in a viewable file format.
        \item Open and view the edited sheet music file to confirm that changes are present and correct.
    \end{enumerate}
  \end{enumerate}
  \subsubsection{User Interface (UI)}
  \begin{enumerate}
    \item \textbf{Test Application Feedback After Audio File is Uploaded} \\
      \newline
      \textbf{Test ID:} FR-UI1 \\
      \textbf{Control:} Automatic \\
      \textbf{Initial State:} Application is running and idle, awaiting audio input from the user. \\
      \textbf{Input:} A sample audio file. \\
      \textbf{Output:} Visual cue(s) on the GUI in 2 seconds or less. \\
      \textbf{Test Case Derivation:} For validation of functional requirement FR-UI1 in section 9.4 of the system requirement specification (SRS). \\
      \textbf{How Test Will Be Performed:}
      \begin{enumerate}
          \item Upload a sample audio file to the application.
          \item Start a timer and detect visual element changes on the GUI using a testing framework (e.g., Jest).
          \item Confirm the following conditions are met:
          \begin{itemize}
              \item The appropriate visual cue is detected.
              \item The elapsed time from upload start to display of visual feedback is at most 2 seconds.
          \end{itemize}
      \end{enumerate}
    
    \item \textbf{Test Availability of System Documentation} \\
      \newline
      \textbf{Test ID:} FR-UI2 \\
      \textbf{Control:} Manual \\
      \textbf{Initial State:} Application is running and idle. \\
      \textbf{Input:} Text-search queries. \\
      \textbf{Output:} Navigation to appropriate documentation/user guide sections. \\
      \textbf{Test Case Derivation:} For use during user testing to meet fit criteria for functional requirement FR-UI2 in section 9.4 of the system requirement specification (SRS). \\
      \textbf{How Test Will Be Performed:}
      \begin{enumerate}
          \item Navigate to the user documentation interface.
          \item Perform a broad text search for major application features or for anything the user requires clarification on.
      \end{enumerate}
    
    \item \textbf{Test User Feedback Report Mechanism} \\
      \newline
      \textbf{Test ID:} FR-UI3 \\
      \textbf{Control:} Automatic \\
      \textbf{Initial State:} Application is running and idle, prepared to receive input through the feedback mechanism. \\
      \textbf{Input:} Plain text. \\
      \textbf{Output:} GUI submission success cue and message, return of the input text. \\
      \textbf{Test Case Derivation:} For validation of functional requirement FR-UI3 in section 9.4 of the system requirement specification (SRS). \\
      \textbf{How Test Will Be Performed:}
      \begin{enumerate}
          \item Submit a plain text report via the user feedback mechanism to the development team (e.g., through email).
          \item Parse through inbox messages for user feedback reports and confirm the reception of the submitted report and entire plain text input.
      \end{enumerate}
  \end{enumerate}
  \subsubsection{Save/Load}
  \begin{enumerate}
    \item \textbf{Test Application Save Function} \\
      \newline
      \textbf{Test ID:} FR-SL1 \\
      \textbf{Control:} Manual \\
      \textbf{Initial State:} Post-audio processing state with generated sheet music file(s) and original audio files accessible for download. \\
      \textbf{Input:} Request to save file(s) to local drive. \\
      \textbf{Output:} Non-corrupt file(s) in destination directories. \\
      \textbf{Test Case Derivation:} For validation of functional requirement FR-SL1 in section 9.5 of the system requirement specification (SRS). \\
      \textbf{How Test Will Be Performed:}
      \begin{enumerate}
          \item Transcribe a sample audio file using the application.
          \item Request to download both the original audio and the generated sheet music to a directory on the local drive.
          \item Compare the contents of the downloaded files to their original copies to check for consistency.
      \end{enumerate}
    
    \item \textbf{Test Application’s Ability to Load Existing Files} \\
      \newline
      \textbf{Test ID:} FR-SL2 \\
      \textbf{Control:} Manual \\
      \textbf{Initial State:} Application running and idle, waiting for user input to modify and/or view. \\
      \textbf{Input:} An existing audio file or existing file containing previously generated sheet music. \\
      \textbf{Output:} Successful loading of files without errors. \\
      \textbf{Test Case Derivation:} For validation of functional requirement FR-SL2 in section 9.5 of the system requirement specification (SRS). \\
      \textbf{How Test Will Be Performed:}
      \begin{enumerate}
          \item Request to load an existing file on the local drive.
          \item Manually inspect the opened file in the editor to ensure the files appear as they were saved.
      \end{enumerate}
  \end{enumerate}
\subsection{Tests for Nonfunctional Requirements}

\wss{The nonfunctional requirements for accuracy will likely just reference the
  appropriate functional tests from above.  The test cases should mention
  reporting the relative error for these tests.  Not all projects will
  necessarily have nonfunctional requirements related to accuracy.}

\wss{For some nonfunctional tests, you won't be setting a target threshold for
passing the test, but rather describing the experiment you will do to measure
the quality for different inputs.  For instance, you could measure speed versus
the problem size.  The output of the test isn't pass/fail, but rather a summary
table or graph.}

\wss{Tests related to usability could include conducting a usability test and
  survey.  The survey will be in the Appendix.}

\wss{Static tests, review, inspections, and walkthroughs, will not follow the
format for the tests given below.}

\wss{If you introduce static tests in your plan, you need to provide details.
How will they be done?  In cases like code (or document) walkthroughs, who will
be involved? Be specific.}

\subsubsection{Area of Testing1}
		
\paragraph{Title for Test}

\begin{enumerate}

\item{test-id1\\}

Type: Functional, Dynamic, Manual, Static etc.
					
Initial State: 
					
Input/Condition: 
					
Output/Result: 
					
How test will be performed: 
					
\item{test-id2\\}

Type: Functional, Dynamic, Manual, Static etc.
					
Initial State: 
					
Input: 
					
Output: 
					
How test will be performed: 

\end{enumerate}

\subsubsection{Area of Testing2}

...

\subsection{Traceability Between Test Cases and Requirements}

For ease of traceability, test cases have been named such that each ID is identical to the ID of 
the functional or nonfunctional requirement it is intended to verify and validate. 

\begin{longtable}{|p{3cm}|p{4cm}|p{8cm}|}
  \hline
  \textbf{SRS Location} & \textbf{Requirement ID} & \textbf{Test IDs} \\
  \hline
  \endfirsthead

  \hline
  \textbf{SRS Location} & \textbf{Requirement ID} & \textbf{Test IDs} \\
  \hline
  \endhead

  \hline
  \endfoot

  \hline
  \endlastfoot

  \multirow{3}{3cm}{Section 9.1} & FR-AR1 & FR-AR1-3-1, FR-AR1-3-2 \\
  \cline{2-3}
   & FR-AR2 & FR-AR2 \\
  \cline{2-3}
   & FR-AR3 & FR-AR1-3-1, FR-AR1-3-2 \\
  \hline
  \multirow{4}{3cm}{Section 9.2} & FR-SP1 & FR-SP1 \\
  \cline{2-3}
   & FR-SP2 & FR-SP2 \\
   \cline{2-3}
   & FR-SP3 & FR-SP3 \\
   \cline{2-3}
   & FR-SP4 & FR-SP4 \\
   \cline{2-3}
   & FR-SP5 & FR-SP2 \\
  \hline
  \multirow{3}{3cm}{Section 9.3} & FR-SMG1 & FR-SMG1 \\
  \cline{2-3}
   & FR-SMG2 & FR-SMG2 \\
  \cline{2-3}
   & FR-SMG3 & FR-SMG3 \\
  \hline
  \multirow{3}{3cm}{Section 9.4} & FR-UI1 & FR-UI1 \\
  \cline{2-3}
   & FR-UI2 & FR-UI2 \\
   \cline{2-3}
   & FR-UI3 & FR-UI3 \\
  \hline
  \multirow{3}{3cm}{Section 9.5} & FR-SL1 & FR-Sl1 \\
  \cline{2-3}
   & FR-SL2 & FR-SL2 \\
  \hline
  \multirow{3}{3cm}{Section 10.1} & NFR & TID, ... \\
  \cline{2-3}
   & NFR & TID, ...\\
  \hline
  \multirow{3}{3cm}{Section 10.2} & NFR & TID, ... \\
  \cline{2-3}
   & NFR & TID, ... \\
  \cline{2-3}
   & NFR & TID, ... \\
  \hline

\end{longtable}

\section{Unit Test Description}

\wss{This section should not be filled in until after the MIS (detailed design
  document) has been completed.}

\wss{Reference your MIS (detailed design document) and explain your overall
philosophy for test case selection.}  

\wss{To save space and time, it may be an option to provide less detail in this section.  
For the unit tests you can potentially layout your testing strategy here.  That is, you 
can explain how tests will be selected for each module.  For instance, your test building 
approach could be test cases for each access program, including one test for normal behaviour 
and as many tests as needed for edge cases.  Rather than create the details of the input 
and output here, you could point to the unit testing code.  For this to work, you code 
needs to be well-documented, with meaningful names for all of the tests.}

\subsection{Unit Testing Scope}

\wss{What modules are outside of the scope.  If there are modules that are
  developed by someone else, then you would say here if you aren't planning on
  verifying them.  There may also be modules that are part of your software, but
  have a lower priority for verification than others.  If this is the case,
  explain your rationale for the ranking of module importance.}

\subsection{Tests for Functional Requirements}

\wss{Most of the verification will be through automated unit testing.  If
  appropriate specific modules can be verified by a non-testing based
  technique.  That can also be documented in this section.}

\subsubsection{Module 1}

\wss{Include a blurb here to explain why the subsections below cover the module.
  References to the MIS would be good.  You will want tests from a black box
  perspective and from a white box perspective.  Explain to the reader how the
  tests were selected.}

\begin{enumerate}

\item{test-id1\\}

Type: \wss{Functional, Dynamic, Manual, Automatic, Static etc. Most will
  be automatic}
					
Initial State: 
					
Input: 
					
Output: \wss{The expected result for the given inputs}

Test Case Derivation: \wss{Justify the expected value given in the Output field}

How test will be performed: 
					
\item{test-id2\\}

Type: \wss{Functional, Dynamic, Manual, Automatic, Static etc. Most will
  be automatic}
					
Initial State: 
					
Input: 
					
Output: \wss{The expected result for the given inputs}

Test Case Derivation: \wss{Justify the expected value given in the Output field}

How test will be performed: 

\item{...\\}
    
\end{enumerate}

\subsubsection{Module 2}

...

\subsection{Tests for Nonfunctional Requirements}

\wss{If there is a module that needs to be independently assessed for
  performance, those test cases can go here.  In some projects, planning for
  nonfunctional tests of units will not be that relevant.}

\wss{These tests may involve collecting performance data from previously
  mentioned functional tests.}

\subsubsection{Module ?}
		
\begin{enumerate}

\item{test-id1\\}

Type: \wss{Functional, Dynamic, Manual, Automatic, Static etc. Most will
  be automatic}
					
Initial State: 
					
Input/Condition: 
					
Output/Result: 
					
How test will be performed: 
					
\item{test-id2\\}

Type: Functional, Dynamic, Manual, Static etc.
					
Initial State: 
					
Input: 
					
Output: 
					
How test will be performed: 

\end{enumerate}

\subsubsection{Module ?}

...

\subsection{Traceability Between Test Cases and Modules}

\wss{Provide evidence that all of the modules have been considered.}
				
\bibliographystyle{plainnat}

\bibliography{../../refs/References}

\newpage

\section{Appendix}

This is where you can place additional information.

\subsection{Symbolic Parameters}

The definition of the test cases will call for SYMBOLIC\_CONSTANTS.
Their values are defined in this section for easy maintenance.

\subsection{Usability Survey Questions?}

\wss{This is a section that would be appropriate for some projects.}

\newpage{}
\section*{Appendix --- Reflection}

The information in this section will be used to evaluate the team members on the
graduate attribute of Lifelong Learning.

\input{../Reflection.tex}

\begin{enumerate}
  \item What went well while writing this deliverable? 
  \item What pain points did you experience during this deliverable, and how
    did you resolve them?
  \item What knowledge and skills will the team collectively need to acquire to
  successfully complete the verification and validation of your project?
  Examples of possible knowledge and skills include dynamic testing knowledge,
  static testing knowledge, specific tool usage, Valgrind etc.  You should look to
  identify at least one item for each team member.
  \item For each of the knowledge areas and skills identified in the previous
  question, what are at least two approaches to acquiring the knowledge or
  mastering the skill?  Of the identified approaches, which will each team
  member pursue, and why did they make this choice?
\end{enumerate}

\end{document}