\documentclass{article}

\usepackage{booktabs}
\usepackage{tabularx}

\title{Development Plan\\\progname}

\author{\authname}

\date{}

\input{../Comments}
%% Common Parts

\newcommand{\progname}{ScoreGen} % PUT YOUR PROGRAM NAME HERE
\newcommand{\authname}{Team \#7, Tune Goons
\\ Emily Perica
\\ Ian Algenio
\\ Jackson Lippert
\\ Mark Kogan} % AUTHOR NAMES                  

\usepackage{hyperref}
    \hypersetup{colorlinks=true, linkcolor=blue, citecolor=blue, filecolor=blue,
                urlcolor=blue, unicode=false}
    \urlstyle{same}
                                


\begin{document}

\maketitle

\begin{table}[hp]
\caption{Revision History} \label{TblRevisionHistory}
\begin{tabularx}{\textwidth}{llX}
\toprule
\textbf{Date} & \textbf{Developer(s)} & \textbf{Change}\\
\midrule
19/09/2024 & Jackson Lippert & Added confidential information, IP to protect, copyright license, and team charter sections\\
Date2 & Name(s) & Description of changes\\
... & ... & ...\\
\bottomrule
\end{tabularx}
\end{table}

\newpage{}

\wss{Put your introductory blurb here.  Often the blurb is a brief roadmap of
what is contained in the report.}

\wss{Additional information on the development plan can be found in the
\href{https://gitlab.cas.mcmaster.ca/courses/capstone/-/blob/main/Lectures/L02b_POCAndDevPlan/POCAndDevPlan.pdf?ref_type=heads}
{lecture slides}.}

\section{Confidential Information?}

There is no confidential information needing protection for our project.

\section{IP to Protect}

There is no IP to protect for our project.

\section{Copyright License}

Our team will be adopting the \href{https://github.com/emilyperica/ScoreGen/blob/main/LICENSE}{MIT License}.

\section{Team Meeting Plan}

\wss{How often will you meet? where?}

\wss{If the meeting is a physical location (not virtual), out of an abundance of
caution for safety reasons you shouldn't put the location online}

\wss{How often will you meet with your industry advisor?  when?  where?}

\wss{Will meetings be virtual?  At least some meetings should likely be
in-person.}

\wss{How will the meetings be structured?  There should be a chair for all meetings.  There should be an agenda for all meetings.}

\section{Team Communication Plan}

\wss{Issues on GitHub should be part of your communication plan.}

\section{Team Member Roles}

\wss{You should identify the types of roles you anticipate, like notetaker,
leader, meeting chair, reviewer.  Assigning specific people to those roles is
not necessary at this stage.  In a student team the role of the individuals will
likely change throughout the year.}

\section{Workflow Plan}

\begin{itemize}
	\item How will you be using git, including branches, pull request, etc.?
	\item How will you be managing issues, including template issues, issue
	classification, etc.?
  \item Use of CI/CD
\end{itemize}

\section{Project Decomposition and Scheduling}

\begin{itemize}
  \item How will you be using GitHub projects?
  \item Include a link to your GitHub project
\end{itemize}

\wss{How will the project be scheduled?  This is the big picture schedule, not
details. You will need to reproduce information that is in the course outline
for deadlines.}

\section{Proof of Concept Demonstration Plan}

What is the main risk, or risks, for the success of your project?  What will you
demonstrate during your proof of concept demonstration to convince yourself that
you will be able to overcome this risk?

\section{Expected Technology}

\wss{What programming language or languages do you expect to use?  What external
libraries?  What frameworks?  What technologies.  Are there major components of
the implementation that you expect you will implement, despite the existence of
libraries that provide the required functionality.  For projects with machine
learning, will you use pre-trained models, or be training your own model?  }

\wss{The implementation decisions can, and likely will, change over the course
of the project.  The initial documentation should be written in an abstract way;
it should be agnostic of the implementation choices, unless the implementation
choices are project constraints.  However, recording our initial thoughts on
implementation helps understand the challenge level and feasibility of a
project.  It may also help with early identification of areas where project
members will need to augment their training.}

Topics to discuss include the following:

\begin{itemize}
\item Specific programming language
\item Specific libraries
\item Pre-trained models
\item Specific linter tool (if appropriate)
\item Specific unit testing framework
\item Investigation of code coverage measuring tools
\item Specific plans for Continuous Integration (CI), or an explanation that CI
  is not being done
\item Specific performance measuring tools (like Valgrind), if
  appropriate
\item Tools you will likely be using?
\end{itemize}

\wss{git, GitHub and GitHub projects should be part of your technology.}

\section{Coding Standard}

\wss{What coding standard will you adopt?}

\newpage{}

\section*{Appendix --- Reflection}

\wss{Not required for CAS 741}

\input{../Reflection.tex}

\begin{enumerate}
    \item Why is it important to create a development plan prior to starting the
    project?
    \item In your opinion, what are the advantages and disadvantages of using
    CI/CD?
    \item What disagreements did your group have in this deliverable, if any,
    and how did you resolve them?
\end{enumerate}

\newpage{}

\section*{Appendix --- Team Charter}

\subsection*{External Goals}

Our team’s primary external goal for this project is to maximize learning and skill development in software engineering. We are focused on creating a technically impressive and innovative project to earn the highest possible grade. Additionally, we want to ensure our project can be showcased in interviews and added to our professional portfolios and résumés. These goals will help us stand out when pursuing future opportunities in the tech industry.

\subsection*{Attendance}

\subsubsection*{Expectations}

Team members are expected to attend all scheduled meetings on time within reason. Leaving early is acceptable if communicated in advance and if that member has completed their contributions for the meeting. Additionally, all team members may leave if they feel the meeting’s agenda has been completed. Missing meetings should be a rare occurrence, and any absences must be communicated at least 24 hours in advance when possible.


\subsubsection*{Acceptable Excuse}

Acceptable excuses for missing a meeting or deadline include illness, family emergencies, or unavoidable academic conflicts (exams or critical project deadlines). Unacceptable excuses include forgetfulness, oversleeping, or personal plans that were not communicated ahead of time. All excuses must be communicated as soon as possible to the rest of the group.

\subsubsection*{In Case of Emergency}

If a team member has an emergency and cannot attend a meeting or complete their work, they should notify the team as soon as possible via the agreed-upon communication method. If possible, the team member should provide updates on their progress and share any work that has been completed so far, so others can step in if necessary. In case of an emergency during critical project milestones, the team will redistribute tasks to ensure deadlines are met.

\subsection*{Accountability and Teamwork}

\subsubsection*{Quality} 

All deliverables should be thoroughly tested, documented, and meet the project’s technical and design standards before being submitted to the team. \@ If a team member encounters difficulties, they should raise a GitHub issue for discussion before the meeting to avoid delays and ensure early feedback.\@ Pull requests raised to change code will be reviewed by at least one other team member, ensuring only quality code is being deployed via CI. Finally, all team members will look over and edit each others’ work for documentation, ensuring everything is clear to the reader.

\subsubsection*{Attitude}

We will foster a positive, supportive, and professional team environment where everyone’s ideas are heard and respected. During discussions, we will prioritize constructive feedback and aim to solve problems together. In every interaction, we will emphasize respect, inclusion, and clear communication. Any conflicts will be addressed through open discussions. If unresolved, a neutral team member may mediate, or the issue will be escalated to the TA.

\subsubsection*{Stay on Track}

For progress tracking we will use GitHub’s issue tracking and Kanban board features to manage tasks, ensuring that all assignments are transparent and visible to the team. Each team member will update the status of their tasks by moving issues across the Kanban board into their respective swim lanes. GitHub commits should reference corresponding issues, ensuring that work is traceable. A meeting chair is appointed to keep meetings on schedule and ensure the meeting agenda is followed.

\subsubsection*{Team Building}

To strengthen our teamwork, members are encouraged to engage with each other informally outside the scope of this project. As a team ritual, we will celebrate milestones by acknowledging successes and accepting every failure gracefully and as a team.

\subsubsection*{Decision Making} 

We will aim for consensus in decision-making. If a consensus cannot be reached within a set timeframe, we will hold a majority vote. For major disagreements, we will discuss the pros and cons of each option and if needed, consult a TA, our supervisor, or the professors of the course.

\end{document}